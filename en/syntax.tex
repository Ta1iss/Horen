\nchapter{Syntax}

\section{Transitivity and Ergativity}

\subsection{Transitivity} Na'vi marks the subject of transitive and
intransitive verbs differently.  To speak any Na'vi sentence
with a verb requires one to understand transitivity.  This means a
deeper and earlier understanding of transitivity is required for Na'vi
than is usually required to learn a Human language.\footnote{Since
formal grammar isn't taught as much as it used to be, some English
speakers have trouble identifying transitive vs.\ intransitive
verbs.  This is further complicated by English grammar, where very
often it's not the verb that is transitive or intransitive, but the
entire phrase.  For example, ``I move'' is intransitive while ``I
move the car'' is transitive, and only the presence of a direct
object triggers the transitive interpretation of the verb.  In
Na'vi, too, it is usually best to think about transitivity as a
clausal, rather than purely verbal, phenomenon.

Here are two quick ways to test for transitivity in English.  First,
if the word immediately after the verb is a person or thing.  So, in
``I see the moon'' the verb is transitive, but in ``he complains
constantly'' the verb is not.  The second test, in case you're
uncertain about what's coming after the verb, is to see if you can
turn the thing after the verb into a passive sensibly.  So, ``The moon
is seen'' is a well-formed passive, while ``constantly is complained''
is gibberish.}\index{transitivity}

\subsubsection{} Many compound verbs are created by pairing an
uninflected noun or adjective, or occasionally an interjection,
with the verb stem \N{si} \E{to do, to make}, which is only used 
in these compounds, \N{irayo si} \E{to thank}, \N{kavuk si} \E{to betray}.
These verbs are always intransitive and use the dative for any object
(\horenref{syn:case:dative-si}).

\subsubsection{} Reflexive verbs with the \N{\INF{äp}} infix are
always intransitive, and causative verbs with the \N{\INF{eyk}} infix
are always transitive.

\subsubsection{Reflexive of Causative}
\label{reflexive-of-causative} \index{reflexive!of a causative}
Verbs that are a reflexive of a causative,
taking, \N{\INF{äp}\INF{eyk}}, are more complex, depending on the
transitivity of the original expression:

\begin{itemize*}
  \item If the original construction is intransitive, \N{\INF{äp}}
    dominates, and the doer is in the subjective case. 
  \item If the original construction is transitive, \N{\INF{eyk}}
    dominates, and the doer is in the agentive case. 
\end{itemize*} 

\noindent For example,

\begin{quotation}
\noindent\Npawl{Oe täpeykaron.} \E{I cause myself to hunt.}\\
\noindent\Npawl{Oel täpeykaron yerikit.} \E{I cause myself to hunt yerik.}
\end{quotation}

\noindent This pattern parallels the verb without any transitivity
changes at all, \N{oe taron} is to \N{oel taron yerikit} as \N{oe
täpeykaron} is to \N{ole täpeykaron yerikit}.
\LNForum{20/8/2021}{https://forum.learnnavi.org/language-updates/ltapeykgt-in-transitive-constructions/}

\subsection{Tripartite} Na'vi marks nouns and pronouns differently if
they are the subject of an intransitive verb, the subject of a
transitive verb or the direct object of a transitive verb
(\horenref{syn:cases}).

\subsubsection{} Though the English concept of the ``subject'' of a
verb in Na'vi is divided in two de\-pend\-ing on the transitivity of
the verb phrases, this division does not apply to participles.  There
is a verbal object adjective (the passive participle) and a verbal
subject adjective (the active participle) which is used for both
subjective and agentive subjects (\horenref{morph:pre-first}).

\subsubsection{} \QUAESTIO{Na'vi is also pragmatically split-ergative.
In connected discourse one may drop the subject pronoun if it
doesn't change.  The subject may be either subjective or agentive.
See \textit{some pragmatics section}.}


\section{Noun Phrases and Adjectives}

\subsection{Number} \QUAESTIO{Are the dual and trial collective
vs. plural distributive?  Or always obligatory?}

\subsubsection{} When used with an attributive numeral, nouns are not
marked for number, \N{mrra zìsìt} \E{five years}.
\index{plural!unmarked with numerals}
\LNWiki{18/6/2010}{https://wiki.learnnavi.org/index.php/Canon/2010/March-June\%23Numbers_take_nouns_in_the_SINGULAR.}

\subsubsection{} The adjectives of quantity --- \N{'a'aw} \E{several},
\N{hol} \E{few}, \N{pxay} \E{many}, \N{polpxay, holpxaype} \E{how
many?} --- also take singular nouns in attributive phrases, \Npawl{lu
poru \uwave{'a'awa 'eylan}} \E{he has \uwave{several friends}}.
\index{'a'aw@\textbf{'a'aw}}\index{hol@\textbf{hol}}\index{pxay@\textbf{pxay}}
\index{polpxay@\textbf{polpxay}}\index{holpxaype@\textbf{holpxaype}}
\NTeri{16/7/2010}{https://naviteri.org/2010/07/vocabulary-update/}

\subsubsection{} In colloquial speech, number may be marked with the
adjective \N{pxay} \E{many}, \Npawl{lu awngar \uwave{aytele apxay} a
teri sa'u pivlltxe} \E{we have \uwave{many matters} to talk about}.
\index{pxay@\textbf{pxay}!with plural noun}
\LNForum{16/7/2010}{https://forum.learnnavi.org/?msg=123484}


\subsubsection{} With verbs of identity (\N{lu} and \N{slu}),
the basic rule of number marking in Na'vi is, ``in referring to the
same entity, express number only once per clause.''\label{syn:noun:concord}

\begin{quotation}
\noindent\Npawl{Menga lu karyu.} \E{You two are teachers}.\\
\noindent\Npawl{Fo lu karyu.} \E{They are teachers}.\\
\noindent\Npawl{Menga lu oeyä 'eylan.} \E{You two are my friends}.
\end{quotation}
\index{lu@\textbf{lu}!number agreement}\index{slu@\textbf{slu}!number agreement}
\index{plural!with \textbf{lu} and \textbf{slu}}

\noindent In the first two sentences, \N{karyu} is not marked for
number since the pronouns are already marked, and the same for
\N{'eylan} in the third sentence.  But see \horenref{syn:pron:q-number}
for the question pronoun \N{tupe}.
\NTeri{30/7/2011}{https://naviteri.org/2011/07/number-in-na\%E2\%80\%99vi/}

\subsubsection{} General statements about a group or class use nouns
in the singular, \index{general statements} 

\begin{interlin}
\glll Nantangìl yom yerikit. \\
      nantang-ìl yom yerik-it \\
      viperwolf-\I{agt} eat hexapede-\I{pat} \\
\trans{Viperwolves eat hexapedes.} \Ipawl{}
\end{interlin}

\NTeri{30/7/2011}{https://naviteri.org/2011/07/number-in-na\%E2\%80\%99vi/}

\subsection{Indefinite} The adjective \N{lahe} \E{other} has the sense
of \E{else} when used with indefinite nouns having the suffix \N{-o},

\begin{interlin}
\glll Lu law \uwave{'uo alahe}, ma eylan. \\
      lu law 'u-o a-lahe, ma eylan \\
      be clear thing-some \I{lig}-other, \I{voc} friends \\
\trans{\uwave{Something else} is clear my friends.} \Ipawl{}
\end{interlin}

\index{indefinite noun}\index{lahe@\textbf{lahe}!with indefinite nouns}

\subsection{Free Choice Indefinites} Na'vi uses the adjective
\N{ketsran} \E{no matter (what), whatever} with generic nouns to
create free choice indefinites.  It can be used like an adjective,
ataching with \N{-a-} (ex.\ref{ketsran:ex01}, \ref{ketsran:ex02}), or
it can be a conjunction (ex.\ref{ketsran:ex03},\ref{ketsran:ex04}).
The clause with \N{ketsran} often, though not always, takes the
subjunctive,
\index{indefinites!free choice}\index{ketsran@\textbf{ketsran}}

\begin{interlin} \label{ketsran:ex01}
\glll 'U aketsran tsun tivam. \\
     'u a-ketsran tsum t\INF{iv}am \\
     thing \I{lig}-whatever can suffice‹\I{subj} \\
\trans{Anything at all will be fine}. \Ipawl{}
\end{interlin}

\begin{interlin} \label{ketsran:ex02}
\glll Pukit aketsran ivinan. \\
    puk-it a-ketsran \INF{iv}inan \\
    book-\I{pat} \I{lig}-whatever read‹\I{subj} \\
\trans{Read any book at all.} \Ipawl{}
\end{interlin}

\begin{interlin} \label{ketsran:ex03}
\glll Ketsran tute nivew hivum, poru plltxe san rutxe 'ivì'awn. \\
ketsran tute n\INF{iv}ew h\INF{iv}um, po-ru plltxe san rutxe '\INF{iv}ì'awn\\
whatever person want‹\I{subj} leave‹\I{subj}, \I{3sg-dat} tell \I{quot} please stay‹\I{subj} \\
\trans{No matter who wants to leave, tell them to please stay.} \Ipawl{}
\end{interlin}

\begin{interlin} \label{ketsran:ex04}
\glll Ketsran tutel 'ivem, tsafnetsngan lu ftxìvä'. \\
   ketsran tute-l '\INF{iv}em, tsa-fne-tsngan lu ftxìvä' \\
   whatever person-\I{agt} cook‹\I{subj}, that-kind-meat be gross \\
\trans{That kind of meat is gross no matter who cooks it.} \Ipawl{}
\end{interlin}


\noindent\NTeri{3/31/2013}{https://naviteri.org/2013/03/whoever-whatever-whenever/}

\subsection{Apposition} Nouns in apposition\footnote{Nouns are
described as \textit{in apposition} when they occur immediately next
to each other, with one describing or defining the other.  In English,
we usually set off the second noun phrase with commas, as in \E{I told
my best friend, Bob, that he should learn Na'vi, too.}  The name
``Bob'' is in apposition to ``my friend.''}
to other nouns are in the subjective case, \Npawl{ay\-lì'u\-fa
\uwave{awngeyä 'eylanä a'ewan Markusì}} \E{in the words \uwave{of
our young friend, Marcus}}.  However, the con\-junct\-ion \N{alu} is
also used for this (see \horenref{syn:conj:alu}).\footnote{The bare
apposition is Early Na'vi.  Using \N{alu} may be better for future
use.} \index{apposition}

\subsubsection{Titles} Titles act as noun modifiers, and are thus not
declined when used with proper names.  The dative of \N{Karyu Pawl}
``teacher Paul'' is \N{Karyu Pawlur}. \index{case!with titles}
% https://forum.learnnavi.org/language-updates/definitive-answers-on-compound-nouns/
% Followup and details: https://forum.learnnavi.org/language-updates/compound-nounsfinal-decision!/

\subsection{Adjective Attribution} Attributive adjectives are joined
to the noun they modify with the affix \N{-a-} (see
\horenref{morph:adj-attr}), \Npawl{sìlpey oe, layu oeru ye'rìn
\uwave{sìltsana fmawn}} \E{I hope I will soon have \uwave{good
news,}} \Npawl{lora aylì'u, lora aysäfpìl} \E{beautiful words and
beautiful thoughts.} \index{adjective!attributive}\label{syn:adj:attr}

\subsubsection{} Regardless of the order of noun and adjective, the
case endings always attach to the noun, never the adjective.
Similarly, an enclitic adposition is always attached to the noun
(see \horenref{syn:adp:position}).

\subsubsection{} When an adverb is used with an attributive adjective,
it must not come between the adjective and its noun, that is,
\Npawl{sìkenong ahìno nì\-hawng} \E{very detailed examples} or
\N{nìhawng hìnoa sìkenong}, never something like \N{*hìno nì\-hawng\-a
sìkenong}.
\index{adjective!attributive!with adverb}

\subsubsection{} If there are two adjectives modifying a noun, Frommer
has a tendency to order them Adj - N - Adj, \Npawl{nìawnomum tolel oel
ta ayhapxìtu lì’fyaolo’ä \uwave{pxaya sìpawmit atxantsan}} \E{as you
know, I have received \uwave{many excellent questions} from members
of the language community}. 

\subsubsection{} For more than two adjectives, or to use some order
other than  Adj - N - Adj given above, the adjectives can be put
into an attributive clause with \N{lu},

\begin{interlin}
\glll yayo a lu lur sì hì'i \\
      yayo a lu lur sì hì'i \\
      bird \I{rel} be pretty and small \\
\trans{a small, pretty bird} \Ipawl{}
\end{interlin}

\noindent However, \N{yayo alur sì hì'i} (without \N{lu}) is permitted,
though not preferred.  This is much more likely and acceptible when a
third adjective is involved in the clause,

\begin{interlin}
\glll mrra yayo atsawl sì layon \\
      mrr-a yayo a-tsawl sì layon \\
      five-\I{lig} bird \I{lig}-big and black \\
\trans{five big, black birds} \Ipawl{}
\end{interlin}

\noindent Without \N{mrr-a} there, \N{tsawla yayo alayon} is preferred. 
\Ultxa{2/10/2010}{https://wiki.learnnavi.org/index.php/Canon/2010/UltxaAyharyu\%C3\%A4\%23Multiple_Attributives}
\NTeri{28/2/2021}{https://naviteri.org/2021/02/aysipawm-si-aysieyng-questions-and-answers/\#comment-32798}

\subsubsection{} A single adjective may be repeated on both sides of
the noun to mark intensity.  The second adjective receives the phrase
stress, \Npawl{lu po lora tuté \uwave{alor}} \E{she's an extremely
beautiful woman.}
\NTeri{2/28/2013}{https://naviteri.org/2013/02/vospxi-ayol-posti-apup-short-post-for-a-short-month/}

\subsubsection{} When repeating a noun with different adjectives
(``the big dog, the little dog, the yappy dog,'' etc.) the prop
noun\footnote{Frommer calls it a ``dummy noun,'' but it can be
reasonably thought of as a kind of pronoun, too.} \N{pum} is used
for the repetitions, \Npawl{lam set fwa Sawtute akawng holum,
\uwave{pum asìltsan} 'ì'awn} \E{it now seems that the evil
sky-people are gone, \uwave{the good ones} remain}.
However, if a more specialized general term is available it is more
elegant to use that.  In answer to \Npawl{polpxaya taronyu kelku si
tsatsraymì?} \E{how many hunters live in that village?} the more
elegant reply is \N{tute amevol} \E{sixteen people,} though \N{pum
amevol} is still perfectly possible.
\label{syn:pum:adj} \index{pum@\textbf{pum}!with attributive adjectives}
\LNForum{30/5/2020}{https://forum.learnnavi.org/language-updates/correspondence-with-karyu-pawl/}

\subsubsection{} The noun element in most \N{si} constructions verb may
have an attributive adjective, \N{wina uvan si} \E{play a quick game}.
\index{si construction@\textbf{si} construction!with attributive adjective}
\LNForum{6/12/2013}{https://forum.learnnavi.org/language-updates/si-verbs-modification/}

\subsection{Predication} Adjective and noun predicates both use the
same construction with the verb \N{lu} \E{be}, as in
\Npawl{\uwave{reltseotu atxantsan lu} nga} \E{you are an excellent
artist}, \Npawl{fìsyulang lu rim} \E{this flower is yellow}.
\index{adjective!predication}\index{noun!predication}\label{syn:predicates}

\subsubsection{} Other verbs that take predicative syntax: \N{slu}
\E{become} and \N{'efu} \E{feel}, such as in \N{ngenga slìyu Na'viyä
hapxì} \E{you will become part of the people}, \N{oe 'efu ohakx} \E{I
am (feel) hungry}.
\index{slu@\textbf{slu}!predicate syntax}\index{'efu@\textbf{'efu}!predicate syntax}

\subsubsection{} If there is ambiguity with \N{slu} \E{become} about
which constituent is the subject and which the predicate, the
predicate can be marked with the adposition \N{ne}, as in \N{taronyu
slu \uwave{ne tsamsiyu}} \E{the hunter becomes \uwave{a warrior}}.
\label{syn:predicate:slu-ne}\index{ne@\textbf{ne}! with \textbf{slu}}
\index{slu@\textbf{slu}!predicate syntax with \textbf{ne}}
\LNWiki{2/10/2010}{https://wiki.learnnavi.org/Canon/2010/UltxaAyharyuä\#Word_Order_with_Slu}

\subsubsection{} \N{Sleyku}, the causative of \N{slu} \E{become}, also
takes an adjective predicate, \Npawl{fula tsayun oeng pivängkxo ye'rìn
ulte ngari oel mokrit stayawm, \uwave{oeti nit\-ram sleyku nìtxan}}
\E{\uwave{it makes me very happy} that we two will soon be able to
chat and that I will hear your voice}.  \QUAESTIO{What about \N{'eykefu}?}
\index{sleyku@\textbf{sleyku}!predicate syntax}

\subsection{Comparison} Adjective comparatives and superlatives (\E{big,
bigger, big\-gest}) are mark\-ed with the particle \N{to}, which, like an
adposition, may come before the noun compared to or be enclitic on it
(\horenref{l-and-s:stress:enclisis}).
\index{to@\textbf{to}}\index{adjective!comparative}
\index{comparison!of adjectives}

\begin{quotation}
\noindent\N{Oe \uwave{to nga} lu koak} \E{I am older \uwave{than you}}.\\
\noindent\N{Oe \uwave{ngato} lu koak} \E{I am older \uwave{than you}}.
\end{quotation}

\subsubsection{} The superlative is handled with \N{\ACC{fra}to}
\E{than all}, \Npawl{fìsyulang arim lu hì’i frato} \E{this yellow
flower is the smallest of all}. \index{frato@\textbf{frato}}
\index{adjective!superlative}

\subsubsection{} Comparisons of equality, ``as big as a tree,'' are
handled with the idiom \N{nìftxan} \E{adjective} \N{na} \E{noun or
pronoun}, as in \Npawl{oe lu nìftxan sìltsan na nga} \E{I am as good
as you.}  If the point of comparison is a pronoun, or definite noun
already part of the discourse, the topical case may be used,
\Npawl{ngari lu oe nìftxan sìltsan}.  This construction is also usable
with adverbs.  \label{syntax:adj-eql-comp}
\index{adjective!equal comparison}\index{nìftxan@\textbf{nìftxan}}
\index{case!topical!point of comparison}
\index{na@\textbf{na}!point of comparison}
\LNWiki{1/12/2010}{https://wiki.learnnavi.org/index.php/Canon/2010/October-December\#As_ADJ.2FADV_as_N.2FPRN}
%% CITE: https://wiki.learnnavi.org/index.php/Canon/2010/October-December#As_ADJ.2FADV_as_N.2FPRN


\subsection{Direct Address}
\index{vocative}\index{direct address}\index{ma@\textbf{ma}}
When speaking to an individual directly the vocative particle \N{ma}
comes immediately in front of the relationship noun, noun phrase or
name, \Npawl{oel ayngati kameie, ma oeyä eylan} \E{I see you, my
friends,} \Nfilm{ma Tsu'tey, kempe si nga?} \E{Tsu'tey, what are you
doing?}.

\subsubsection{} With polar questions, the vocative follows the
final \N{srak}, as in \N{ngaru lu fpom srak, ma Txewì?} \E{how are
you, Txewì?}  And while it is most common for the vocative to come at
the beginning or end of a clause, it may occur within a
clause, \N{nga, ma Neytiri, plltxe nìltsan} \E{you, Neytiri, speak
well}.
\LNWiki{26/2/2018}{https://wiki.learnnavi.org/Canon/2018}

\subsubsection{} If multiple individuals are addressed \N{ma} is not
repeated, \N{ma smukan sì smuke} \E{brothers and sisters}.

\subsubsection{}
\index{vocative!with animals}\index{direct address!with animals}
\index{ma@\textbf{ma}!with animals}
The vocative is obligatory when speaking to people (and Eywa), but
optional when talking to animals.
\LNForum{6/4/2010}{https://forum.learnnavi.org/language-updates/info-on-duals-and-vocative/}

\subsubsection{} Collective nouns may take the suffix \N{-ya}, as in
\Nfilm{mawey, Na'viya, mawey} \E{(be) calm, people, (be) calm!}
\index{-ya@\textbf{-ya}!vocative}


\section{Pronouns}

%\subsection{Animacy}

\subsection{Gender} The gendered third person pronouns, \N{poan} and
\N{poe}, are used only when it will help to avoid ambiguity in
discourse.  Speakers of English and other Western European languages
should take care to not use them too often. \label{syn:pron:gender}

\subsection{Number} The forms of the question pronoun \N{tupe} have
a behavior that differs from the number agreement rules discussed
in \horenref{syn:noun:concord}. \label{syn:pron:q-number}
Here, the pronoun may be marked for number even when the noun has
been, too.  Note the answers to these questions,

\begin{quotation}
\noindent\Npawl{Tsaysamsiyu lu \uwave{tupe}?} \E{Who are those warriors?}\\
\noindent\N{(Fo) lu 'eylan Tsu’teyä.} \E{They are Tsu'tey's friends.}\\

\noindent\N{Tsaysamsiyu lu \uwave{supe}?} \E{Who are those warriors?}\\
\noindent\N{(Fo) lu Kamun, Ralu, Ìstaw, sì Ateyo.}\\
\indent\E{They're Kamun, Ralu, Ìstaw, and Ateyo.}
\end{quotation}

\noindent The plural forms ask for the identity of individual members,
while the singular asks about a group characteristic.
% https://naviteri.org/2011/07/number-in-na’vi/

\subsection{Similarity} Pronouns may take the adverbial prefix
\N{nì-}, producing a form like \N{nìnga} \E{like you}. These forms
are used to indicate way of acting, \Npawl{plltxe po nìayoeng} \E{she
speaks like us} or \E{as we do}. For describing how someone is
perceived, the the adpositions \N{na} or \N{pxel} are preferred. 
\LNForum{16/8/2016}{https://forum.learnnavi.org/language-updates/eapressions-of-'like-we'/}

\subsection{Fko} The indefinite pronoun \N{fko} is like the English
pronoun ``one'' or the less formal ``you'' in the general sense, as in
\E{one doesn't say such things} vs. \E{you don't say things like
that}. \Npawl{Tsat ke tsun fko yivom} \E{you can't eat that;}
\Npawl{tsun fko sivar hänit fte payoangit stivä'nì} \E{one can use a
net to catch a fish}.
\index{fko@\textbf{fko}}

\subsubsection{} \N{Fko} is also used where English would use an
unspecified ``they'' when making general statements, as in
\Npawl{\uwave{plltxe fko} san ngaru lu mowan Txilte ulte poru nga}
\E{They say you like Txilte and vice versa}.\footnote{Frommer's
translation of this is \E{I hear you like Txilte and vice versa}.}

\subsubsection{} \N{Fko} can be used for the English passive when the
agent of the verb\footnote{The agent of a passive verb is the person
or thing you put with the preposition ``by'' in English, as in \E{I
was hit \uwave{by a car}}.} is animate, as in the idiom \N{oeru syaw fko
Wìlyìm} \E{my name is William, I am called William,} \Npawl{tsalì’uri
fko pamrel si fyape?} \E{how is that word written, how does one write
that word?}
\label{syn:prn:fko}
\index{passive!with \textbf{fko}}\index{fko@\textbf{fko}!for English passive}

\subsection{Sno} \index{sno@\textbf{sno}}
The reflexive pronoun \N{sno} refers to the subject or agent of the
clause it occurs in.  In the genitive (\N{sneyä}) it may be
translated \E{his own, her own, their own, etc.}  It is used to clear
up situations found in a sentence like ``he prepared his meal.''
Without clarification, it may not be clear if ``his'' refers to the
person preparing the dinner or someone else:

\begin{quotation}
\noindent\N{Pol 'olem peyä wutsot.} \E{He prepared his (someone else's) meal.}\\
\noindent\N{Pol 'olem sneyä wutsot.} \E{He prepared his own meal.}
\end{quotation}

\noindent \N{Sno} can refer back to a topical acting as the subject,

\begin{quotation}
\noindent\N{Skxawngìri zìmup ulte sneyä tsko kxakx.}\\
\indent\E{The idiot fell and broke his/her (own) bow.}
\end{quotation}

%\noindent If the topical is not the subject of the clause, \N{sno}
%will refer instead to the subject:
%
%\begin{quotation}
%\noindent\N{Skxawngìri sa'nok zìmup ulte sneyä tsko kxakx.}\\
%\indent\E{That idiot's mother fell and broke her (own) bow}.
%\end{quotation}
%\LNForum{27/11/2020}{https://forum.learnnavi.org/language-updates/sno-with-topical/}

\QUAESTIO{As of Dec 2020 there remain questions about how to
resolve \N{sneyä}, etc., when a topical and subject are both in play,
as well as whether or not \N{sneyä} can refer to things in an outer
clause, \E{he thinks that \uwave{his} father...}.}

Note that the reflexive can anticipate the noun it refers to, as
in \Npawl{sìpawmìri sneyä aynumeyuä karyu 'eyng} \E{the teacher
responds to his students' questions}.  Here \N{sneyä} is coming
before \N{karyu} \E{teacher}.
% https://naviteri.org/2020/05/tipusawm-tiuseyng-si-okvur-a-eltur-titxen-si-asking-answering-and-an-interesting-story/

\N{Sno} is for third person antecedents only.
\LNWiki{23/1/2018}{https://wiki.learnnavi.org/Canon/2018}


\subsection{Lahe} The adjective \N{lahe} \E{other, another} can also
be used alone as a pronoun, \Nfilm{fìpoti oel tspìyang, fte tìkenong
liyevu \uwave{aylaru}} \E{I will kill this one as a lesson to the
\uwave{others}} (see \horenref{morph:lahe:dat-pl} for the form).
\index{lahe@\textbf{lahe}!as pronoun}

\subsection{PRO-Drop} \index{PRO-drop}
A subject pronoun (either subjective or agentive) may be dropped if it
is the same as the subject of the previous statement.  Note the lack
of a subject pronoun in the second sentence:


\begin{quotation}
\noindent\Npawl{Fayupxaremì \uwave{oe} payängkxo teri horen lì’fyayä
leNa’vi fpi sute a tsun srekrr tsat sivar. Ayngeyä sìpawmìri kop
fmayi fìtsenge tivìng sì’eyngit.}

\medskip
\noindent\E{In these messages \uwave{I} will chat about the rules of
the Na'vi language for people who can already use it.  \uwave{I}
will also try to give answers here concerning your questions.}
\end{quotation}

\subsection{Contrastive Demonstratives}\index{demonstrative!contrastive}
To focus contrasting elements, forms of the prenouns \N{fì-} and
\N{tsa-} are paired with forms of the independent demonstratives
\N{fì'u} and \N{tsa'u} used with \N{alu}:
\index{alu@\textbf{alu}!with contrastive demonstratives}

\begin{quotation}
\noindent\Npawl{Fìfkxen alu FÌ'u lu ftxìlor; tsafkxen\footnote{Or
    \N{pum}.} alu TSA'u ngati tspang.}\\
\noindent\E{THIS vegetable is delicious; THAT one will kill you.}\\

\noindent\Npawl{Fìkaryu alu fìpo lu tsulfätu; tsakaryu alu tsapo lu skxawng.}\\
\noindent\E{This teacher is a master; that teacher is a fool.}
\end{quotation}

\noindent There is also a vocal constrastive stress on the independent
forms of \N{fì'u} and \N{tsa'u} in this con\-struc\-tion.
\NTeri{31/12/2011}{https://naviteri.org/2011/12/one-more-for-2011/}


\section{Use of the Cases}
\label{syn:cases}
\subsection{Subjective} The unmarked subjective case is used as the
subject of intransitive verbs, the predicate noun in predicate
constructions (\horenref{syn:predicates}) and with
adpositions. \index{case!subjective}\index{subject}

\subsubsection{} With verbs of motion, if the destination comes
immediately after the verb, the ad\-posi\-tion \N{ne} may optionally be
dropped, leaving an unmarked noun, \Npawl{za’u \uwave{fìtseng}, ma
’itetsyìp} \E{come \uwave{here}, little daughter}. \label{syn:subjective:ne}
\index{ne@\textbf{ne}!omitted with verbs of motion}

\subsubsection{} The subjective is also used in exclamations, when a
noun or noun phrase is used by itself as an utterance, \Npawl{lora
aylì'u, lora aysäfpìl} \E{beautiful words and beautiful thoughts,}
\Npawl{aylì’u apawnlltxe nìltsan} \E{words well spoken!}
\index{case!subjective!exclamatory}

\subsubsection{} A time word with the indefinite \N{-o} is used in the
subjective to indicate a duration of time, \Nfilm{\uwave{zìsìto amrr}
ftolia ohe} \E{I studied \uwave{for five years}}, \Npawl{herwì zereiup
\uwave{fìtrro nìwotx}!} \E{It's been snowing \uwave{all day}!}
\index{-o@\textbf{-o}!in time expressions}
\LNWiki{1/12/2010}{https://wiki.learnnavi.org/index.php/Canon/2010/October-December\#Duration_and_Loan_Word.2C_.22Jesus.22}

\subsection{Agentive} The agentive case is used for the subject of
transitive verbs, \N{\uwave{oel} ngati kameie} \E{I See you}.
\index{case!agentive}\index{subject}

\subsection{Patientive} The patientive is used as the direct object of
transitive verbs, \Npawl{\uwave{tì'eyngit} oel tolel a krr}
\textit{when I receive an answer}.
\index{case!patientive}\index{direct object}

\subsection{Dative} The dative is used for the indirect object of
ditransitive verbs, \Npawl{sìltsana fmawn a tsun oe \uwave{ayngaru}
tivìng} \E{good news which I can give \uwave{to you}}.
\index{case!dative}\index{indirect object}

\subsubsection{} The object of a \N{si}-verb takes the dative, \N{oe
irayo si ngaru} \E{I thank you.} \label{syn:case:dative-si}

\subsubsection{} The causee for the causative of a transitive verb may
be in the dative, \N{oel \uwave{ngaru} tseyk\-ìy\-e'a tsat} \E{I will make
\uwave{you} see it} (see \horenref{syn:trans-causative}). \index{case!dative!with causative}

\subsubsection{} The verb \N{lu} with the dative forms an idiom for
possession, where English uses the verb ``have,'' \N{lu oeru ikran}
\E{I have an ikran}.  In this construction the verb usually comes
first in the clause.  \index{case!dative!with \textbf{lu}}
\index{lu@\textbf{lu}!with dative}
\LNWiki{28/1/2010}{https://wiki.learnnavi.org/index.php/Canon\%23Dative_.2B_copula_possessive}

\subsubsection{} The dative of interest limits the scope of an
adjective to the judgement \QUAESTIO{or benefit} of a particular
individual, \N{fì'u oeru prrte' lu} \E{this is pleasant to me},
\N{tìpängkxo ayoengeyä mowan lu oeru nìngay} \E{our chat is truly
enjoyable (to me)}.

\subsubsection{} With verbs of speaking, including a word like
\N{pawm} \E{ask,} the person addressed goes in the dative, \N{oel poru
polawm fì'ut} \E{I asked him this}.
\index{case!dative!with verbs of speaking}

\subsection{Genitive} The genitive case marks possession, as in
\N{oeyä 'eylan} \E{my friend}.\index{case!genitive} But see below for
inalienable possession (\horenref{syn:topical:poss}).\index{possessive}

\subsubsection{} The genitive can be used predicatively, as in
\N{fìtseng lu awngeyä} \E{this place is ours}.  How\-ever, the prop
noun \N{pum} \E{possession, thing possessed} is more often used,
\Npawl{kelku ngeyä lu tsawl; \uwave{pum oeyä} lu hì’i} \E{your house
is large; \uwave{mine} is small}. \label{syn:pum:genitive}
\index{pum@\textbf{pum}!with genitive}

\subsubsection{} The partitive genitive marks the larger whole of
which something is part, \Nfilm{Na'viyä luyu hapxì} \E{you are part
of the people}.  This is also used with fractions, \Npawl{Tsu'teyìl
tolìng oer mawlit \uwave{smarä}} \E{Tsu'tey gave me a half \uwave{of
the prey}}. 
\index{case!genitive!partitive}\label{syn:partitive-gen}

\subsubsection{} The genitive is occasionally separated from the noun
phrase it goes with, \Nfilm{Na'viyä luyu hapxì} \E{you are part of the
people}.\index{case!genitive!dislocation}

\subsubsection{} The genitive is also used as the object of verbal
nouns, as in \Npawl{tìftia kifkeyä} \E{study of the natural world}.

\subsection{Topical} The topical case marks the topic in a
topic-comment construction.  See \textit{Topic-Comment},
\horenref{pragma:topic-comment}, for a longer discussion of this use.
The topical has a few more fixed uses, as well.\index{case!topical}

\subsubsection{}\index{case!topical!word order} \label{syn!topical!word-order}
\index{case!topical!with \textbf{srake}}
In prose, a topical noun phrase will come as early in the clause as
possible: first in a main clause, but often after the conjunction if
in a subordinate clause.  If the question marker \N{srake} is used,
the topical may either come before or after that.

  \begin{interlin}
  \glll Srake ngari re'o tìsraw si? \\
     srake nga-ri re'o tìsraw si \\
     \I{q} you-\I{top} head pain do \\
  \trans{Does your head hurt?}
  \end{interlin}

  \begin{interlin}
  \glll Ngari srake re'o tìsraw si? \\
     nga-ri srake re'o tìsraw si \\
     you-\I{top} \I{q} head pain do \\
  \trans{Does your head hurt?}
  \end{interlin}

\noindent Similarly, if a complex sentence is introduced with a
conjunction, the topical can come before that,

\begin{interlin}
\glll Fori mawkrra fa renten ioi säpoli holum. \\
      fo-ri mawkrra fa renten ioi s\INF{äp}\INF{ol}i h\INF{hol}um \\
      they-\I{top} after by.means.of goggles adornment \I{refl}›do‹\I{pfv} depart‹\I{pfv}\\
\trans{After they put on their goggles, they left.}
\end{interlin}

\LNWiki{8/10/2011}{https://wiki.learnnavi.org/index.php/Canon/2011/April-December\%23Topical_Position}
\LNForum{6/5/2022}{https://forum.learnnavi.org/language-updates/which-comes-first-srake-or-the-topical/}

\subsubsection{} Some expressions have particular uses of the topical
which are idiomatic.  For example, the topical is often used with
the \N{si}-verb \N{irayo si} \E{to thank} to indicate the thing for
which you're giving thanks, \Npawl{\uwave{tìmweypeyri ayngeyä} seiyi
irayo nì\-ngay} \E{I really thank you \uwave{for your patience}}.
These will need to be learned from the dictionary.

\subsubsection{} The topical can be used to mark inalienable
possession.  Inalienable possession is posses\-sion of those things
which are intrinsically yours, and which in theory cannot be given
away or taken (except by damage).  In most languages that have this,
words for blood relatives and body parts are the most likely to have
special grammar for inalienable possession.  Note below that one's
spirit and one's life count as inalienable in Na'vi.

\begin{quotation}
\noindent\Npawl{\uwave{Oeri} nì'i'a \uwave{tsyokx} zoslolu.} \E{\uwave{My hand} is finally healed.}\\
\noindent\Npawl{\uwave{Oeri} tìngayìl \uwave{txe'lanit} tivakuk.} \E{Let the truth strike \uwave{my heart}.}\\
\noindent\Nfilm{\uwave{Oeri} ta peyä fahew akewong \uwave{ontu} teya längu.}\\
\indent\E{\uwave{My nose} is full of his alien smell.}\\
\noindent\Nfilm{\uwave{Ngari} hu Eywa salew \uwave{tirea}, \uwave{tokx} 'ì'awn slu Na’viyä hapxì.}\\
\indent\E{\uwave{Your spirit} goes with Eywa, \uwave{your body} remains to become part of the People.}\\
\noindent\Npawl{\uwave{Ngari tswintsyìp} sevin nìtxan lu nang!} \E{What a pretty \uwave{little queue you} have!}\\
\noindent\Npawl{Tseiun oe pivlltxe san \uwave{oeri} lu \uwave{tìrey} sìltsan nìngay sìk.}\\
\indent\E{Happily I'm able to say that \uwave{my life} is really good.}
\end{quotation}

\noindent Note in most of the examples that the possessed noun need
not fall immediately next to the topical. 
\index{possession!inalienable}\label{syn:topical:poss}
\index{case!topical!inalienable possession}
\index{topical!inalienable possession}
\NTeri{11/7/2010}{https://naviteri.org/2010/07/diminutives-conversational-expressions/}
% oeri ta peyä fahew akewong ontu teya längu.

\subsubsection{} The topical can be used for the point of comparison
in comparisons of equality (see \horenref{syntax:adj-eql-comp}).


\section{Adpositions}\index{adpositions}
\noindent Na'vi adpositions may govern nouns, pronouns and adverbs
of place and time.  Please see the dictionary maintained at
\href{https://learnnavi.org/navi-vocabulary/}{LearnNavi.org} or Stefan
Müller's 
\href{https://forum.learnnavi.org/projects/an-annotated-dictionary-(draft)/}{Annotated Dictionary}
for the range of uses and meanings for individual adpositions.

\subsection{Position} Adpositions can fall in two places.  First,
they may come before the entire noun phrase they modify, and are
written as separate words, \Npawl{\uwave{ta} peyä fahew akewong}
\E{\uwave{with (from)} his alien smell,} \Nfilm{ngari \uwave{hu Eywa}
salew tirea} \E{your spirit goes \uwave{with Eywa}}.  Second, they may
be enclitic, in which situation they are always attached to the noun,
\Npawl{fìtrr\uwave{mì} letsranten} \E{on this important day},
\Npawl{aylì'u\uwave{fa} awngeyä 'eylanä a'ewan} \E{in the words of our
young friend,} \Npawl{lala tsarelmì arusikx} \E{in that old movie}.
\index{adpositions!position}\label{syn:adp:position}

\subsection{Lenition}\index{plural!short} Several of the adpositions
cause lenition in the following word.  In dictionaries these are
generally indicated as \E{adp.+}, with the plus sign, as usual,
indicating lenition.

\subsubsection{} Enclitic adpositions do not cause lenition in the
noun they are attached to.  So, \N{mì hilvan} \E{in a river,} but
\N{kilvanmì}.  The combination \N{hilvanmì} can only mean \E{in
rivers}.  Enclitic adpositions also do not cause lenition on a
following word, so \N{fo kilvanmì kllkxem} \E{they stand in a river,}
not \N{fo kilvanmì *hllkxem}.

However, whatever word immediately follows a non-enclitic adposition
will be lenited.  It doesn't have to be the noun, \N{mì hivea trr}
\E{on the seventh day} (not \N{*mì kivea srr}).
\LNWiki{24/8/2010}{https://wiki.learnnavi.org/index.php/Canon/2010/July-September\%23Fmawno}

\subsubsection{} Since lenition alone is also used as the short plural
(\horenref{morph:short-plural}), there is a chance for number
uncertainty depending on the conversational context.  To be clear
about number, use the full plural prefix \N{ay+}; the lenited form
without \N{ay+} should be interpreted as singular.
\index{plural!short!with leniting adpositions}\label{syn:adp:short-plural}
\NTeri{1/7/2010}{https://naviteri.org/2010/07/thoughts-on-ambiguity/}


%\subsection{Äo} \E{Below}.  \N{Äo Utral Aymokriyä} \E{under the Tree
%of Voices}.
%\index{aäo@\textbf{äo}}\label{syn:adp:äo}
%
%\subsection{Eo} \E{Before, in front of} (place).  May be used
%metaphorically, \Nfilm{eo ayoeng lu txana tìkawng} \E{a great evil is
%upon us,} \Npawl{tokx eo tokx} \E{face to face, in person}.
%\index{eo@\textbf{eo}}\label{syn:adp:eo}
%
%\subsection{Fa} \E{With, by means of, using.}  Do not confuse with
%\N{hu} (\horenref{syn:adp:hu}).
%\index{fa@\textbf{fa}}\label{syn:adp:fa}
%
%\subsubsection{} \N{Fa} may introduce words about to be quoted,
%\Npawl{\uwave{aylì'ufa} awngeyä 'eylanä a'ewan} \E{\uwave{in the
%words} of our young friend}.
%
%\subsubsection{} \N{Fa} is also one way to express the causee when a
%transitive verb takes the causative (see \horenref{syn:trans-causative}).
%
%\subsection{Few} \E{across, (towards) the opposite side}.  Do not
%confuse with \N{ka} (\horenref{syn:adp:ka}).  \Npawl{Po spä few payfya
%fte smarit sivutx} \E{he jumped across the stream to track his prey.}
%\index{few@\textbf{few}}\label{syn:adp:few}
%
%\subsection{Fkip} \E{Up among}
%\index{fkip@\textbf{fkip}}\label{syn:adp:fkip}
%
%\subsection{Fpi (+)} \E{For the benefit or sake of}.  Refers to
%people, \Npawl{fayupxare layu aysngä\-’i\-yufpi} \E{these messages will be
%for beginners}, or inanimates, \Npawl{'uo a fpi rey'eng Eywa\-'e\-veng\-mì
%’Rrtamì tsranten nìtxan awngaru nìwotx} \E{something that matters a
%lot to all of us for the sake of The Balance of Life on both Pandora
%and Earth}.
%\index{fpi@\textbf{fpi}}\label{syn:adp:fpi}
%
%\subsection{Ftu} \E{From} (direction).  This is used mostly with
%volitional verbs of motion, such as \N{kä}, \N{rikx}, etc.
%\Nfilm{ftu fìtseng zene hivum} \E{we have to get out of here}.
%\index{ftu@\textbf{ftu}}\label{syn:adp:ftu} 
%
%See also \N{ta} below.
%
%\subsection{Ftumfa} \E{Out of, from inside}. \Npawl{riti tswolayon
%ftumfa slär} \E{the stingbat flew out of the cave;} \Npawl{reypay
%skxirftumfa herum} \E{blood is coming out of (from inside of) the
%wound}.
%\index{ftumfa@\textbf{ftumfa}}\label{syn:adp:ftumfa}
%\NTeri{25/4/2013}{https://naviteri.org/2013/04/wheres-the-bathroom-and-other-useful-things/}
%
%\subsection{Hu} \E{With}.  Of accompaniment only --- do not confuse
%with \N{fa} (\horenref{syn:adp:fa}).  \N{Tsun oe ngahu pivängkxo a
%fì’u oeru prrte’ lu} \E{it is a pleasure to be able to chat with you.}
%\index{hu@\textbf{hu}}\label{syn:adp:hu}
%
%\subsection{Io} \E{Above}.  \Npawl{Kllkxayem fìtìkangkem oeyä rofa — ke
%io — pum feyä} \E{this work of mine will stand beside — not above —
%theirs}.
%\index{io@\textbf{io}}\label{syn:adp:io}
%
%\subsection{Ìlä (+)} \E{By, via, following}.  \Npawl{Rerol tengkrr kerä
%/ Ìlä fya'o avol / Ne kxam\-tseng} \E{(we) sing while going via the
%eight paths to the center}; \Npawl{ayfo solop ìlä hilvan fa uran}
%\E{they traveled along (up, down) the river by boat}.
%\index{ilä@\textbf{ìlä}}\label{syn:adp:ìlä}
%
%\subsubsection{} It also means \E{according to,} \Npawl{ìlä Feyral,
%muntxa soli Ralu sì Newey nìwan mesrram} \E{according to Peyral,
%Ralu and Newey were secretly married the day before yesterday.}
%\NTeri{5/7/2012}{https://naviteri.org/2012/07/meetings-waterfalls-and-more/}
%
%\subsection{Ka} \E{Across, covering}.  Do not confuse with
%\N{few} (\horenref{syn:adp:few}).
%\index{ka@\textbf{ka}}\label{syn:adp:ka}
%
%\subsection{Kam} \E{Ago}.  \Npawl{Tskot sngolä'i po sivar 'a'awa
%trrkam} (or \N{kam trr a'a'aw}) \E{he started to use the bow several
%days ago}.
%\index{kam@\textbf{kam}}\label{syn:adp:kam}
%\NTeri{24/9/2011}{https://naviteri.org/2011/09/miscellaneous-vocabulary/}
%
%\subsection{Kay} \E{From now (in the future)}.  \Npawl{Zaya'u Sawtute
%fte awngati skiva'a kay zìsìt apxey} (or \N{pxeya zìsìtkay}) \E{the
%Sky People will come to destroy us three years from now!}
%\index{kay@\textbf{kay}}\label{syn:adp:kay}
%\NTeri{24/9/2011}{https://naviteri.org/2011/09/miscellaneous-vocabulary/}
%
%\subsection{Kip} \E{Among}.  \Nfilm{Tivìran po ayoekip} \E{let her walk
%among us}.
%\index{kip@\textbf{kip}}\label{syn:adp:kip}
%
%\subsection{Krrka} \E{During}. \Npawl{Krrka tsawlultxa
%Uniltìrantokxolo’ä} \E{during the Avatar Community Meet-up}.
%\index{krrka@\textbf{krrka}}\label{syn:adp:krrka}
%
%\subsection{Kxamlä} \E{through (via the middle of)}.
%\Npawl{Palukanit tsole'a, yerik lopx hifwo kxamlä zeswa},
%\E{spotting a thanator, the hexapede panicked and escaped through the
%grass.}
%\index{kxamlä@\textbf{kxamlä}}\label{syn:adp:kxamlä}
%
%\subsection{Lisre (+)} see \N{li}, \horenref{syn:li:sre}.
%
%\subsection{Lok} \E{Close to}.
%\index{lok@\textbf{lok}}\label{syn:adp:lok}
%
%\subsection{Luke} \E{Without}.  \Npawl{Luke pay, ke tsun ayoe tìreyti
%fmival} \E{without water we cannot sustain life}. This may also be
%used with nominalized phrases (see \horenref{syn:rel:nom-adp}).
%\index{luke@\textbf{luke}}\label{syn:adp:luke}
%
%\subsection{Maw, Pximaw} \E{After} (time).  \Npawl{Maw hìkrr ayoe
%tìyätxaw} \E{we will return after a short time}.
%\index{maw@\textbf{maw}}\label{syn:adp:maw}
%\index{pximaw@\textbf{pximaw}}\label{syn:adp:pximaw}
%
%\subsection{Mì (+)} \E{In, on}.  This indicates being in a location.  Motion
%inward or into is \N{nemfa}.
%\index{miì@\textbf{mì}}\label{syn:adp:mì}
%
%\subsubsection{} \N{Mì} describes location in or on the body,
%\N{aylì'u na ayskxe \uwave{mì te'lan}} \E{the words (are) like stones
%\uwave{in my heart}} (from the film script), \N{\uwave{mì tal} ngeyä
%prrnenä a sanhì lor nìtxan lu nang} \E{what pretty stars your baby
%has \uwave{on his back}}.  It also describes location in (or ``on'')
%a planet, \Npawl{lì'fyari leNa'vi \uwave{'Rrtamì}, vay set 'almong a
%fra'u zera'u ta ngrrpongu} \E{everything that has gone on with
%(blossomed regarding) Na’vi until now \uwave{on Earth} has come from a
%grassroots movement}.
%
%\subsubsection{} It can also be used in expressions for times that
%have duration, \Npawl{fìtrr\uwave{mì} letsranten} \E{on this important
%day}.  An expression such as \N{*mì kxamtrr} \E{*at noon}, however,
%is not allowed, since that refers to a particular point in time.
%
%\noindent\LNForum{7/8/2014}{https://forum.learnnavi.org/language-updates/concerning-time-words-and-fimeu/}
%
%\subsubsection{} \QUAESTIO{How to explain this: \Npawl{law lu oeru fwa
%nga \uwave{mì reltseo} nolume nìtxan!}  Restriction of scope, like
%\N{mì sìrey}?}
%
%\subsubsection{} Other idioms with \N{mì}: \N{tì'efumì oeyä} \E{in my
%opionion}. 
%
%\subsubsection{} Though the writing doesn't change, when followed by
%the plural prefix \N{ay+} the vowel \N{ì} is dropped.  \N{Mì ayhilvan}
%is pronounced as though *\N{mayhilvan} (\horenref{l-and-s:elision-i}).
%
%\subsection{Mìkam} \E{Between}
%\index{miìkam@\textbf{mìkam}}\label{syn:adp:mìkam}
%
%\subsection{Mungwrr} \E{Except}
%\index{mungwrr@\textbf{mungwrr}}\label{syn:adp:mungwrr}
%
%\subsection{Na} \E{Like, as}.  \N{Aylì'u na ayskxe mì te'lan} \E{the
%words are like stones in (my) heart}.
%\index{na@\textbf{na}}\label{syn:adp:na}
%
%\subsubsection{} \N{Na} is used to specify shades of colors,
%\Npawl{fìsyulang lu \uwave{ean na ta’leng}} or \Npawl{fìsyulang lu
%\uwave{ta’lengna ean}} \E{this flower is (Na'vi-)skin-blue}. See
%\horenref{syn:attr:na} for attributive color phrases with \N{na}.
%
%\subsubsection{} \N{Na} is used to mark the point of comparison in
%comparisons of equality (see \horenref{syntax:adj-eql-comp}).
%
%\subsection{Ne} \E{To, towards} (direction).  This marks the
%destination in verbs of motion.  \Npawl{Terìran ayoe \uwave{ayngane}}
%\E{we are walking \uwave{your way}}.  Sometimes \N{ne} can be omitted
%(see \horenref{syn:subjective:ne}).
%
%\subsubsection{} Idioms with \N{ne}: \Npawl{ke \uwave{zasyup} lì'Ona
%\uwave{ne} kxutu a mìfa fu a wrrpa} \E{The l'Ona will not
%\uwave{perish to} the enemy within or the enemy without;}
%\Npawl{zola’u nìprrte’ ne pìlok Na’\-vi\-te\-ri} \E{welcome to the
%Na'viteri blog}.
%\NTeri{29/3/2010}{https://wiki.learnnavi.org/index.php/Canon/2010/March-June\%23Intentional_Future_Details}
%
%\subsubsection{} \N{Ne} may be used to disambiguate the predicate of
%the verb \N{slu} \E{become} (\horenref{syn:predicate:slu-ne}).
%\index{ne@\textbf{ne}}\label{syn:adp:ne}
%
%\subsection{Nemfa} \E{Into}.  See also \N{mì} (\horenref{syn:adp:mì}).
%\index{nemfa@\textbf{nemfa}}\label{syn:adp:nemfa}
%
%\subsection{Nuä (+)} \E{beyond (at a distance)}.  Note the contrast
%with \N{few}, \Npawl{fo kelku si few 'ora} \E{they live across the
%lake} (on the other side) vs.\ \Npawl{fo kelku si nuä ora} \E{they
%live beyond the lake} (at a great distance and out of sight).
%\index{nuaä@\textbf{nuä}}\label{syn:adp:nuä}
%\NTeri{15/8/2011}{https://naviteri.org/2011/08/new-vocabulary-clothing/comment-page-1/\%23comment-986}
%
%\subsection{Pxaw} \E{Around}.  \N{Po pxaw txep srew} \E{he danced
%around the fire}.
%\index{pxaw@\textbf{pxaw}}\label{syn:adp:pxaw}
%
%\subsection{Pxel} \E{Like, as.} \Npawl{Fwa sute pxel nga tsun oeyä
%hì'ia tìngopit sivar fte pivlltxe nìlor fìtxan oeru teya si} \E{that
%people like you are able to use my little creation to speak so
%beautifully fills me with joy}.
%\index{pxel@\textbf{pxel}}\label{syn:adp:pxel}
%
%\subsection{Ro (+)} \E{At (locative only)}
%\index{ro@\textbf{ro}}\label{syn:adp:ro}
%
%\subsection{Rofa} \E{Beside, alongside}.  \Npawl{Kllkxayem fìtìkangkem
%oeyä rofa — ke io — pum feyä} \E{this work of mine will stand beside —
%not above — theirs;} \Npawl{maw sätswayon ayol ayoe kllpolä mì tayo a
%lu rofa kilvan} \E{after a short flight we landed in a field beside
%the river}.
%\index{rofa@\textbf{rofa}}\label{syn:adp:rofa}
%
%\subsection{Sìn} \E{On, onto}.  \Npawl{Aywayl yìm kifkeyä / 'Ìheyut avomrr
%/ Sìn tireafya’o avol} \E{the songs bind the thirteen spirals of the
%world onto the eight spirit paths}.
%\index{siìn@\textbf{sìn}}\label{syn:adp:sìn}
%
%\subsection{Sko (+)} \E{As, in the capacity of, in the role of}. 
%\Npawl{Sko Sahìk ke tsun oe mìftxele tsngivawvìk} \E{as Tsahik, I
%cannot weep over this matter}.
%\index{sko@\textbf{sko}}\label{syn:adp:sko}
%\NTeri{31/3/2012}{https://naviteri.org/2012/03/spring-vocabulary-part-2/}
%
%\subsection{Sre (+), Pxisre (+)}  \E{Before (time)}
%\index{sre@\textbf{sre}}\label{syn:adp:sre}
%\index{pxisre@\textbf{pxisre}}\label{syn:adp:pxisre}
%
%\subsection{Ta} \E{From} (various uses). \Npawl{Oeri ta peyä fahew
%akewong ontu teya längu} \E{my nose is full of (``from'') his alien
%smell}.
%\index{ta@\textbf{ta}}\label{syn:adp:ta}
%
%\subsubsection{} \N{Ta} indicates land of origin, \Npawl{Markusì ta
%Ngalwey} \E{Marcus from Galway}.
%
%\subsubsection{} Of time, \N{ta} means \E{since}, \Npawl{trr’ongta
%txon’ongvay po tolìran} \E{he walked from dawn until dusk.}
%(Also see \N{takrra}, \horenref{syn:attr:takrra}.)
%
%\subsubsection{} Frommer often uses \N{ta Pawl} \E{from Paul} at the
%end of his email and blog posts.
%
%\subsubsection{} With transitive verbs \N{ta} is more likely to be
%used to indication motion than \N{ftu}, as in \Npawl{pot 'aku
%fìtsengta} \E{get him out of here!}
%\NTeri{15/8/2011}{https://naviteri.org/2011/08/new-vocabulary-clothing/comment-page-1/\%23comment-994}
%
%
%\subsection{Takip} \E{From among}
%\index{takip@\textbf{takip}}\label{syn:adp:takip}
%
%\subsection{Tafkip} \E{From up among}
%\index{tafkip@\textbf{tafkip}}\label{syn:adp:tafkip}
%
%\subsection{Teri} \E{About, concerning}.  \Npawl{Fayupxaremì oe
%payängkxo teri horen lì'fya\-yä leNa'vi} \E{in these messages I will
%chat about the rules of the Na'vi language}.
%\index{teri@\textbf{teri}}\label{syn:adp:teri}
%
%\subsection{Uo} \E{Behind}
%\index{uo@\textbf{uo}}\label{syn:adp:uo}
%
%\subsection{Vay} \E{Up to, until}.  This may be used of both time and
%space, \Npawl{tsakrrvay, ayngeyä tìmwey\-pey\-ri irayo seiyi oe} \E{until
%that time, I thank (you) for your patience}. \QUAESTIO{There's a line
%from the video game with a local use.}
%\index{vay@\textbf{vay}}\label{syn:adp:vay}
%
%\subsubsection{} The phrase \N{vay set ke} means \E{not yet}.
%\index{vay@\textbf{vay}!\textbf{vay set ke}}
%
%\subsection{Wä (+)} \E{Against} (as in ``fight against'').
%\Npawl{Peyä tsatìpe'un a sweylu txo wivem ayoeng Omatikayawä lu fe'}
%\E{his decision that we should fight against the Omaticaya was a bad one}.
%\index{waä@\textbf{wä}}\label{syn:adp:wä}
%%https://naviteri.org/2011/07/txantsana-ultxa-mi-siatll-great-meeting-in-seattle/
%
%\subsection{Yoa} \E{In exchange for,} primarily used with verbs of
%trade and exchange, \Npawl{oel tolìng ngaru tsnganit yoa fkxen} \E{I
%gave you meat in exchange for vegetables}. May be used with
%nominalized clauses, \Npawl{käsrolìn oel nikroit Peyralur yoa fwa po
%rol oer} \E{I loaned Peyral a hair ornament in exchange for her
%singing to me}.
%\index{yoa@\textbf{yoa}}\label{syn:adp:yoa}
%\NTeri{28/2/2014}{https://naviteri.org/2014/02/barter-and-exchange/}

\section{Adverbs}
\index{adverbs}

\subsection{Degree and Quantity} Adverbs of degree and quantity very
often follow the element they modify, \Npawl{’Rrtamì tsranten
\uwave{nìtxan} awngaru \uwave{nìwotx}} \E{on Earth it matters
\uwave{very much} to us \uwave{all}}.

\subsubsection{} With predicate adjectives a very common pattern is
ADJ \N{lu} ADV, \Npawl{ngeyä lì'fya leNa'vi txantsan \uwave{lu
nìngay}} \E{your Na'vi is truly excellent}.

\subsection{With Gerunds} The gerund retains enough of its verbal
nature that it, too, may take an adverb, \Npawl{Koren a'awve
\uwave{tìruseyä 'awsiteng}} \E{the first rule \uwave{of living
together}}.  \index{gerund!with adverb}\label{syn:adverbs:gerund}

\subsection{Correlative Comparisons} The verbs \N{'ul} \E{increase}
and \N{nän} \E{decrease} are used idiomatically as correlative
adverbs, \N{'ul... 'ul} \E{the more... the more} and \N{nän... nän}
\E{the less... the less}.
\index{comparison!correlative}

\begin{quotation}
\noindent\Npawl{'Ul tskxekeng si, 'ul fnan.}\\
\noindent\E{The more you practice, the better you’ll get.}\\

\noindent\Npawl{'Ul tute, 'ul tìngäzìk.}\\
\noindent\E{The more people, the more problems.}\\

\noindent\Npawl{Nän ftia, nän lu skxom a emza'u.}\\
\noindent\E{The less you study, the less chance you have of passing.}\\

\noindent\Npawl{Nän yom kxamtrr, 'ul 'efu ohakx kaym.}\\
\noindent\E{The less you eat at noon, the hungrier you’ll feel in the evening.}
\end{quotation}
\noindent\NTeri{29/2/2012}{https://naviteri.org/2012/02/trr-asawnung-lefpom-happy-leap-day/}

\subsection{Fìtxan and Nìftxan} Both adverbs \N{fìtxan} and
\N{nìftxan} are used with the conjunction \N{kuma}
(\horenref{syn:attr:kuma}) for result clauses,

\begin{quotation}
\noindent\Npawl{Lu poe sevin nìftxan (\textrm{or} fìtxan) kuma yawne
  slolu oer.}\\
\noindent\E{She was so beautiful that I fell in love with her.}
\end{quotation}

\noindent In these constructions the \N{akum/kuma} must be contiguous
with the \N{fìtxan/nìftxan}.
\NTeri{19/6/2012}{https://naviteri.org/2012/06/spring-vocabulary-part-3/}

\subsection{Keng} The adverb \N{keng}, \E{even}, is used to prop up
unexpected information, \N{yom teylut \uwave{keng oel}}
\E{\uwave{even I} eat teylu.}
\index{keng@\textbf{keng}}
\LNWiki{31/12/2010}{https://wiki.learnnavi.org/index.php/Canon/2010/October-December\%23Keng}

\subsection{Li}  The primary meaning of \N{li} is \E{already},
\Npawl{tìkangkem li hasey lu} \E{the work is already finished.}
\index{li@\textbf{li}} 
\NTeri{20/2/2011}{https://naviteri.org/2011/02/new-vocabulary-part-2/}

\subsubsection{``Not Yet''} The negative, \N{ke li}, means ``not
yet,'' and uses pleonastic negation (\horenref{syn:neg:pleon}),
\N{fo ke li ke polähem} \E{they have not yet arrived}.
\index{not yet}\index{ke@\textbf{ke}!\textbf{ke li}}
\NTeri{4/9/2011}{https://naviteri.org/2011/09/\%E2\%80\%9Cby-the-way-what-are-you-reading\%E2\%80\%9D/comment-page-1/\%23comment-1092}



\subsubsection{Commands} With imperatives \N{li} indicates strong
urgency, \Npawl{Ngal mi fìtsengit terok srak? Li kä!} \E{You're still
here?  Get going!}  With \N{ko} (\horenref{syn:particle:ko}), \N{li
ko} (accented on \N{li}) it means ``well, get to it, then,'' or
``let's get on it.''

\subsubsection{Hesitation} In answers it conveys a somewhat hesitant
``yes,'' much like English ``sort of,''

\begin{quotation}
\noindent A: \Npawl{Nga mllte srak?} \E{Do you agree?}\\
\noindent B: \N{Li, slä\dots} \E{Well, yes, I guess so, but\dots}.
\end{quotation}

\noindent The negative of this, \N{ke li}, means something like ``not
really.'' 

\subsubsection{With ``sre''} When \N{li} is paired with the
adposition \N{sre} they mean ``by'' in the temporal sense of ``before
or up to but, not after,'' \Npawl{kem si li trraysre} \E{do it by
tomorrow}.\label{syn:li:sre} If \N{sre} comes before the noun, it
combines with \N{li} into \N{\ACC{li}sre}, which like \N{sre} will
cause lenition, \Npawl{kem si lisre srray} \E{do it by tomorrow}.
\index{lisre@\textbf{lisre}}


\subsection{Nìwotx} The adverb \N{nìwotx} \E{all (of), in toto,
completely} is frequently used with plural nouns and pronouns to
give a collective sense, \Npawl{ayeylanur oeyä sì eylanur lì'fyayä
leNa'vi \uwave{nìwotx}} \E{to \uwave{all} my friends and friends of
the Na'vi language}, \Npawl{tìfyawìntxuri oeyä perey aynga
\uwave{nìwotx}} \E{you are \uwave{all} waiting for my guidance}.
\index{nìwotx@\textbf{nìwotx}}

\subsubsection{``Both''} With dual number, the sense of \N{nìwotx} is
\E{both,} \N{mefo nìwotx yolom} \E{they both ate}.
\index{nìwotx@\textbf{nìwotx}!with dual}\index{both}
\NTeri{15/8/2011}{https://naviteri.org/2011/08/new-vocabulary-clothing/}

\subsection{Nìfya'o} A noun phrase built on \N{fya'o} can be used
freely to produce adverbs of manner.  In this construction the entire
noun phrase is adverbialized, not just the word \N{nì-} is prefixed
to, \N{nì-[fya'o letrrtrr]} \E{in an ordinary way}, \Npawl{poe poltxe
nìfya'o alaw}
\E{she spoke clearly}.
\index{nìfya'o@\textbf{nìfya'o}}\label{syn:nifyao}

\subsubsection{} \N{Nìfya'o} can also take attributive phrases,
\Npawl{nìfya'o a hek} \E{in a way that's strange.}

\subsubsection{} \QUAESTIO{Note about sentence adverbs vs. \N{nìfya'o} forms?}

\subsection{``Kop'' and ``nìteng''} Both \N{kop} and \N{nìteng} answer
to the English adverb \E{also}.  \N{Kop} has more the sense of \E{in
addition, further,} while \N{nìteng} means \E{similarly, too,
likewise}.  Compare \N{oel poleng kop poru tsa'ut} \E{I also (in
addition) told him that} to \N{oel poleng nìteng poru tsa'ut} \E{I
told him that, too}.
\index{kop@\textbf{kop}}
\index{nìteng@\textbf{nìteng}}

\subsubsection{} They can even be used together, \Npawl{furia nga lu
nitram, lu oe kop nitram nìteng} \E{since you're happy, I, too, am
also happy}.
% https://naviteri.org/2011/05/weather-part-2-and-a-bit-more-2/comment-page-1/#comment-779

\subsection{Sunkesun} The adverb \N{\ACC{sun}kesun} \E{like it or not}
is a shortened form of \N{sunu ke sunu} with the default addressee
being \E{you}, \Npawl{sunkesun po slayu olo'eyktan} \E{whether you
like it or not, he's going to become chief.}
\index{sunkesun@\textbf{sunkesun}}
\NTeri{6/6/2019}{https://naviteri.org/2019/06/50a-liu-amip-40-new-words/}

\subsubsection{} Note that \N{sunkesun} can only be addressed to the
listener, otherwise one must use a \N{ftxey... fuke}
construction, \Npawl{pol vìyewng ayevengit fìha'ngir, ftxey sunu
fuke} \E{He is going to take care of the children this afternoon,
whether he likes it or not.}


\section{Aspect and Tense}

\subsection{The Role of Context} Na'vi verbs are frequently unmarked
for tense or aspect, leaving a verb without infixes, or at most the
subjunctive infix.  Absent other information, such as an adverb of
time or some break in discourse, an unmarked verb continues the tense
and/or aspect of the verb in the previous sentence.\index{verb!unmarked}

\subsubsection{} Although a subordinate clause may occur before the
main clause, it takes its temporal and aspectual context from the main
clause, \Npawl{oel foru fìaylì'ut \uwave{tolìng} a krr, kxawm oe
\uwave{harma\-hä\-ngaw}} \E{when I \uwave{gave} them these words perhaps
I \uwave{was sleeping}}, \Npawl{tì'eyngit oel \uwave{tolel} a krr,
ayngaru \uwave{payeng}} \E{when I \uwave{receive} an answer, I
\uwave{will tell} you}.

\subsection{The Unmarked Verb} The unmarked verb form has two
additional jobs.  First, it can indicate the present tense,
\Npawl{ayngaru seiyi irayo} \E{I thank you}.  Second, it is used for
habitual or general statements, \Npawl{nga za'u fìtseng pxìm srak?}
\E{do you come here often?} \N{lu fo lehrrap} \E{they are dangerous}.
\index{verb!present tense}\index{verb!unmarked}

\subsection{Aspect} In general, Na'vi marks aspect more than it marks
tense.\footnote{Verb aspect can be difficult for speakers of English
and most European languages, since these mix tense and aspect
together in their verbs, making it difficult to distinguish the
ideas.  The dangerous confusion for beginners is this idea that verb
aspect is about the completion or non-completion of an act.  This is
not the case.  Rather, verb aspect is about how the speaker wishes to
\textit{present} a scene.  For example,

\begin{quotation}
\noindent 1.  I went to the store.  (perfective)\\
2.  While I was going to the store (imperfective), I saw the most amazing thing. (perfective)
\end{quotation}
In both sentence (1) and (2) the act of going to the store is done
and over, but I use the imperfective in sentence (2) because
it's background to the next, perfective, statement.

In complex sentences aspects might take on senses related to
completion or non-completion with respect to other clauses in the full
sentence, but these are special uses.}  It is useful to think of the
perfective as a snapshot presentation of an event, while the
imperfective sets the background, \Npawl{tengkrr palulukan moene kxll
\uwave{sarmi}, \uwave{poltxe} Neytiril aylì'ut a frakrr 'ok seyä
layu oer} \E{as the thanator \uwave{was charging} towards the two of
us, Neytiri \uwave{said} something I will always remember.}

\subsection{Simultaneous Imperfective} Because the imperfective
presents a state of affairs as on\-going, it can be used in complex
sentences to indicate simultaneous action, \Npawl{fìtxon yom tengkrr
\uwave{teruvon}} \E{this night (we) eat while leaning}.
\index{imperfective!simultaneous}
\LNWiki{14/3/2010}{https://wiki.learnnavi.org/index.php/Canon/2010/March-June\%23A_Collection}

\subsection{Anterior Perfective} In complex sentences, the perfective
in a subordinate clause can indicate the completion of an action prior
to the event in the main clause, \index{perfective!anterior}

\begin{quotation}
\noindent\Npawl{Tì'eyngit oel \uwave{tolel} a krr, ayngaru payeng}.\\
\indent\E{When I \uwave{receive} an answer, I will let you know}.\\

\noindent\Npawl{Fori mawkrra fa renten \uwave{ioi säpoli} holum}.\\
\indent\E{After they \uwave{put on} their goggles, they left.}
\end{quotation}

\subsection{Punctual Perfective} The perfective is used in several
single verb expressions to indicate the event occurred in an instant,
\N{tslolam} \E{got it, I understand,} \N{rolun} \E{found it!}
Frommer says \N{tolel}, \E{got it!}, is for a ``flash of insight.''
\index{perfective!punctual}

\subsection{Tense} Na'vi tense, as in Human languages, simply locates
an event in time.

\QUAESTIO{There are too few examples of complex sentences to be sure
about relative tense in subordinate clauses.}

\subsection{Proximal Tense} The proximal past and future mark events
in the ``near'' past or future, where nearness is not an absolute
scale, but is determined by context and the perspective of the
speaker.

\subsection{Intentional Future} The intentional future forms in
\N{\INF{ìsy}} and \N{\INF{asy}} indicate the deter\-mina\-tion by the
speaker to bring about a state of affairs, rather than a prediction
about the future.  \Npawl{Ayoe ke \uwave{wasyem}} \E{we will not
fight}, \Npawl{tafral ke \uwave{lìsyek} oel ngeyä keye'ungit}
\E{therefore I will not heed your insanity}.\label{syn:verb:intenfut}
\index{future!intentional}


\section{Subjunctive}
\index{subjunctive}
\noindent The subjunctive is very frequent in Na'vi.  Outside its use in
independent sentences, the sub\-junct\-ive is highly grammaticalized, that
is, its use is simply required in certain grammatical constructions
without necessarily hinting at an \textit{irrealis} sense.

\subsection{Optative} It is used to indicate a wish, \Npawl{oeyä swizaw
nìngay tivakuk} \E{let my arrow strike true}.\index{subjunctive!optative}

\subsection{Nìrangal} Unrealizable wishes use the adverb \N{nìrangal}
followed by the imperfective sub\-junc\-tive to indicate an unattainable
wish in the present, with the perfective subjunctive for an
unattainable wish in the past.  This can be expressed in English with
phrases like, ``if only'' or ``I wish,'' \N{nìrangal lirvu oeyä
frrnenur lora sanhì} \E{I wish my children had pretty stars},
\N{nìrangal oel tslilvam nì'ul} \E{if only I had understood more}.
\index{nìrangal@\textbf{nìrangal}}
\LNWiki{14/3/2010}{https://wiki.learnnavi.org/index.php/Canon/2010/March-June\%23A_Collection}

\subsection{Modal Complement} \index{modal verb}
The verbal complement to a modal verb, such as \N{zene} \E{must},
\N{tsun} \E{can}, etc., will take the subjunctive, as in \N{ayngari
zene hivum} \E{you must leave}, \Npawl{oe new nìtxan ayngaru
fyawivìntxu} \E{I want very much to guide you}, \Npawl{fmawn a tsun oe
ayngaru tivìng} \E{news which I can give to you}. \label{syn:modals}


\subsubsection{} \label{syn:modal-syntax} The verb controlled by the
modal will only takes the pre-first position infixes (the reflexive
and causative, \horenref{morph:pre-first}) and the subjunctive.  It
will not take tense, aspect, or affect infixes.  So, marking for these
should go on the modal, \N{oe namew tsive'a} \E{I wanted to see,}
never \N{*oe new tsimve'a}. 
\LNForum{14/10/2010}{https://forum.learnnavi.org/language-updates/sieyng-a-ftu-naring-11-\%28ni\%29tengfya-genders-modals-and-infixes/}

\begin{quotation}
\noindent\Npawl{Pori mesyokx rìkxi, ha ke \uwave{tsayun} yerikit \uwave{tivakuk}.}\\
\indent\E{His hands tremble, so he will not be able to hit the hexapede}\\
\noindent\Npawl{Furia \uwave{tsolun} oe ngahu \uwave{pivängkxo}, oeru prrte' lu nìngay.}\\
\indent\E{It was really a pleasure to speak with you}.\\
\noindent\Npawl{Fteria oel lì'fyati leNa'vi, slä mi ke \uwave{tsängun pivlltxe} na hufwe.}\\
\indent\E{I'm studying Na'vi, but I'm afraid I still can't speak it fluently.}
\end{quotation}

Except in poetry or ceremonial language, the modal verb will always
come before the control\-led verb.
\NTeri{3/19/2011}{https://naviteri.org/2011/03/word-order-and-case-marking-with-modals/}


\subsubsection{} Known modal verbs and verbs with modal
syntax:\footnote{\QUAESTIO{Other candidates: \N{flä} \E{succeed},
    \N{hawl} \E{prepare}.}}
% This is what I get for attaching several words to the same footnote.
\addtocounter{footnote}{1}
\newcounter{modalsnote}\setcounter{modalsnote}{\value{footnote}}
\begin{center}
\begin{tabular}{ll}
\N{fmi} & try, attempt \\
\N{ftang} & stop \\
\N{kan} & intend to\footnotemark[\value{modalsnote}] \\
\N{kom} & dare\\
\N{may'} & try (experiential) \\
\N{new} & want\footnotemark[\value{modalsnote}] \\
\N{nulnew} & prefer \\
\end{tabular}
\hskip 3em
\begin{tabular}{ll}
\N{sngä'i} & begin, start \\
\N{sto} & refuse\footnotemark[\value{modalsnote}] \\
\N{tsun} & can, be able \\
\N{var} & keep on, continue to \index{var@\textbf{var}} \\
\N{zene} & must, have to \\
\N{zenke} & must not \\
\end{tabular}
\end{center}
\footnotetext[\value{modalsnote}]{See also \horenref{syn:modal:new}.}
\index{fmi@\textbf{fmi}!modal}\index{ftang@\textbf{ftang}!modal}
\index{new@\textbf{new}!modal}\index{kan@\textbf{kan}!modal}
\index{snaä'i@\textbf{sngä'i}!modal}\index{sto@\textbf{sto}!modal}
\index{tsun@\textbf{tsun}!modal}
\index{var@\textbf{var}!modal}\index{zene@\textbf{zene}!modal}
\index{zenke@\textbf{zenke}!modal}\index{may'@\textbf{may'}!modal}\index{kom@\textbf{kom}!modal}

\noindent\NTeri{25/5/2011}{https://naviteri.org/2011/05/some-miscellaneous-vocabulary/}
\LNWiki{1/12/2010}{https://wiki.learnnavi.org/index.php/Canon/2010/October-December\%23As_X_as_Y.3B_Keep_on_keepin.27_on}
\LNWiki{2/2/2011}{https://wiki.learnnavi.org/index.php/Canon/2011/January-March\%23Stop.21}
\Ultxa{2/10/2010}{https://wiki.learnnavi.org/index.php/Canon/2010/UltxaAyharyu\%C3\%A4\%23.C3.ACm.C3.ACy_and_modal_kan}
\LNWiki{13/12/2012}{https://wiki.learnnavi.org/index.php/Canon/2012/July-December\%23.22sto.22_has_the_same_syntax_as_.22new.22}
\NTeri{6/6/2019}{https://naviteri.org/2019/06/50a-liu-amip-40-new-words\%ef\%bb\%bf/}
\NTeri{23/12/2020}{https://naviteri.org/2020/12/mrra-tipangkxotsyip-five-little-discussions/}

\subsubsection{} Note that the modal verbs are considered intransitive
with the subject of the modal phrase in the subjective case,
regardless of the transitivity of the controlled verb, \N{\uwave{oe}
new yivom teylut} \E{I want to eat teylu}.  But see ``word order
effects,'' \horenref{pragma:word-order-effects:modals}, for some
exceptional patterns.

If the modal is one capable of taking a subclause, however, then it
behaves transitively, \Npawl{oel new futa nga srew} \E{I want you to
dance} (\horenref{syn:modal:subclause}).

\subsubsection{}\label{syn:modal:neg}
When \N{ke} is used with modal verbs, it may go either in front of the
modal verb or in front of the controlled verb.  In some cases, this
changes the meaning as with \Npawl{ke zene kivä} \E{do not have to go}
versus \N{zene ke kivä} \E{must not go} or \Npawl{po ke tsun
yivom} \E{they are not able to eat} versus \Npawl{po tsun ke
yivom} \E{they are able to not eat}.
\LNForum{23/7/2019}{https://forum.learnnavi.org/language-updates/negation-and-modal-verbs/}

\subsection{Modals with Subclauses}
\label{syn:modal:new}\label{syn:modal:subclause}
Several of the modals may also take subclauses introduced
with \N{futa} or \N{a fì'ut} (\horenref{morph:fwa-tsawa}).  Normally
when the subject of the modal and the controlled verb are the same,
you do not need to use one of the conjunctions, but it is possible.
\index{fmi@\textbf{fmi}!with \N{futa}} \index{kan@\textbf{kan}!with \N{futa}}
\index{may'@\textbf{may'}!with \N{futa}} \index{new@\textbf{new}!with \N{futa}}
\index{nulnew@\textbf{nulnew}!with \N{futa}} \index{sto@\textbf{sto}!with \N{futa}}

\begin{quotation}
\noindent\Npawl{Oe new srivew}.  \E{I want to dance.} \\
\noindent\Npawl{Oel new futa srew}.  \E{I want to dance,} lit., \E{I want that I dance}
\end{quotation}

\noindent Note here than the sublcause does not require the
subjunctive.  However, due to analogy with the very common pattern
seen in \N{oe new srivew}, it is permitted to use the subjunctive as
well, \N{oel new futa srivew}.  This also applies if the the subject
of the subclause is different from the main clause:

\begin{quotation}
\noindent\Npawl{Oel new futa nga srew}.  \E{I want you to dance.} \\
\noindent\Npawl{Oel new futa nga srivew}.  \E{I want you to dance}. \\

\noindent\N{Poel stolatso futa mefo tivaron tsaha'ngir.} \\
\indent\E{She must have refused (their request) to hunt that afternoon.}
\end{quotation}
\LNForum{29/11/2020}{https://forum.learnnavi.org/language-updates/causative-with-new-and-other-modals/}
\LNForum{13/12/2012}{https://forum.learnnavi.org/language-updates/sto-has-the-same-syntax-as-new/}

\noindent The verbs which permit this variation in the subclause
are: \N{fmi, kan, may', new, nulnew,} and \N{sto}.

\subsubsection{} The causative of transitive \N{new} will also take a
\N{futa} clause, \Npawl{pol oeru neykew futa oel yivom teyluti}
\E{he made me want to eat teylu}, lit. \E{he made me want that I eat
teylu}. \index{new@\textbf{new}!causative}
\LNWiki{3/10/2010}{https://wiki.learnnavi.org/Canon/2010/UltxaAyharyu\%C3\%A4\#He\_made\_me\_want\_to\_make\_you\_eat\_teylu}
\LNForum{29/11/2020}{https://forum.learnnavi.org/language-updates/causative-with-new-and-other-modals/}



%%%%%%%%%%%%%%%%%%%%%%%%%%%%%%%%%%%%%%%%%%%%%%%%%%%%%%%%%%%%
%%% 2020dec02 - preserved for examples, just in case

%\subsection{New} In addition to the simple modal use given above
%(\horenref{syn:modals}), \N{new} \E{want} may also intro\-duce a
%subclause with a different subject than that of the \N{new} clause.
%The verb is transitive in this construction, and the subclause is
%attached to \N{a fì'ut} or \N{futa} (\horenref{syn:clause-nom}) and
%takes the subjunctive. \label{syn:modal:new}\index{new@\textbf{new}}
%\LNWiki{20/1/2010}{https://wiki.learnnavi.org/index.php/Canon\%23Extracts_from_various_emails}
%
%\begin{quotation}
%\noindent \N{Oel new futa po kivä} \E{I want him to go} (lit. \E{I
%  want that he go}). \\
%\noindent \N{Ngal tslivam a fì'ut new oel} \E{I want you to understand}.
%\end{quotation}
%
%\subsubsection{} The modal use of \N{kan} \E{aim} for \E{intend}
%follows the same syntax, \N{oe kan kivä} \E{I intend to go} and \N{oel
%kan futa po kivä} \E{I intend him to go}.  \index{kan@\textbf{kan}}
%
%\subsubsection{} The verb \N{sto} \E{refuse} also follows the syntax
%of \N{new}, \N{stolo po hivum fohu} \E{she refused to leave with
%them,} \N{poel stolatso futa mefo tivaron tsaha'ngir}
%\E{she must have refused (their request) to hunt that afternoon}.
%\index{sto@\textbf{sto}}

%%% 2020dec02 - preserved for examples, just in case   ^ ^ ^ ^ ^ ^ ^ ^
%%%%%%%%%%%%%%%%%%%%%%%%%%%%%%%%%%%%%%%%%%%%%%%%%%%%%%%%%%%%

\subsection{Other Uses} The subjunctive is also used in purpose
clauses with \N{fte} (\horenref{syn:purpose}), conditional sentences
(\horenref{syn:conditionals}), and with the conjunction \N{tsnì}
when used with certain verbs (\horenref{syn:tsni}). 


\section{Participles and Gerunds}

\subsection{Participles} \index{participle!use}
Na'vi participles are restricted in their use --- they may only be
used attribu\-ti\-vely, never as predicates.  Since they are adjectives,
they are linked to the noun they go with using the attributive affix
\N{-a-} (\horenref{morph:adj-attr}), \Npawl{\uwave{palulukan atusaron}
lu lehrrap} \E{a \uwave{hunting thanator} is
danger\-ous}.\label{syn:part:attr}

\subsubsection{} Some derived words have participles in them, and
these may be used predictively, as in \Npawl{lu nga txantslusam}
\E{you are wise}, with the active participle \N{tsl\INF{us}am} in it.

\subsubsection{} The participles of \N{si} construction verbs are
counted as a single word.  They are written with a hyphen connecting
\N{si} and the other word and the attributive \N{a} is attached to the
entire phrase, not just \N{si}: \label{syn:participle:si-const}
\index{si construction@\textbf{si} construction!participle}

\begin{quotation}
\noindent\N{srung-susi\uwave{a} tute}\\
\noindent\N{tute \uwave{a}srung-susi}
\end{quotation}

\noindent Both phrases mean \E{a helping person}.

\subsection{Gerunds} Any verb may be freely turned into a gerund, a
noun describing the action of the verb (\horenref{lingop:gerund}).
They can be used with adverbs (\horenref{syn:adverbs:gerund}),
but they may not take subjects or direct objects, \Npawl{tìyusom
'o' lu} \E{eating is fun}.\label{syn:gerund}
\LNWiki{18/6/2010}{https://wiki.learnnavi.org/index.php/Canon/2010/March-June\%23Fwa_with_adpositions}

\subsubsection{} English often uses gerunds to nominalize a phrase
(``running a mara\-thon is difficult'').  In Na'vi such clause
nominalization is handled with \N{fì'u} or \N{tsa'u} 
(\horenref{syn:clause-nom}), \Npawl{\uwave{fwa yom teylut} 'o' lu}
\E{\uwave{eating teylu} is fun}.\index{gerund!use}
\Ultxa{3/10/2010}{https://wiki.learnnavi.org/index.php/Canon/2010/UltxaAyharyu\%C3\%A4\%23Gerunds_vs._Fwa}


\section{Reflexive}
\index{reflexive!syntax}
\subsection{True Reflexives} The reflexive infix \N{\INF{äp}}
indicates the subject of the verb is performing an act on themself.
The subject is in the subjective, not agentive, case, as in \Npawl{oe
tsäpe'a} \E{I see myself}.
\LNWiki{1/2/2010}{https://wiki.learnnavi.org/index.php/Canon\%23Reflexives_and_Naming}

\subsection{Intransitive Reflexives} With intransitive verbs that take
dative objects reflexive pro\-noun \N{sno} is used,
\index{reflexive!of intransitive}

\begin{quotation}
\noindent\N{Po yawne lu snor.} \E{He loves himself.}
\end{quotation} \index{sno@\textbf{sno}!with intransitive reflexives}
\noindent \NTeri{31/12/2011}{https://naviteri.org/2011/12/one-more-for-2011/}

\subsection{Detransitive} The reflexive infix may also be used to
create intransitive verbs,\footnote{Students of Romance languages will
find this familiar, \textit{je me lave} vs. \textit{je lave ma
voiture}.} such as \N{win säpi} \E{to hurry}.

\subsection{Reciprocal} When a reflexive verb occurs with the adverb
\N{fìtsap} \E{each other,} the meaning is reciprocal, \Npawl{mefo
fìtsap mäpoleyam tengkrr tsngawvìk} \E{the two of them hugged each
other and wept}.
\NTeri{30/10/2011}{https://naviteri.org/2011/10/more-vocabulary-a-bit-of-grammar/}
\index{fiìtsap@\textbf{fìtsap}}\index{reciprocal}

\subsubsection{} With intransitive verbs that take dative objects
there are two possibilities,

\begin{quotation}
\noindent\Npawl{Moe smon moeru fìtsap.} \E{We know each other.} \\
\noindent\Npawl{Moe smon fìtsap.} \E{We know each other.}
\end{quotation}

\noindent With third person reflexives of any number, the dative of
\N{sno} is used,

\begin{quotation}
\noindent\Npawl{Fo smon (snoru) fìtsap nìwotx.} \E{They all know each other}.
\end{quotation}

\noindent \NTeri{31/12/2011}{https://naviteri.org/2011/12/one-more-for-2011/}

\section{Causative}
\noindent The causative infix \N{\INF{eyk}} increases the transitivity
of a verb, adding another argument.  All causa\-tive verbs are thus
transitive, requiring the agentive case for the subject.
\label{syn:causative}

\subsection{Causative of Intransitive Verb} When an intransitive verb
is made causative, the causee, which had been in the subjective case,
is in the patientive.\index{causative!of intransitive}

\begin{quotation}
\noindent\N{\uwave{Oe} kolä neto} \E{I went away}.\\
\noindent\N{Pol \uwave{oeti} keykolä neto} \E{She made me go away}.
\end{quotation}

\subsection{Causative of Transitive Verb} When a transitive verb is
made into a causative, the causee, which had been in the agentive
case, goes into the dative.  This leaves the original accus\-at\-ive in
place. \index{causative!of transitive} \label{syn:trans-causative}
\index{case!dative!with causative}

\begin{quotation}
\noindent\N{\uwave{Neytiril} \uuline{yerikit} tolaron} \E{Neytiri hunted hexapede}.\\
\noindent\Npawl{Eytukanìl \uwave{Neytirir} \uuline{yerikit} teykolaron}\\
\indent \E{Eytukan made Neytiri hunt a hexapede.} 
\end{quotation}

\subsubsection{} The causee may also be indicated with the adposition
\N{fa}, \E{by means of}.  This defocuses the causee somewhat, focusing
instead on either the causer or object.\index{fa@\textbf{fa}!with causative}

\begin{quotation}
\noindent\N{\uwave{Neytiril} \uuline{yerikit} tolaron} \E{Neytiri hunted hexapede}.\\
\noindent\Npawl{Eytukanìl \uwave{fa Neytiri} \uuline{yerikit} teykolaron} \\
\indent \E{Eytukan had a yerik hunted by Neytiri.} 
\end{quotation}



\section{Ambitransitivity}
\noindent A normally transitive verb can be paired with a subjective,
rather than agentive, noun as the subject.  This is used when the
direct object is considered irrelevant and only the verbal action
matters. For example, \N{oe taron} \E{I hunt} is a general statement
about one's activities, where what one is hunting in particular
doesn't matter. Or, \index{antipassive} \index{ambitransitive}
\NTeri{28/3/2012}{https://naviteri.org/2012/03/spring-vocabulary-part-1/}

\begin{quotation}
\noindent\Npawl{Ngal pelun faystxenut frakrr tsyär?}\\
\indent\E{Why do you always reject these offers?} \hskip3em vs. \\
\noindent\Npawl{Nga pelun frakrr tsyär?}\\
\indent\E{Why do you always reject everything?} or  \E{such things?}
\end{quotation}

\noindent This pattern of alternation can also be called an
``antipassive'' construction, and may be freely used in Na'vi.

\subsection{Omitted Object} This use should be distinguished from
omission of a direct object that exists in the context. For example,

\begin{quotation}
\noindent\N{Ngal ke tse'a txepit srak?} \E{Do you not see the fire?}\\
\noindent\N{Oel tse'a.} \E{I see (it).}
\end{quotation}

\noindent Here the direct object is simply not mentioned, rather than
suppressed entirely, so the verb and subject still follow the normal
transitive syntax.\index{direct object!omitted}

\subsection{Causative} There is no way to distinguish the antipassive
in the causa\-tive.  For example, the resulting action of the sentence
\N{oel poru teykaron} \E{I make him hunt} could either be \N{po taron}
\E{he hunts (something we don't care about)} or \N{pol taron} \E{he
hunts (something in particular)}.
\index{antipassive!causative}
\Ultxa{3/10/2010}{https://wiki.learnnavi.org/index.php/Canon/2010/UltxaAyharyu\%C3\%A4\%23Causative_for_ambitransitive_verbs}


\section{Commands}
\index{commands}\index{imperative}
\subsection{Unmarked} Commands in Na'vi require no special infix.
Positive commands are simply a verb stem, \N{Kä! Kä!} \E{Go! Go!}, 
\Nfilm{mefoti yìm} \E{bind them!}  The pronoun may also be stated
explicitly, \Npawl{'awpot set ftxey ayngal} \E{(you) choose one now}.
\index{command!plain}

\subsection{With the Subjunctive} A command may also use the
subjunctive infix \N{\INF{iv}}. Frommer says, ``at an earlier point in
the history of the language there was probably a polite/familiar
distinction (the \N{\INF{iv}} form being the politer one), but that's no
longer the case. They're used interchangeably.  So to say `Go!' you can
say either \N{kivä} or just \N{kä}.''
\index{subjunctive!for commands}\index{command!with subjunctive}

\subsection{Prohibitions} Negative commands are not negated with the
usual negative adverb \N{ke}, but rather use the word \N{rä'ä}, as in
\N{rä'ä hahaw} \E{don't sleep}.
\label{syntax:prohibitions}\index{command!negative}\index{prohibitions}

\subsubsection{} \N{Rä'ä} may follow the verb for special emphasis,
\Npawl{oeti 'ampi rä'ä, ma skxawng!} \E{don't touch me, you moron!}
\NTeri{27/11/2012}{https://naviteri.org/2012/11/renu-ayinanfyaya-the-senses-paradigm/}

\subsubsection{} With \N{si}-construction verbs, \N{rä'ä} intrudes
between the noun and \N{si}, \N{txopu rä'ä si} \E{don't be afraid},
\N{tsakem rä'ä sivi} \E{don't do that (action)} (see also
\horenref{syn:neg:si-const}).\index{si construction@\textbf{si} construction!prohibition}



\section{Questions}
\index{question}
\subsection{Polar Questions} \index{question!polar}\index{question!yes-no}
Simple yes-no questions are marked with the particle \N{srak(e)} which
occurs at the start or the end of the clause.  When the particle
appears at the end of the clause, it is usually just \N{srak}, the
longer \N{srake} occurring at the beginning of the
clause.  \Npawl{Ngaru lu fpom srak?} \E{are you well?}

\subsubsection{Negative Polar Questions} \index{question!negative}
The particle \N{srak(e)} is requesting confirmation on the truth or
falsity of the entire statement it is attached to.  So the correct
answer to \Npawl{nga ke lu Txewì srak?} \E{you're not Txewi?}
is \N{srane} if you are not, and \N{kehe} if you are.  Note that
English handles this situation differently, and English speakers will
need to take care with how they answer negative questions.
\NTeri{28/2/2018}{https://naviteri.org/2018/02/negative-questions-in-navi/}

\subsection{Ftxey... Fuke} In addition to \N{srak(e)} a yes-no
question can be made with an idiom using \N{ftxey} \E{choose} and
\N{fu\ACC{ke}} \E{or not}.  You can say either \Npawl{srake nga za'u?}
\E{are you coming} or \Npawl{ftxey nga za'u fuke} \E{are you coming or
not?} \index{ftxey@\textbf{ftxey}}\index{fuke@\textbf{fuke}}
\index{question!direct with \textbf{ftxey... fuke}}\label{syn:question:ftxey}
\LNWiki{24/3/2010}{https://wiki.learnnavi.org/index.php/Canon/2010/March-June\%23If_and_Whether}

\subsection{Wh-Questions} Use of a question word that contains
\N{-pe+} is sufficient to create a quest\-ion, \N{kempe si nga?} \E{what
are you doing?}  In many languages a question word must come first
in the sentence.  Na'vi has no such requirement, \Nfilm{fìswiräti ngal
\uwave{pelun} molunge fìtsenge?} \E{\uwave{why} did you bring this
creature here?}\index{question!with question words}

\subsection{Tag Question}
The Na'vi tag question (Eng. ``right?'', Fr. ``n'est-ce pas?'') is
marked with either \N{kefya srak} or simply \N{kefyak} (ultimately
derived from \N{ke fìfya srak?}).
\index{question!tag}\index{kefak, kefya srak@\textbf{kefyak, kefya srak}}
\LNWiki{1/3/2010}{https://wiki.learnnavi.org/index.php/Canon\%23Tag_Question}

\subsection{Conjectural Questions} Questions which the speaker
doesn't expect even the listeners to know the answer to are marked
with the evidential infix \N{\INF{ats}}, \Npawl{pol pesenget tatsok?}
\E{where in the world could she be?} \Npawl{srake pxefo li
polähatsem?} \E{I wonder if the three of them have already arrived.}
\index{evidential!in questions}\index{question!conjectural}
\NTeri{30/10/2011}{https://naviteri.org/2011/10/more-vocabulary-a-bit-of-grammar/}

\subsection{Choice Questions} A question in which the speaker offers two choices is formed by placing \N{fu} before each choice, \Npawl{Nulnew ngal fu fì’ut fu tsa’ut?}
\E{Do you want this or that?}
\index{question!with choices}\
\NTeri{30/09/2019}{https://naviteri.org/2019/09/choice-statements-vs-choice-questions-and-some-insults/}

\section{Affect and Evidence}

\subsection{Affect} Two second position infixes are used to mark the
speaker's attitude about what they are saying, \N{\INF{ei}} for
positive orientation and \N{\INF{äng}} for negative orientation,
\Nfilm{oel ngati kameie} \E{I see you,} \Nfilm{oeri ta peyä fahew
akewong ontu teya längu} \E{his alien smell fills my nose}.
\index{affect!positive}\index{affect!negative}

\subsubsection{} If a statement inherently encodes very positive or
negative emotion the infix is likely to be omitted, as in \Npawl{nga
yawne lu oer} \E{I love you}.
\LNWiki{1/2/2010}{https://wiki.learnnavi.org/index.php/Canon\%23Extracts_from_various_emails}

\subsection{Evidence} The second position infix \N{\INF{ats}} is used
to mark a suppositional statement from evidence,\footnote{This roughly
corresponds to English ``must'' in such statements as ``it must have
rained'' or ``he must be having trouble with his homework.''}
\Npawl{'uol ikranit txopu sleykolatsu, taluna po tsìk yawo}
\E{something must have fright\-ened the banshee, because it suddenly
took to the air.}
\index{evidential}
\LNWiki{19/2/2010}{https://wiki.learnnavi.org/index.php/Canon\%23Evidential}
%\NTeri{9/1/2012}{https://naviteri.org/2012/01/mipa-zisit-ayliu-amip-new-words-for-the-new-year/}

% see def of yawo for example of evidential
%example of evidential 


\section{Negation}

\subsection{Simple Negation} The adverb \N{ke} is used to negate a
sentence, \Npawl{fìtxon na ton alahe nìwotx pelun ke lu teng?} \E{why
is this night not like all other nights?}
\index{negation}\index{ke@\textbf{ke}}

\subsubsection{} With \N{si}-construction verbs, the \N{ke} comes
before \N{si}, as in \N{po pamrel ke si} \E{he doesn't write}.  The
phrase accent shifts from the noun or adjective part of the
\N{si}-verb to \N{ke}, \N{pamrel \ACC{ke} si} (see also
\horenref{syntax:prohibitions}). \label{syn:neg:si-const}
\index{si construction@\textbf{si} construction!negation}
\index{negation!si construction@\textbf{si} construction}
\index{ke@\textbf{ke}!with \textbf{si} construction}

\subsubsection{} Imperatives are negated with the adverb \N{rä'ä}.
See \horenref{syntax:prohibitions}.\index{prohibitions}

\subsubsection{} In some cases, the position of \N{ke} in a modal construction changes the meaning.  See \horenref{syn:modal:neg}.

\subsection{Pleonastic Negation} When a negative adverb or pronoun
(\horenref{morph:correlatives}) is used \N{ke} is still re\-qui\-red with
the verb, \Npawl{ke'u ke lu ngay} \E{nothing is true}, \N{slä ke
stä'nì kawkrr} \E{but (he) never catches (her)}.
\index{negation!double negatives}\label{syn:neg:pleon}
\LNWiki{2/5/2010}{https://wiki.learnnavi.org/index.php/Canon/2010/March-June\%23Double_Negatives_Required}

\subsubsection{} When the prenoun \N{fra-} is negated the verb is also
negated, \Npawl{ke frapo ke tslolam} \E{not everyone understood}.
\index{fra-@\textbf{fra-}!with \textbf{ke}}
\index{ke@\textbf{ke}!with \textbf{fra-}}
\Ultxa{3/10/2010}{https://wiki.learnnavi.org/index.php/Canon/2010/UltxaAyharyu\%C3\%A4\%23Ke_with_fra-}

\subsection{Kaw'it} A word or phrase may be singled out for negation
with \N{ke... kaw\ACC{'it}} \E{not... at all}, as in \N{fo ke lu 'ewan
kaw'it} \E{they are not young at all}.
\index{kaw'it@\textbf{kaw'it}}
\LNWiki{6/4/2010}{https://wiki.learnnavi.org/index.php/Canon/2010/March-June\%23April_6_Miscellany}


\section{Complex Sentences}

\subsection{Tense and Aspect in Dependent Subjunctives} \QUAESTIO{Do
dependent verbs have TAM-solidarity with their controlling verb?}

\subsection{Purpose} Purpose clauses take the conjunction \N{fte}
(negative \N{fteke}) with the subjunctive, \N{sawtute zera'u fte fol
Kelutralti skiva'a} \E{the sky people are coming to destroy Hometree;}
\Npawl{makto kawl, ma samsiyu, fte tsivun pivähem nìwin} \E{ride hard,
warriors, so you can get there fast!} \Npawl{tsun fko sivar hänit fte
payoangit stivä'nì} \E{one can use a net to catch a fish}.

\label{syn:purpose}\index{purpose clause}
\index{fte@\textbf{fte}}\index{fteke@\textbf{fteke}}

\subsubsection{} In Na'vi purpose clauses are used in several
situations where English would simply use an infinitive, \Npawl{pxiset
ke lu oeru krr \uwave{fte} tì'eyngit \uwave{tivìng}} \E{right now I
don't have time \uwave{to give} an answer}.

\subsection{Asyndeton} Short, parallel phrases\footnote{That is,
phrases following the same pattern of grammar.} may be joined
without a conjunction connecting them.  \Npawl{Yola krr, txana krr,
ke tsranten} \E{it doesn't matter how long it takes,} literally
\E{short time, long time, doesn't matter;} \Npawl{'uo a fpi rey'eng
\uwave{Eywa'eveng\-mì 'Rrtamì} tsranten nìtxan awngaru nìwotx}
\E{something that matters a lot to all of us for the sake of The
Balance of Life \uwave{on both Pandora and Earth;}} \Npawl{lora
aylì'u, lora aysäfpìl} \E{beautiful words and beautiful ideas}.
\index{asyndeton}\index{conjunctions!omitting}

\subsubsection{} \index{verb!sequential}
Two verbs in sequence without a conjunction are
sequential, \Npawl{za'u kaltxì si ko!} \E{come (and then) say hello!}
\Nfilm{ngari hu Eywa salew tirea, tokx \uwave{'ì'awn slu} Na’viyä hapxì}
\E{your spirit goes with Eywa and your body \uwave{remains and
becomes} part of the People,} \Npawl{pol tsatxumit noläk terkup} \E{he
drank the poison and died}.

Verbs controlled by a modal may also be sequenced in this way, with
the possibility that both might take the subjunctive.  Both of the
following are acceptible:

\begin{quotation}
\noindent\Npawl{Tsun nekll zivup tsawng.} \E{It can fall to the ground
and break}. \\
\noindent\Npawl{Tsun nekll zivup tsivawng.} \E{It can fall to the
ground and (can) break}.
\end{quotation}

\noindent Additional verbs can be introduced in a sequence by
inserting \N{tsakrr}. 
\LNForum{2/5/2020}{https://forum.learnnavi.org/language-updates/on-sequential-verbs-and-quoting/}
\NTeri{21/4/2020}{https://naviteri.org/2020/04/aawa-u-amip-a-few-new-things/}


\section{Relative Clauses and Phrase Attribution}
\index{relative clause}
\subsection{Particle ``A''} Na'vi relative clauses are created with
the attributive particle \N{a}.\index{a@\textbf{a}}\label{syn:a} As
with adjective attribution, a relative clause may either precede or
follow the word it modifies, \Npawl{\uwave{po tsane karmä a tsengit}
ke tsìme'a oel} \E{I didn't see \uwave{the place he was going to}},
\Npawl{\uwave{palulukan a teraron} lu lehrrap} \E{\uwave{a thanator
that's hunting} is dangerous.}

\subsubsection{} Note that the attributive \N{a} is a particle, not a
pronoun, and will not take case marking.

\subsection{Referential Hierarchy} When the head\footnote{The ``head''
of the relative clause is the noun to which the relative clause is
attached.  It has a syntactic role in both the main clause and the
relative clause.  For example, in the sentence \E{I see the man who is
running,} the word ``man'' is the direct object of the main clause ``I
see the man'' but is the subject of the relative clause ``the man is
running.''  This element common to both clauses is sometimes also
called a ``pivot.''} of a relative clause is the subject or direct
object in that relative clause, it is omitted,

\begin{quotation}
\noindent \N{\uwave{Ngal tse'a a tute} lu eyktan}.
  \E{\uwave{The man whom you see} is leader}.\\
\noindent \N{\uwave{Ngati tse'a a tute} lu eyktan}.
  \E{\uwave{The man who sees you} is leader}. 
\end{quotation}
\noindent For other cases or adpositional phrases, a resumptive
pronoun must be used --- \N{po} for animate heads and \N{tsaw} for
inanimates.
% Feb 18: https://wiki.learnnavi.org/index.php/Canon#More_extracts_from_various_emails

\begin{quotation}
\noindent \Npawl{poru mesyal lu a ikran} \E{an ikran with two wings}\\
\noindent \Npawl{Po \uwave{tsane} karmä a tsengit ke tsìme'a oel}.\\
\indent\E{I didn't see the place which she was going \uwave{(to it)}}.\\
\noindent \N{Fìpo lu tute a oe \uwave{pohu} perängkxo.} \\
\indent\E{This is the person who I'm talking \uwave{with (him)}}.
\end{quotation}

\subsubsection{} When the head of the relative clause is a direct
object in it, the subject of the verb must still take the agentive
marking, as in \N{\uwave{ngal} tse'a a tute} \E{the man whom you see}
from above, not *\N{\uwave{nga} tse'a a tute} and \Npawl{teylu a
\uwave{oel} yerom lu ftxìlor} \E{the teylu I'm eating is delicious}.
\NTeri{28/3/2012}{https://naviteri.org/2012/03/spring-vocabulary-part-1/}

\subsection{Other Attributive Phrases} Though English can
modify nouns directly with pre\-posi\-tional phrases (``the man on the
moon''), Na'vi attaches such phrases to nouns with \N{a}, as in
\Nfilm{fìpo lu \uwave{vrrtep a mì sokx atsleng}} \E{this is a
\uwave{demon in a false body,}} \N{ngeyä \uwave{teri faytele a
aysänumeri}} \E{your \uwave{instructions about these matters}}.
\index{adposition!attributive phrase}

\subsubsection{} The shades of colors can be made more precise with
the adposition \N{na} \E{like}.  To use such a phrase attributively
the entire phrase is hyphenated and treated like a normal adjective.
So, from \Npawl{ean na ta'leng} \E{(Na'vi-)skin-blue}:
\begin{quotation}
\indent\N{Fìsyulang \uwave{aean-na-ta'leng} lor lu nìtxan.}\\
\indent\N{Fìsyulang \uwave{ata'lengna-ean} lor lu nìtxan.}\\
\indent\N{\uwave{Ean-na-ta'lenga} fìsyulang lor lu nìtxan.}\\
\indent\N{\uwave{Ta'lengna-eana} fìsyulang lor lu nìtxan.}
\end{quotation}
\index{na@\textbf{na}!with colors}\label{syn:attr:na}

\subsubsection{} Single adverbs may also be used attributively,
\Npawl{ke zasyup lì'Ona ne kxutu a mìfa fu a wrrpa} \E{The l'Ona will
not perish to the enemy within or the enemy without}.
\index{adverbs!attributive}

\subsection{With Adjectives} \index{relative clause!with adjective}
A relative clause may follow an attributive adjective on the same side
of the noun,

\begin{interlin}
\glll Kaltxì, oeyä eylanur a'ewan a tok Toitslanti. \\
      kaltxì, oe-ä ay-'eylan-ur a-'ewan a tok Toitslan-ti \\
      hello, I-\I{gen} \I{pl}-friend-\I{dat} \I{lig}-young \I{rel} inhabit Germany-\I{pat}\\
\trans{Hello, my young friends from Germany.} \Ipawl{}
\end{interlin}

\noindent Another acceptable word order is \Npawl{Kaltxì, oeyä 'ewana
eylanur a tok Toitslanti}.
\NTeri{28/2/2021}{https://naviteri.org/2021/02/aysipawm-si-aysieyng-questions-and-answers/}

\subsection{Clause Nominalization} Entire clauses can be turned into
nouns and brought into the syntax of another sentence using the
attributive particle, with either \N{fì'u} or \N{tsa'u} to anchor the
phrase in the main clause.  This is common enough that certain
combinations of pronoun and attributive particle contract (see
\horenref{morph:fwa-tsawa}). \label{syn:clause-nom}
\index{fwa@\textbf{fwa}!use}
\index{fula@\textbf{fula}!use}
\index{futa@\textbf{futa}!use}
\index{furia@\textbf{furia}!use}

\subsubsection{} Just as in a relative clause, the anchor pronoun is
inflected to match its role in the main clause.  For example, in the
subjective (\N{fwa}) as the intransitive subject of \N{lu}:

\begin{quotation}
\noindent\Npawl{Law lu oeru \uwave{fwa nga mì reltseo nolume nìtxan}}.\\
\indent\E{It is clear to me \uwave{that you have learned much in art}}.
\end{quotation}

\noindent In the topical (\N{a fì'uri}) with \N{irayo si}:
\begin{quotation}
\noindent\Npawl{\uwave{Ngal oeyä 'upxaret aysuteru fpole' a fì'uri}, ngaru
  irayo seiyi oe nìtxan.}\\
\indent\E{I thank you very much \uwave{for sending my message to people}}.
\end{quotation}

\noindent As the direct object (\N{futa}) of the verb \N{omum}:
\begin{quotation}
\noindent\Npawl{Ulte omum oel \uwave{futa tìfyawìntxuri oeyä perey
    aynga nìwotx}.}\\
\indent\E{And I know \uwave{that you all are waiting for my guidance.}}
\end{quotation}

\subsubsection{} Very often particular verbs and idioms will require a
particular clause nominalization.  For example, subclauses with
\N{omum} \E{know} will generally take an accusative clause (usually
\N{futa} or \N{a fì'ut}).

\subsubsection{} Clauses may also be nominalized with forms of
\N{tsa'u}.  The difference between \N{fì'u} and \N{tsa'u} is that the
\N{tsa'u} form can be used when the clause it anchors refers to
something old in the discourse, something which has been previously
discussed.  This subtlety is not required, however, and forms in
\N{fì'u} are never wrong. \QUAESTIO{Example conversation using both?}
\index{tsawa@\textbf{tsawa}!use}
\index{tsata@\textbf{tsata}!use}
\index{tsaria@\textbf{tsaria}!use}
\LNWiki{18/6/2010}{https://wiki.learnnavi.org/index.php/Canon/2010/March-June\%23The_contrast_between_fwa.2Ftsawa.2C_furia.2Ftsaria}

\subsubsection{} The noun \N{tìkin} \E{need} is used with an
attributive clause for the idiom ``need to,'' \Npawl{awngaru lu tìkin
a nume nì'ul} \E{we need to learn more} (literally, ``we have the need
to learn more''). \index{tìkin@\textbf{tìkin}!with attributive clause}
It can also be used impersonally, \N{lu tìkin a \dots} \E{there is a
need to/for \dots}


\subsection{Nominalized Clauses with Adpositions} Nominalized clauses
may be used with some adpositions, giving sense that match certain
English conjunctions and gerund clauses.  \N{Oe ke tsun stivawm
fayfnelì'ut \uwave{luke fwa sngä'i tsngi\-vaw\-vìk}} \E{I cannot hear
such words \uwave{without starting to cry}}.\label{syn:rel:nom-adp}
\index{adpositions!with nominalized clauses}
% https://forum.learnnavi.org/language-updates/txelanit-hivawl/


\subsubsection{} \QUAESTIO{A list of legal ones might be nice. Other
likely candidates: \N{fpi}, \N{mìkam},  \N{pxel/na}?  Or the
dictionary may be the better place for a full list.}  

% Mungwrr fwa seen:
%  https://naviteri.org/2020/02/some-words-for-leap-year-day/
% Vat fwa seen:
%  https://naviteri.org/2012/03/spring-vocabulary-part-1/

\subsection{Nominalizations as Conjunctions} There are a few Na'vi
constructions involving nouns and the attributive particle that do
what English uses conjunctions for.  Because of this, what appear to
be identical conjunctions have two forms --- one for when the
conjunction comes at the end of a clause, and one for when it comes at
the start.  Often these phrases have contracted into one word,
sometimes with sound changes.

\begin{center}
\begin{tabular}{rlll}
 & At the start & At the end \\
\hline
\E{after} & \N{mawkrra} & \N{akrrmaw} & from \N{maw krr a} \\
\E{because} & \N{talun(a)} & \N{alunta} & from \N{ta lun a} \\
\E{because} & \N{taweyk(a)} & \QUAESTIO{\N{aweykta}} & from \N{ta oeyk a}\\
\E{when} & \N{krra} & \N{a krr} \\
\E{that} (as a result) & \N{kuma} & \N{akum} \\
\E{since} (from the time) & \N{takrra} & \N{akrrta} & from \N{ta krr a}\\
\end{tabular}
\end{center}\label{syn:attr:takrra}\label{syn:attr:kuma}
\index{mawkrra@\textbf{mawkrra}}\index{akrrmaw@\textbf{akrrmaw}}
\index{talun(a)@\textbf{talun(a)}}\index{alunta@\textbf{alunta}}
\index{taweyk(a)@\textbf{taweyk(a)}}\index{aweykta@\textbf{aweykta}}
\index{takrra@\textbf{takrra}}\index{akrrta@\textbf{akrrta}}
\index{kuma@\textbf{kuma}}\index{akum@\textbf{akum}}
\index{krr@\textbf{krr}!with attributive \N{a}}
\NTeri{31/3/2012}{https://naviteri.org/2012/03/spring-vocabulary-part-2/}
\NTeri{19/6/2012}{https://naviteri.org/2012/06/spring-vocabulary-part-3/}

\noindent \Npawl{Tì'eyngit oel tolel \uwave{a krr}, ayngaru payeng}
\E{\uwave{when} I receive an answer, I will let you know} could also
be \N{krra tì'eyngit oel tolel, \dots}
\LNWiki{1/2/2010}{https://wiki.learnnavi.org/index.php/Canon\%23Some_Conjunctions_and_Adverbs}
\LNWiki{1/2/2010}{https://wiki.learnnavi.org/index.php/Canon\%23Extracts_from_various_emails}
\NTeri{15/8/2011}{https://naviteri.org/2011/08/new-vocabulary-clothing/comment-page-1/\%23comment-986}



\section{Conditional Sentences}
\noindent Na'vi conditional sentences are introduced with the
conjunction \N{txo} \E{if} for the condition, and the consequent
optionally by \N{tsakrr} \E{then}.
\label{syn:conditionals}\index{conditional sentence}
\index{txo@\textbf{txo}}\index{tsakrr@\textbf{tsakrr}}

\subsection{General} General conditions describe situations that are
commonly or generally true, such as ``if it doesn't rain, plants and
animals suffer.''  In Na'vi, a \QUAESTIO{general condition takes
\N{txo} with the subjunctive in the condition and a non-future
indicative in the consequent,} \Npawl{txo fkol ke fyivel uranit paywä,
zene fko slivele} \E{if one does not seal a boat against water, one
must swim.}
\index{conditional sentence!general}
\NTeri{19/6/2012}{https://naviteri.org/2012/06/spring-vocabulary-part-3/}

\subsection{Future Conditional} In English future conditionals have
the present tense in the con\-dition and the future in the consequent,
``If you do this, I will do that.''  In Na'vi, the condition takes the
subjunctive and the consequent takes the future, \Npawl{pxan
\uwave{l\INF{iv}u} txo nì'aw oe ngari / Tsakrr nga Na'viru
\uwave{yomt\INF{ìy}ìng}} \E{Only if I \uwave{am} worthy of you /
\uwave{Will} you feed the People}. \index{conditional sentence!future}

\subsection{Hypothetical} \QUAESTIO{No examples yet.}
\index{conditional sentence!hypothetical}

\subsection{Use of the Subjunctive} \index{conditional sentence!subjunctive}
The use of the subjunctive in the \N{txo} clause depends on the status
of the condition.  That is, if the speaker knows for certain that a
situation holds, when \N{txo} has a bit of meaning ``since,'' then the
subjunctive is not required.  For example, in ex.\ref{txo:subj:ex01}
the speaker does not really know if the person they are talking to is
tired. 

\begin{interlin} \label{txo:subj:ex01}
\glll Txo nga ngeyn livu, tsurokx. \\
      txo nga ngeyn l\INF{iv}u, tsurokx \\ if you tired be‹\I{subj},
rest \\
\trans{If you're tired, rest.}
\end{interlin}

\noindent But if the speaker has heard someone say that they're tired,
they can drop the subjunctive, as in ex.\ref{txo:subj:ex02},

\begin{interlin} \label{txo:subj:ex02}
\glll Txo nga ngeyn lu, tsurokx. \\
      txo nga ngeyn lu, tsurokx \\ if you tired be‹\I{subj},
rest \\
\trans{Since you're tired, rest.}
\end{interlin}

\noindent When to use the subjunctive is somewhat flexible, and there
will be situations where either use can be justified.
\LNForum{14/1/2022}{https://forum.learnnavi.org/language-updates/use-of-the-subjunctive-after-txo/}

\subsection{Contrafactual} Contrafactual questions use a separate set
of conjunctions, \N{zun} \E{if} and \N{zel} \E{then}.  The subjunctive
is used in both clauses, with the following tense senses:
\index{conditional sentence!contrafactual}
\index{zun@\textbf{zun}}\index{zel@\textbf{zel}}

\begin{center}
\begin{tabular}{lll}
Past & Present & Future \\
\hline
  \N{\INF{imv}}, \N{\INF{ilv}} & 
  \N{\INF{iv}}, \N{\INF{irv}} & 
  \N{\INF{iyev}}
\end{tabular}
\end{center}

\noindent So, for present situations the bare subjunctive or the
imperfective subjunctive is used, for past situations the past or
perfective subjunctive is used, and finally for future situations the
future subjunctive is used (see \horenref{morph:verb:first-position}
for the infix forms).

\begin{quotation}
\noindent\Npawl{Zun oe yawne livu ngar, zel 'ivefu oe nitram nì'aw.}\\
\indent\E{If you loved me, I would be so happy.}\\
\noindent\Npawl{Zun oe yawne limvu ngar, zel 'imvefu oe nitram nì'aw.}\\
\indent\E{If you had loved me, I would have been so happy.}\\
\noindent\Npawl{Zun tompa zìyevup trray, zel fo srìyevew.}\\
\indent\E{If it rained tomorrow, they'd do a dance.}\\
\noindent\Npawl{Zun ayoe livu tsamsiyu, zel tsakem ke simvi.}\\
\indent\E{If we were warriors, we wouldn't have done that.}
\end{quotation}

\noindent When the time of both clauses is the same, and only then,
the verb in the \N{zel} clause may take the bare verb, without the
subjunctive,

\begin{quotation}
\noindent\Npawl{Zun oe yawne livu ngar, zel 'efu oe nitram nì'aw.}\\
\indent\E{If you loved me, I would be so happy.}\\
\noindent\Npawl{Zun oe yawne limvu ngar, zel 'efu oe nitram nì'aw.}\\
\indent\E{If you had loved me, I would have been so happy.}\\
\noindent\Npawl{Zun tompa zìyevup trray, zel fo srew.}\\
\indent\E{If it rained tomorrow, they'd do a dance.}
\end{quotation}
\NTeri{4/30/2013}{https://naviteri.org/2013/04/zun-zel-counterfactual-conditionals/}

\subsection{Imperatives in Conditions} When imperatives are used as
the consequent of a condition, imperative mood and syntax rules
override the normal conditional patterns.  For example, a future
conditional with imperative consequent, \Npawl{txo \uwave{tsive'a}
ayngal keyeyt, rutxe oeru \uwave{piveng}} \E{if you see errors,
please tell me}.  \index{imperative!in conditional sentences}


\section{Conjunctions}
\noindent This section lists conjunctions that have not been discussed
elsewhere, but which still deserve mention in some way.  I omit
conjunctions that require no special comment.

\subsection{Alu} The primary use of \N{alu} is for nouns in
apposition, \Npawl{tskalepit oel tolìng oeyä \uwave{tsmu\-ka\-nur alu
Ìstaw}} \E{I gave the crossbow to \uwave{my brother Istaw}}.  Note
that the noun after \N{alu} is in the subjective case.
\NTeri{16/7/2010}{https://naviteri.org/2010/07/vocabulary-update/}
\index{alu@\textbf{alu}}\label{syn:conj:alu}\index{apposition}

\subsubsection{} \N{Alu} may also be used conversationally to mark a
restatement, like ``that is to say,'' or ``in other words.''
\Npawl{Txoa livu, yawne lu oer Sorewn...\ \uwave{alu}...\ ke tsun oeng
muntxa slivu} \E{Sorry, but I love Sorewn... \uwave{in other words},
you and I cannot marry.}

\subsubsection{} In discussions of grammar and language, \N{alu} can
clarify the word or construction you're speaking about,
\Npawl{tsalsungay tsalì’u \uwave{alu zeykuso} lu eyawr}
\E{nonetheless, that word, \uwave{namely \N{zeykuso}}, is correct},
\Npawl{lì’uri alu tskxe pamrel fyape?} \E{how do you spell the word `tskxe'?}

\subsection{Ftxey} In addition to forming yes-no questions
(\N{ftxey... fuke}, \horenref{syn:question:ftxey}), \N{ftxey} can be
used to enumerate \E{whether... or...} options, \Npawl{sìlpey
oe, \dots\ frapo --- \uwave{ftxey sngä'iyu ftxey tsulfätu} ---
tsìyevun fìtsenge rivun 'uot lesar} \E{I hope ... everyone
--- \uwave{whether beginner or expert} --- will be able to find
something useful here}. 
\index{ftxey@\textbf{ftxey}}

\subsection{Fu} The conjunction \N{fu}, \E{or}, may be used to combine
either noun phrases or verb phrases.  \Npawl{Ke zasyup lì'Ona ne kxutu
a mìfa fu a wrrpa} \E{The l'Ona will not perish to the enemy within or
the enemy without;} \Npawl{rä'ä fmivi livok fu emkivä ayekxanit a fkol
ngolop fpi sìkxuke ayfrrtuä sì ayioangä} \E{do not attempt to approach
or cross any barriers designed for Guest and animal safety}.
\index{fu@\textbf{fu}}

\subsubsection{} For a choice statement, use \N{fu} once.
For a choice  question, use \N{fu} twice, before each of the two choices.
\Npawl{Nulnew oel fì’ut \uwave{fu} tsa’ut} \E{I want this \uwave{or} that}
vs.\ \Npawl{nulnew ngal \uwave{fu} fì’ut \uwave{fu} tsa’ut?}
\E{Do you want this, or do you want that? What’s your choice?}

\subsection{Ki} The conjunction \N{ki}, \E{but rather, but instead},
is paired with the negative adverb \N{ke}.  Take care to distinguish
this from \N{slä} \E{but}.  \Npawl{Nga plltxe ke nìfyeyntu ki
nì'eveng} \E{you speak not like an adult but a child.}
\NTeri{16/7/2010}{https://naviteri.org/2010/07/vocabulary-update/}
\index{ki@\textbf{ki}}

\subsection{Sì} The conjunction \N{sì} \E{and} is used for making lists
and combining elements of the same idea.  It is not used to join
clauses, which is the job of \N{ulte} (\horenref{syn:ulte}).
\Npawl{Lu pìlokur pxesìkan sì pxefne’upxare} \E{the blog has three
needs and three sorts of message,} \Npawl{ma smukan sì smuke}
\E{brothers and sisters.}  \index{siì@\textbf{sì}}\label{syn:sì}

\subsubsection{} Though \N{sì} is most often found joining noun
phrases, pronouns and adjectives, it can join verbs that are closely
related, \Nfilm{sänume sivi poru fte \uwave{pivlltxe sì
tivìran} nìayoeng} \E{teach him to \uwave{speak and walk} like us}.

\subsubsection{} Clauses that have been nominalized, such as
with \N{fwa}, \N{futa}, etc. (\horenref{syn:clause-nom}), may be
joined to a list of nouns with \N{sì}, too, as in \N{sunu poru syulang
sì mauti sì fwa tswayon yaka} \E{he likes flowers, fruits, and to fly
through the air}.
\LNWiki{23/1/2018}{https://wiki.learnnavi.org/Canon/2018}

\subsubsection{} \N{Sì} can also be enclitic
(\horenref{l-and-s:stress:enclisis}).  In that situation it follows
the word or phrase it is joining to the list, \Npawl{ta 'eylan
\uwave{karyusì} ayngeyä, Pawl} \E{from your friend \uwave{and
teacher}, Paul,} \Npawl{tsakrr paye'un sweya fya'ot a zamivunge
oel ayngar aylì'ut \uwave{horentisì} lì'fyayä leNa'vi} \E{and I will
then decide the best way to bring you the words \uwave{and rules} of
Na'vi}.  \index{siì@\textbf{sì}!enclitic}

%\subsection{Tengfya} \QUAESTIO{Needed?}

\subsection{Tengkrr} The sense of \N{tengkrr}, \E{while, the same time
as} requires it to be used with the imperfective, \Npawl{fìtxon yom
\uwave{tengkrr teruvon}} \E{on this night (we) eat \uwave{while leaning}}.
\index{tengkrr@\textbf{tengkrr}}
\LNWiki{14/3/2010}{https://wiki.learnnavi.org/index.php/Canon/2010/March-June\%23A_Collection}

\subsection{Tìk} \index{tìk@\textbf{tìk}} \label{syn:tìk}
This adverb meaning \E{immediately, without delay,} can also be used
as a conjunction indicating that a second action immediately follows a
first, \Npawl{fìioang ke tsun slivele; nemfapay zup tìk
spakat} \E{this animal cannot swim; (if) it falls into the water, it
immediately drowns}.
\NTeri{31/12/2021}{https://naviteri.org/2021/12/zolau-niprrte-ma-3746-welcome-2022/}

\subsection{Tsnì} \label{syn:tsni}\index{tsnì@\textbf{tsnì}}
The conjunction \N{tsnì} \E{that} introduces some kinds of report
clause, \N{ätxäle si tsnì livu oheru Uniltaron} \E{I respectfully
request the Dream Hunt}, \Npawl{sìlpey oe tsnì fìtìoeyktìng law livu
ngaru set} \E{I hope that this explanation is clear to you now}.
\NTeri{20/2/2011}{https://naviteri.org/2011/02/new-vocabulary-part-2/}

\subsubsection{} Verbs known to take \N{tsnì}: \N{ätxäle si},
\N{rangal} (a marginal use), \N{sìlpey}, \N{la'um}, \N{mowar
si}, \N{fe'pey}, \N{leymfe'}, \N{leymkem}, \N{srefey},
and \N{srefpìl}.  Some verbs, such as \N{sìlpey} \E{hope,} require
the \N{tsnì} clause to take the subjunctive, while others, such
as \N{la'um} \E{pretend,} will not take the subjunctive.  The
dictionary is the best place to verify.
\LNForum{18/8/2011}{https://forum.learnnavi.org/language-updates/confirmation-on-use-of-rangal/}
\NTeri{1/3/2020}{https://naviteri.org/2020/02/some-words-for-leap-year-day/\#comment-30895}

\subsection{Ulte} This conjunction connects clauses, \N{oel ngati
kameie, ma tsmu-kan, \uwave{ulte} ngaru seiyi irayo} \E{I see you,
brother, \uwave{and} thank you}.  Do not confuse with \N{sì}
(\horenref{syn:sì}).
\index{ulte@\textbf{ulte}}\label{syn:ulte}



\section{Direct Quotation}
\label{syn:direct-quote}

\subsection{San... sìk} Na'vi does not have indirect quotes (\E{He
said \uwave{that they were gone}}), but instead uses direct
quotation, with the quoted words put between the particles \N{san} and
\N{sìk}, as in \Npawl{slä nì'i'a tsun oe pivlltxe \uwave{san Zola'u
nìprrte' ne pìlok Na'viteri sìk}!} \E{but now I can finally say
``\uwave{welcome to the blog Na'viteri}.''}\index{quotation!direct}
\index{san@\textbf{san}}\index{siìk@\textbf{sìk}}
\NTeri{31/8/2011}{https://naviteri.org/2011/08/reported-speech-reported-questions/}

\subsubsection{} If the beginning or end of a quotation coincides with
the beginning or end of an utterance, one or the other of the
\N{san... sìk} pair can be dropped.

\begin{quotation}
\noindent 1. \Npawl{Poltxe Eytukan \uwave{san} oe kayä \uwave{sìk}, slä oel pot ke spaw.}\\
\indent\E{Eytukan said he would go, but I don't believe him.}

\noindent 2. \N{Poltxe Eytukan \uwave{san} oe kayä.}\\
\indent\E{Eytukan said he would go.}
\end{quotation}

\noindent In (2), since nothing is said after the quote, there is no
need to close the quotation with \N{sìk}.  Similarly, \N{san} can be
dropped if there no serious ambiguity, \Npawl{frawzo sìk, slä oel poet
ke spaw} \E{(she said) ``all's well,'' but I don't believe her}. 
\LNWiki{21/1/2010}{https://wiki.learnnavi.org/index.php/Canon\%23Extracts_from_various_emails}
\LNForum{4/8/2020}{https://forum.learnnavi.org/language-updates/on-omitting-san/}

\subsection{Questions} Reported questions are also quoted directly,
\Npawl{polawm po san srake Säli holum sìk} \E{he asked whether Sally
left,} literally \E{he asked, ``did Sally leave?''}
\LNWiki{24/3/2010}{https://wiki.learnnavi.org/index.php/Canon/2010/March-June\%23If_and_Whether}

\subsubsection{} With \N{pawm}, but not other verbs of speaking,
\N{san... sìk} may be dropped, \Npawl{Polawm po, Neytiri kä pesengne?}
\E{he asked where Neytiri was going.}
\NTeri{31/8/2011}{https://naviteri.org/2011/08/reported-speech-reported-questions/}


\subsection{Transitivity} When a verb of speaking uses \N{san...  sìk}
it follows intransitive syntax, \N{po poltxe san srane} \E{she said
``yes.''} \index{transitivity!with verbs of speaking}
\Ultxa{2/10/2010}{https://wiki.learnnavi.org/index.php/Canon/2010/UltxaAyharyu\%C3\%A4\%23Transitivity_with_Speaking_Verbs}

\subsubsection{} When the speaking verb has a direct object, it
follows transitive syntax, \Npawl{ke poltxe pol tsaylì'ut} \E{she
didn't say that}, \N{oel poru pasyawm tsat} \E{I will ask him that}.
\NTeri{31/8/2011}{https://naviteri.org/2011/08/reported-speech-reported-questions/}

\subsection{Quotation Nominalization} In addition to the
\N{san... sìk} pair, reported speech may be anchored to the nouns
\N{fmawn} \E{news}, \N{tì'eyng} \E{answer} and \N{faylì'u} \E{these
words} with the attributive \N{a} (see \horenref{morph:fmawn} for
contractions).
\NTeri{31/8/2011}{https://naviteri.org/2011/08/reported-speech-reported-questions/}

\begin{center}
\begin{tabular}{ll}
Verb & Quotation \\
\hline
\N{plltxe} \E{say} & \N{san... sìk}, \N{faylì'u} \\
\N{stawm} \E{hear}, \N{peng} \E{tell} & \N{fmawn} \\
\N{pawm} \E{ask} & \N{san... sìk}, \N{tì'eyng}, nothing \\
\N{vin} \E{ask (for)} & \N{tì'eyng} 
\end{tabular}
\end{center}
\index{pawm@\textbf{pawm}}\index{stawm@\textbf{stawm}}
\index{plltxe@\textbf{plltxe}}\index{vin@\textbf{vin}}

\noindent The quotations attached to these are still in the direct
form,

\begin{quotation}
\noindent\Npawl{Poltxe pol fayluta oe new kivä.} \E{She said she
  wanted to go}.\\
\indent Lit., ``she said, `I want to go.'''\\
\noindent\Npawl{Ngal poleng oer fmawnta po tolerkup.}
 \E{You told me that he died.}\\
\noindent\N{Volin pol tì’eyngit a Neytiri kä pesengne.} \\
\indent\E{He asked where Neytiri was going.}
\end{quotation}
\label{syn:quot:nominalized}

\subsubsection{} Other verbs introducing indirect questions may use
\N{tì'eyng} nominalizations,

\begin{quotation}
\noindent\Npawl{Ke omum oel teyngta fo kä pesengne.}\\
\indent\E{I don’t know where they’re going.} \\
\noindent\Npawl{Teynga lumpe fo holum ke lu law.} \\
\indent\E{It’s not clear why they left.}
\end{quotation}


\section{Particles}

\subsection{Ko} The sentence-final particle \N{ko} is used to solicit
agreement of various sorts, including such senses as ``let's,''
``don't you think?,'' ``why don't you? why don't I?''  Often heard in
the film, \N{makto ko} \E{let's ride}.
\index{ko@\textbf{ko}}\label{syn:particle:ko}

\subsubsection{} In the special case of \N{siva ko} \E{rise to the challenge}, 
the phrase can be written as a single word: \N{sivako}.
\NTeri{3/8/2019}{https://naviteri.org/2016/06/mrrvola-lifyavi-amip-forty-new-expressions/comment-page-1/\%23comment-30185}

\subsection{Nang} This particle marks surprise, exclamation or
encouragement.  It is always sentence-final and appears with adverbs
of degree or approbation, such as \N{nìngay}, \N{nìtxan},
\N{fìtxan}, etc. \N{Txantsana sìpawm apxay fìtxan lu ngaru nang!}
\E{you have so many excellent questions!}  \Npawl{Ngari tswintsyìp
sevin nìtxan lu nang!} \E{what a pretty little queue you have!}
\index{nang@\textbf{nang}}

\subsection{Pak} This particle follows the word it goes with and marks
disparagement, \N{tsamsiyu pak!} \E{a warrior? yeah, right!}.
\index{pak@\textbf{pak}}

\subsection{Tut} This is a particle of continuation, so far only seen
in pick-up questions in dialogs, 

\begin{quotation}
\noindent A: \N{Ngaru lu fpom srak?} \E{How are you?} \\
\noindent B: \N{Oeru lu fpom.  \uwave{Ngaru tut?}} \E{I'm well.  \uwave{You?}}
\end{quotation}
\index{tut@\textbf{tut}}

\subsection{Tse} This particle is a marker of conversational
hesitation, \E{well}. \QUAESTIO{In English ``well'' relates to
felicity conditions in divergent ways.}
\index{tse@\textbf{tse}}


\section{Other Notable Words}

\subsection{Sweylu} \label{syn:sweylu} The syntax of this verb meaning
``should'' (from \N{swey lu} \E{it's best}) changes depending on
whether the obligation refers to something that has not yet happened or
if it refers to an event that has already taken place.
\index{sweylu@\textbf{sweylu}}

\subsubsection{} To refer to the future, \N{txo} with the subjunctive
is used, \Npawl{sweylu txo \uwave{nga} kivä} or \N{\uwave{nga} sweylu
txo kivä} for \E{\uwave{you} should go}.  Note that the negative is
in the \N{txo} clause, \Npawl{sweylu txo nga ke kivä} or \N{nga sweylu
txo ke kivä} \E{you shouldn't go}.

\subsubsection{} For something that has already happened, use \N{fwa}
or \N{tsawa} with a past or perfective indicative,

\begin{quotation}
\noindent Tsenu: \Npawl{Spaw oe, fwa po kolä längu kxeyey.}\\
\noindent\E{I believe it was a mistake for him to go/have gone.} \\

\noindent Kamun: \N{Kehe, kehe! Sweylu fwa po kolä!}\\
\noindent\E{No, no! He should have gone!}
\end{quotation}

\noindent Note that this refers to a past event that did happen and
was the right thing to do, not an unfulfilled past action (which is
another use of ``should'' in English).
\NTeri{5/4/2011}{https://naviteri.org/2011/04/\%e2\%80\%99a\%e2\%80\%99awa-li\%e2\%80\%99fyavi-amip\%e2\%80\%94a-few-new-expressions/}
