\nchapter{Introduction}

We do not yet have an official grammar of the Na'vi language, written
by Paul Frommer and blessed by the financial and intellectual property
Powers that Be at Lightstorm Entertainment or 20th Century Fox.  As of
this writing\footnote{July 2010} it does not seem likely we'll be
getting one very soon. In light of that, I decided to turn a grammar
summary I wrote into a longer document.

Like that grammar summary, this document will not teach you Na'vi.
Instead, it is intended to provide a concise and accurate reference on
the current state of our knowlege about the language.  It is based on
all the analytical work that has gone on in the months and years since
the film was released, as well as any communications from Frommer
which clarify language points.

I rely heavily on the Corpus and Canon wiki pages at LearnNavi.org,
without which resources this document would not be possible.  The
recent appearance of Frommer's own blog has also provided
material.\footnote{In late June 2010, \url{https://naviteri.org}}


\section{History of Decipherment}
It is important for newcomers to Na'vi to understand how it is we know
what we know about the Na'vi language.

Our earliest hints about the language came out in interviews with
Frommer in December of 2009, leading up to the release of the film.
Na'vi had ejective consonants.  It had a tripartite case division.  We
had a few phrases.

The big break came when someone among the IMDB refugees on their own
forum posted the Na'vi word
list.\footnote{\url{https://kcbluesman.websitetoolbox.com/post?id=4013403}
requires login}  It was transcribed from the \textit{Activist
Survival Guide}.\footnote{Wilhelm, Maria; Mathison, Dirk (2009). \textit{James
Cameron's Avatar: A Confidential Report on the Biological and Social
History of Pandora (An Activist Survival Guide),} It Books (HarperCollins).}
That list was republished in a public blog post on December
11th.\footnote{\url{http://www.suburbandestiny.com/?p=611}} All
current dictionaries are based on that initial post.  So, now we had
enough vocabulary to start analyzing the sentences coming out in
Frommer's interviews.

On December 15th, in an interview with the UGO Movie
Blog\footnote{\url{https://web.archive.org/web/20100818193039/http://www.ugo.com/movies/paul-frommer-interview}}
we got for the first time that fundamental Na'vi greeting, \N{oel
ngati kameie} \E{I See you}.  This was in addition our first
sighting of the agentive and patientive case endings.  Thanks to the
dictionary, we could guess \N{-l} for agentive and \N{-ti} for
patientive.

Our next big break came a few days later, with the Language Log guest
blog post on December
19th.\footnote{\url{https://languagelog.ldc.upenn.edu/nll/?p=1977}}
This is still fundamental reading for every student of Na'vi.  In it
we learn a good deal about the Na'vi sound system.  It also told us
enough about Na'vi grammar to guide all our future analysis of
the examples coming out in interviews.

Even now, much of what we know has come not from Frommer directly
telling us, for example, ``this is the genitive case ending,'' but by
him saying in an interview that there is a genitive, and people using
that information to analyze Na'vi language examples.  Some of the
early analysis was incomplete, which has led to some confusion,
especially about case endings.  Our earliest examples of the genitive
were all in \N{-yä}.  Only later did we see evidence of the \N{-ä}
ending.  One can still find older documentation giving the genitive as
\N{-yä} only.

In the months since then, Frommer himself has provided larger examples
of Na'vi, each of which has been analyzed in great detail in order to
extract as much grammatical information as possible.  Frommer has also
answered some direct questions about the language.  This often
confirms what we suspected from analysis, sometimes corrects what we
thought we know, and sometimes gives us new information.

I have tried as much as possible to ensure that everything in this
grammar is confirmed directly by Frommer himself or, absent that, by
giving enough examples from Frommer's own Na'vi to make the case for
the grammatical point being explained.  Nonetheless, this document is
necessarily provisional.  It is Frommer's prerogative to tweak and
update the language in light of his own understanding of the
language's needs, to correct misconceptions that may have escaped his
notice until now, and to fill in grammatical gaps as he gets to them.
We must also assume that future \textit{Avatar} movies will alter the
Na'vi language in unexpected ways, not only to satisfy Cameron's
demands for his movies, but from the inevitable changes a created
language undergoes when actors finally speak it on the set.


\section{Notation and Conventions}

Na'vi text is given in bold face type and English translations in
italics, \N{fìfya} \E{thus}.

When a Na'vi example comes directly and unmodified from the
interviews, email or blog of Paul Frommer there will be an
$\mathcal{F}$ floating in the margin, as in \Npawl{kìyevame}.  The
\textit{Hunt Song} and the \textit{Weaving Song} from the
\textit{Activist Survival Guide} are also so marked.  Examples from
the movie use $\mathcal{A}$.

This work uses the digraphs \N{ts} and \N{ng} instead of the
scientific orthography Frommer developed (\horenref{l-and-s:cg}).  The
majority of people are more familiar with the digraph system.

In Frommer's original documentation for the actors stress accent was
indicated by underlining the stressed syllable.  This grammar follows
that practice, as in \N{\ACC{tu}te} \E{person} vs. \N{tu\ACC{te}}
\E{woman}.  To avoid confusion with Frommer's accenting convention,
this document uses a wavy \uwave{underline} to draw attention to parts
of words or phrases.

Following the usual convention in technical linguistics works,
examples that are hypothetical or have some sort of error are marked
with a leading asterisk, *\N{m'resh'tuyu}.  Prefixes are indicated by
putting a dash at the end of the prefix, as in \N{fì-}.  Leniting
prefixes (\horenref{l-and-s:lenition}) use a plus sign, as in \N{ay+}.
Suffixes are indicated with a leading dash, \N{-it}, and infixes with
small brackets, \N{\INF{ol}}.  Transcription using the International
Phonetic Alphabet goes between square brackets, [fɪ.ˈfja].

When quoting one of the four songs Frommer translated for the film, I
use a single slash to separate lines, \N{Rerol tengkrr kerä
/ Ìlä fya’o avol}.

Starting in September of 2011, links to citations for grammatical
points are included for new material.  They occur at the end of a
section, and look like this:
\NTeri{11/7/2010}{https://naviteri.org/2010/07/diminutives-conversational-expressions/}.
Note that the dates follow European convention, Day/Month/Year.
``NT'' is for Frommer's blog, including his replies in comments,
``Wiki'' is for the LN.org Wiki, ``Forum'' is the LN.org forum, and
``Ultxa'' is for the October 2010 meeting.  There remain gaps in
citations for some areas, and I fill these in as I notice them.

Text \QUAESTIO{in maroon} is for matters that seem to me to be serious
questions about the language but for which no answer is currently
available.  Some will require simply confirmation from Frommer, others
will require much deeper thought and work on his part.  This grammar
aspires to someday be maroon-free.

\bigskip

\bigskip
Thanks are due to LearnNavi.org members `Eylan Ayfalulukanä, Taronyu
and Ftiafpi for looking at drafts of this grammar and making
suggestions.  I did not always follow their advice, so any flaws are
my own.

Thanks are also due to everyone who has commented and suggested
corrections since this grammar first appeared.

\bigskip
