% Sun Dec  2 12:35:11 2012 - REORGANIZE the charts to look more like
% https://forum.learnnavi.org/navi-lernen/das-navi-zahlensystem/
\nchapter{Numbers}
\noindent The Na'vi language has an \textit{octal}, or base eight,
number system, like a very small number of Human
languages.\footnote{Apparently a result of counting not the fingers but
  the spaces between them.}  Rather than calculating numbers in the
form $(m \times 10) + n$ (as in $(4 \times 10) + 2 = 42_{10}$,
\E{forty-two}), the numbers are calculated from $(m \times 8) + n$
(as in $(5 \times 8) + 2 = 52_8$, \N{mrrvomun}, $42_{10}$).

\section{Cardinal Numerals}
\index{numbers!cardinal}

\subsection{The ``Ones''} The independent forms of the numerals from
one to eight are:

\begin{center}
\begin{tabular}{ll}
1 & \N{'aw} \\
2 & \N{\ACC{mu}ne} \\
3 & \N{pxey} \\
4 & \N{tsìng} \\
\end{tabular}
\hskip 3em
\begin{tabular}{ll}
5 & \N{mrr} \\
6 & \N{\ACC{pu}kap} \\
7 & \N{\ACC{ki}nä} \\
8 & \N{vol} \\
\end{tabular}
\end{center}

\subsection{Powers of Eight} Rather than ``tens,'' Na'vi has
``eights:''

\begin{center}
\begin{tabular}{ll}
8 (1 $\times$ 8) & \N{vol} \\
16 (2 $\times$ 8) & \N{\ACC{me}vol} \\
24 (3 $\times$ 8) & \N{\ACC{pxe}vol} \\
32 (4 $\times$ 8) & \N{\ACC{tsì}vol} \\
\end{tabular}
\hskip 3em
\begin{tabular}{ll}
40 (5 $\times$ 8) & \N{\ACC{mrr}vol} \\
48 (6 $\times$ 8) & \N{\ACC{pu}vol} \\
56 (7 $\times$ 8) & \N{\ACC{ki}vol} \\
64 (8 $\times$ 8) & \N{zam} \\
\end{tabular}
\end{center}

\noindent The higher powers of eight are \N{\ACC{vo}zam} (512, octal
1000) and \N{\ACC{za}zam} (4096, octal 10000).

\subsection{Dependent Forms} When combined with powers of eight words,
the basic number words take abbreviated, single-syllable forms, with
lenition where possible: \label{numbers:dependent} \index{lenition!numbers}

\begin{center}
\begin{tabular}{ll}
1 & \N{(l)-aw} \\
2 & \N{-mun} \\
3 & \N{-pey} \\
4 & \N{-sìng} \\
\end{tabular}
\hskip 3em
\begin{tabular}{ll}
5 & \N{-mrr} \\
6 & \N{-fu} \\
7 & \N{-hin} \\
\end{tabular}
\end{center}

\subsubsection{} All of the dependent forms except ``one'', \N{(l)-aw},
evict the final \N{-l} of the ``eights'' forms.  Similarly, the
final \N{-m} in the \N{zam, vozam,} and \N{zazam} forms is dropped
before all of the forms except ``one,'' \N{zamaw,} but \N{za\ACC{mun},
za\ACC{pey}}, etc.

\subsubsection{} The attached dependent forms take the word accent.
Combined with \N{vol} \E{eight}: 

\begin{center}
\begin{tabular}{ll}
9 (1$\times$8 $+$ 1) & \N{vo\ACC{law}} \\
10 (1$\times$8 $+$ 2) & \N{vo\ACC{mun}} \\
11 (1$\times$8 $+$ 3) & \N{vo\ACC{pey}} \\
12 (1$\times$8 $+$ 4) & \N{vo\ACC{sìng}} \\
\end{tabular}
\hskip 3em
\begin{tabular}{ll}
13 (1$\times$8 $+$ 5) & \N{vo\ACC{mrr}} \\
14 (1$\times$8 $+$ 6) & \N{vo\ACC{fu}} \\
15 (1$\times$8 $+$ 7) & \N{vo\ACC{hin}} \\
16 (2$\times$8 $+$ 0) & \N{\ACC{me}vol} \\
\end{tabular}
\end{center}

\noindent The pattern will continue this way with \N{\ACC{me}vol}:
\N{mevo\ACC{law}}, \N{mevo\ACC{mun}}, \N{mevo\ACC{pey}}, etc.

After \N{zam} the count goes: \N{zam, za\ACC{maw}, za\ACC{mun},
za\ACC{pey}, za\ACC{sìng}, za\ACC{mrr}, za\ACC{fu},
za\ACC{hin}, \ACC{za}vol}, and then continuing as \N{zavo\ACC{law}},
etc.  For example, octal 211 is \N{mezavolaw}.
\NTeri{1/4/2014}{https://naviteri.org/2014/03/value-and-worth/\#comment-2678}
\LNForum{27/1/2021}{https://forum.learnnavi.org/language-updates/cardinal-zam-vozam-and-zazam/}

\section{Ordinal Numbers}

\subsection{Suffix -ve} The ordinal numbers are formed by means of the
suffix \N{-ve}, which does not alter the word accent, though it does
cause changes to a few number stems. \index{-ve@\textbf{-ve}}
\index{numbers!ordinal}

\begin{center}
\begin{tabular}{rll}
Ordinal & Independent & Dependent \\
\hline
first & \N{\ACC{'aw}ve} & \N{(l)-\ACC{aw}ve} \\
second & \N{\ACC{mu}ve} & \N{-\ACC{mu}ve} \\
third & \N{\ACC{pxey}ve} & \N{-\ACC{pey}ve} \\
fourth & \N{\ACC{tsì}ve} & \N{-\ACC{sì}ve} \\
fifth & \N{\ACC{mrr}ve} & \N{-\ACC{mrr}ve} \\
sixth & \N{\ACC{pu}ve} & \N{-\ACC{fu}ve} \\
seventh & \N{\ACC{ki}ve} & \N{-\ACC{hi}ve} \\

\end{tabular}
\hskip2em
\begin{tabular}{rll}
\\
eighth & \N{\ACC{vol}ve} & \N{-\ACC{vol}ve} \\
64th & \N{\ACC{za}ve} & \N{-\ACC{za}ve} \\
512th & \N{vo\ACC{za}ve} & \N{-vo\ACC{za}ve} \\
4096th & \N{za\ACC{za}ve} & \N{-za\ACC{za}ve} \\ 
\end{tabular}
\end{center}

\subsubsection{} Ordinal numbers are treated as adjectives, and
take \N{-a-} when used attributively (\horenref{morph:adj-attr}), as
in \Npawl{mrrvea ikran} \E{fifth banshee}.
\LNForum{27/1/2021}{https://forum.learnnavi.org/language-updates/cardinal-zam-vozam-and-zazam/}

\subsubsection{} All ordinals can combine freely with \N{nì-} to form
adverbs, \N{nì'awve} \E{first,} \N{nìmuve} \E{second,} etc.
\LNWiki{5/6/2013}{https://wiki.learnnavi.org/Canon/2013\%23Ordinals_.26_nume}


\section{Fractions}
\index{numbers!fractions}\index{fractions}

\subsection{-Pxì} Except for \E{half} and \E{third}, which have
separate lexical forms, fractions are formed by replacing the \N{-ve}
of an ordinal with \N{-pxì}.  Note the accent shift:
\index{-pxiì@\textbf{-pxì}}

\begin{center}
\begin{tabular}{rl}
half & \N{mawl} \\
third & \N{pan} \\
fourth & \N{tsì\ACC{pxì}} \\
fifth & \N{mrr\ACC{pxì}} \\
\end{tabular}
\hskip2em
\begin{tabular}{rl}
sixth & \N{pu\ACC{pxì}} \\
seventh & \N{ki\ACC{pxì}} \\
eighth & \N{vo\ACC{pxì}} \\
\\
\end{tabular}
\end{center}

\subsubsection{Word Class} Note that unlike the cardinals and
ordinals, the fraction words are nouns, not adjectives
(see \horenref{syn:partitive-gen} for syntax).

\subsection{Numerator} To make higher fractions, combine an
attributive cardinal with a fraction noun, \N{munea mrrpxì} \E{two
fifths}.

\subsubsection{Two Thirds} The fraction \E{two thirds} has a special
form, \N{mefan}, the dual of \N{pan}. \index{mefan@\textbf{mefan}}


\section{Other Forms}

\subsection{Alo} The word \N{\ACC{a}lo} \E{time, turn} combines with
numbers to form instance adverbs.  Four of these form compounds,
\N{\ACC{'aw}lo} \E{once}, \N{\ACC{me}lo} \E{twice}, \N{\ACC{pxe}lo}
\E{thrice, three times} and \N{\ACC{fra}lo} \E{each time, every time}.
All others combine as normal attributive adjectives, \Npawl{\uwave{alo
amrr} poan polawm} \E{he asked \uwave{five times}}. \index{melo@\textbf{melo}}
\index{'awlo@\textbf{'awlo}}\index{alo@\textbf{alo}}\index{fralo@\textbf{fralo}}
\index{pxelo@\textbf{pxelo}}

\subsection{-lie} The word \N{'aw\ACC{li}e} refers to a single event
in the past. \index{-lie@\textbf{-lie}}\index{'awlie@\textbf{'awlie}}

\subsection{Alien Digits} When quoting English digits, Na'vi will use
\N{'eyt} for \E{eight} and \N{nayn} for \E{nine}.  These are not used
for counting, but for things like phone numbers.
\index{'eyt@\textbf{'eyt}}\index{nayn@\textbf{nayn}}

\subsubsection{} \N{Kew} is \E{zero}.  \QUAESTIO{Current documentation
doesn't make clear if this idea is native or imported from the Humans.}
\index{kew@\textbf{kew}}
