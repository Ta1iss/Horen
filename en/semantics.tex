\nchapter{Semantics}

This section discusses a few matters which already have descriptions
spread out over the grammar and dictionary, but which it is useful to
describe all in one place to make certain patterns clear.

\section{The Copula and Predication} \index{lu@\textbf{lu}!semantics}
The main verb of noun and adjective predication is \N{lu} \E{be,}
along with verb \N{slu} \E{become}.

\begin{quotation}
\noindent\N{Oe lu ngeyn.} \E{I am tired.} \\
\noindent\N{Oe slu ngeyn.} \E{I become tired.} \\
\noindent\N{Oe layu taronyu.} \E{I will be a hunter.} \\
\noindent\N{Oe slayu taronyu.} \E{I will become a hunter.}
\end{quotation}

\subsection{Existence}
The verb \N{Lu} is also used to indicate existence, where English uses
\E{there is, there are,} as in \Npawl{äo fìutral lu tsmìm 'angtsìkä}
\E{there is a hammerhead track under this tree;} \Npawl{frauvanìri lu
yora'tu, lu snaytu} \E{in every game, there's a winner and a loser}.

\subsection{Possession}
Finally, \N{lu} is used with the dative to indicate possession, where
English uses \E{to have}, as in \Nfilm{lu oeru aylì'u frapor} \E{I have
something to say,} (lit., ``I have words for everyone'').

\subsection{Becoming}
Due to Na'vi's flexible word order, it could happen that the
relationships between two nouns with the verb \N{slu} would be
unclear.  In that situation, the adposition \N{ne} is used to clarify
the target, \N{taronyu slu ne tsamsiyu} \E{the hunter becomes a
warrior}. 

\subsection{Location} Na'vi has a separate verb, \N{tok}, for \E{be
at,} which is used instead of simple \N{lu} for location
predication, 

\begin{interlin}
  \glll Awngal tok kelkut. \\
  awnga-l tok kelku-t \\
  we.\I{incl}-\I{agt} be.at home-\I{pat} \\
  \trans{We're home.}\Ipawl{}
\end{interlin}

\noindent Notice that the verb \N{tok} is transitive.  It takes an
agentive subject (\N{awngal}) and patientive for the location
(\N{kelkut}).

\subsection{Colloquial Omission}
Both \N{lu} and \N{tok} can be omitted in colloquial speech
(\horenref{prag:colloq:omit}).  The normal case relations remain as
they were, such as \N{oe ngeyn} \E{I'm tired} which omits \N{lu},
and \N{oel fìtsengit} \E{I'm here} which omits \N{tok}.


\section{Perception} \index{perception}
Na'vi expressions of perception distinguish activity, sensation, and
ability.  Further, the verbs distinguish whether one is in control of
the perception (\E{look at}) or if one is not (\E{see}).

{\small
\begin{center}
  \begin{tabular}{l|ccccc}
    & \I{vtr} & \I{vtr} & \I{vin} & \I{n} & \I{n} \\
    & -control & +control & +control & sensation & ability \\
    \hline
sight & \N{tse'a} \E{see} & \N{nìn} \E{look at} & \N{tìng nari} \E{look} 
  & \N{'ur} \E{sight, look} & \N{tse'atswo} \E{sight, vision} \\

hearing & \N{stawm} \E{hear} & \N{yune} \E{listen to} & \N{tìng
  mikyun} \E{listen}
  & \N{pam} \E{sound} & \N{stawmtswo} \E{hearing}\\

smell & \N{hefi} \E{smell} & \N{syam} \E{smell} & \N{tìng ontu} \E{smell} 
  & \N{fahew} \E{smell} & \N{hefitswo} \E{sense of smell} \\

taste & \N{ewku} \E{taste} & \N{may'} \E{taste} & \N{tìng ftxì} \E{taste}
  & \N{sur} \E{taste, flavor} & \N{ewktswo} \E{sense of taste} \\

touch & \N{zìm} \E{feel} & \N{'ampi} \E{touch} & \N{tìng zekwä} 
  & \N{zir} \E{feel, texture} & \N{zìmtswo} \E{sense of touch}
  \end{tabular}
\end{center}
}

\noindent Take special note of the senses of smell and taste, where
English uses the verbs \E{smell} and \E{taste} for quite a range of
activities, Na'vi uses different words.  To happen to catch a scent of
something requires \N{hefi}, while smelling something on purpose will
use \N{syam}.

The compound expressions with \N{tìng} are mostly used when there is
no overt direct object, \N{tìng nari!} \E{look at that!}  But they can
be used with the dative for the thing perceived, \Npawl{poru tìng
nari!} \E{look at him!} though \Npawl{poti nìn!} \E{look at him!} will
be more common.
\NTeri{27/11/2012}{https://naviteri.org/2012/11/renu-ayinanfyaya-the-senses-paradigm/}

\subsection{Phenomena} \index{perception!phenomena} \index{fkan@\textbf{fkan}} \index{perception!phenomena!fkan@\textbf{fkan}}
To express the experience of phenomena the verb \N{fkan} is used with
the sensation nouns.  \N{Fkan} itself means something like
\E{resemble in a sensory modality, come to the senses as}.

\begin{interlin} \label{ex:percep2}
\glll Fìnaerìri sur fkan oeru kalin. \\
    fì-naer-ìri sur fkan oe-ru kalin \\
  this-drink-\I{top} flavor come.to.the.senses I-\I{dat} sweet \\
\trans{This drink tastes sweet to me.} \Ipawl{}
\end{interlin}

\noindent Note in ex.\ref{ex:percep2} that the experiencer of the
perception is in the dative.  This can be omitted for general
statements.  The sense noun may also be omitted when a sense
interpretation is obvious in the context.

\N{Fkan} is used with \N{na} \E{like} to make a comparison,

\begin{interlin}
\glll Raluri fahew fkan oeru na yerik. \\
    Ralu-ri fahew fkan oe-ru na yerik \\
    Ralu-\I{top} odor come.to.the.senses I-\I{dat} like hexapede \\
\trans{Ralu smells like a hexapede to me.} \Ipawl{}
\end{interlin}

\noindent Again, the dative experiencer and the sense noun can be
omitted if the meaning is clear from context.
