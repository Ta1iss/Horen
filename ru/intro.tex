\nchapter{Введение}

У нас все еще нет официального издания грамматики На'ви, которое было бы написано Полом Фроммером и одобрено владельцами коммерческих и интеллектуальных прав в Lightstorm Entertainment или 20th Century Fox. И врядли появится в обозримом будущем.\footnote{Первое упоминание июль 2010.  Без изменений на январь
2024.} В свете этого, я решил превратить краткое содержание грамматики в более развернутый труд.

Как и прошлое собрание, эта работа не научит вас На'ви напрямую.
Но взамен вы получите краткое и точное описание знаний о языке. Собраных за годы анализа с момента выхода первого фильма, а также на  основе сообщений Фроммера, которые уточняли языковые моменты.

Я во многом полагаюсь на страницы wiki Corpus и Canon, размещенные на LearnNavi.org,
без которых этот документ не вышел бы в свет. Сильно помогло появление личного блога Фроммера.\footnote{Конец июня 2010, \url{https://naviteri.org}}


\section{История расшифровки}
Новичкам в На'ви важно понять как мы стали обладателями сегодняшних знаний о языке.

Ранние догадки появились после первого интервью Фроммера, накануне выхода фильма в декабре 2009 года. В На'ви присутствовали взрывные согласные. Было тройственное разделение на падежи. Мы знали несколько фраз.

Большим шагом вперед появление списка навийских слов, размещенных одним из пользователей IMDB на своем сайте.\footnote{\url{https://kcbluesman.websitetoolbox.com/post?id=4013403}
requires login}  Его составили благодаря разбору \textit{Activist
Survival Guide}.\footnote{Wilhelm, Maria; Mathison, Dirk (2009). \textit{James
Cameron's Avatar: A Confidential Report on the Biological and Social
History of Pandora (An Activist Survival Guide),} It Books (HarperCollins).}
и разместили в блоге 11 декабря.\footnote{\url{http://www.suburbandestiny.com/?p=611}}Практически все словари берут от него начало.  У нас оказалось достаточно слов, чтобы начать вникать в предложения на На'ви в интервью Фроммера.

15го декабря, в интервью для UGO Movie
Blog\footnote{\url{https://web.archive.org/web/20100818193039/http://www.ugo.com/movies/paul-frommer-interview}}
мы познакомились с главным приветствием На'ви, \N{oel
ngati kameie} \E{Я Тебя Вижу}.  И первым появлением окончаний эргативного (agentive) и винительного (patientive) падежей.  Сравнив со словарем, мы выяснили что \N{-l} относилось к эргативному, а \N{-ti} к винительному.

Следующая крупная подвижка произошла спустя несколько дней после выхода поста от 19-го декабря в гостевом блоге Language Log.\footnote{\url{https://languagelog.ldc.upenn.edu/nll/?p=1977}}
Это все еще важный к прочтению материал для каждого ученика изучающего На'ви. Здесь мы многое почерпнули о звуках На'ви. А также 
достаточно узнали о грамматике На’ви, чтобы улучшить весь наш будущий анализ
примеров из интервью.

Даже сейчас, то многое что мы знаем, появилось не от прямых указаний Фроммера, например, ``это окончаниe притяжательного падежа (genitive),'' а благодаря представленным им примерам, где он говорил что здесь использован притяжательный падеж и люди сами разбирали языковые ситуации.  Некоторые из ранних наблюдений могли вводить в заблуждение. Особенно часто это бывало с падежами.  Во всех наших ранних примерах притяжательного падежа были окончания \N{-yä}. Лишь позднее мы узнали о том что существует окончание \N{-ä}.  В некоторых старых источниках до сих пор может быть указано только 
\N{-yä}.

Спустя месяцы Фроммер сам представил более подробные примеры на На'ви, каждый из которых досконально изучили, чтобы извлечь как можно больше информации о грамматике.  Фроммер в том числе ответил на несколько прямых вопросов о языке.  Таким путем часто 
подтверждаются наши догадки или исправляются вещи, кажущиеся изученными, или представляется новая информация.

Я постарался, насколько это возможно, чтобы все в этом
грамматическом сборнике имело подтверждение от самого Фроммера, либо было обосновано на достаточном количестве его собственных примеров на языке На'ви.  Тем не менее, этот документ точно ждут правки. Фроммер, будучи создателем, имеет исключительное право дополнять и исправлять На'ви в свете его собственного понимания того что нужно языку или какие пробелы в правилах необходимо заполнить по мере его развития.
Предположу что будущие фильмы франшизы \textit{Avatar} могут повлиять на язык На'ви самым неожиданным образом, и связано это будет не только с запросами Кэмерона к своему произведению, но и от неизбежных изменений от того как на искусственном языке будут общаться  актеры на съёмочной площадке. Но они, к сожалению или счастью, обленились и использовали английский во втором фильме.


\section{Условности и Обозначения}

Текст на языке На'ви выделяется жирным шрифтом, а его перевод курсивом, \N{fìfya} \E{так}.

Если пример на На'ви является прямой цитатой из интервью, письма с электронной почты или блога Пола Фроммера то он будет отмечен значком
$\mathcal{F}$ на полях, как тут \Npawl{kìyevame}. 
\textit{Hunt Song - Песнь Охоты} и \textit{Weaving Song - Песнь Ткацкого дела} из
\textit{Activist Survival Guide} отмечаются также.  Примеры из фильмов обозначены с помощью $\mathcal{A}$ и $\mathcal{A}^2$.

Эта работа использует диграфы \N{ts} и \N{ng} вместо, разработанной Фроммером, научной орфографии (\horenref{l-and-s:cg}).  Для большинства людей диграфы  привычнее.

В начальной документации для актеров Фроммер выделял ударения в словах с помощью подчеркивания нужного слога.  Наш сборник грамматики также следует этому правилу, как в \N{\ACC{tu}te} \E{личность} vs. \N{tu\ACC{te}}
\E{женщина}.  Чтобы избежать путаницы с условностями обозначения ударений Фроммера здесь используется волнистое \uwave{подчеркивание} чтобы обратить внимание на нужные части слов или фраз.

Примеры, предположительно неверные или содержащие ошибки  разного рода, выделяются стоящей впереди звездочкой, *\N{m'resh'tuyu}.  Префиксы выделены знаком тире на конце, как в \N{fì-}.  Префиксы, вызывающие леницию (смягчение), (\horenref{l-and-s:lenition}) используют знак плюс, как в \N{ay+}.
Суффиксы обозначеются знаком тире впереди, \N{-it}, а инфиксы маленькими скобками, \N{\INF{ol}}.  Звуковая транскрипция использует международный звуковой алфавит внутри квадратных скобок, [fɪ.ˈfja].

При цитировании одной из четырех песен, переведенных Фроммером для фильма, я буду использовать косую черту для разделения строк, \N{Rerol tengkrr kerä
/ Ìlä fya’o avol}.

Начиная с сентября 2011 года, к новым материалам добавляются ссылки на цитирование грамматических указаний.  Они появляются в конце разделы и выглядят так:
\NTeri{11/7/2010}{https://naviteri.org/2010/07/diminutives-conversational-expressions/}.
Обратите внимание, что даты соответствуют европейскому формату: день/месяц/год.
``NT'' для блога Фроммера, включая его ответы в комментариях,
``Wiki'' для LN.org Wiki, ``Forum'' для форума сайта learnnavi.org, и
``Ultxa'' для встречи в октябре 2010 года.  Здесь еще присутствуют пробелы в цитировании и я их заполню со временем.

Текст \QUAESTIO{в темно-бордовом цвете} для тем, которые, как мне кажется, поднимают серьезные языковые вопросы, не имеющие на сегодня ответа.  Некоторые требуют лишь подтверждения от Фроммера, а другие могут занять у него приличное время на размышления и проработку. В свое время эти части грамматики перестанут быть темно-бордовыми.


\subsection{Чтение Подстрочников}
Эта грамматика была впервые написана с мыслью, что любой, кто ее читает,
уже имел бы базовое владение На'ви. На сегодня (2024) это не настолько верное предположение, так что я начал добавлять более детальные примеры в форме подстрочного текста.  Так что есть возможность полностью расписать все происходящее в примере — грамматику, лексику, морфологию даже для тех кто далек от языка На'ви. Оформляется следующим образом:

\begin{interlin}
 \glll Oel ngati kameie. \\
     oe-l nga-ti kam‹ei›e \\
     \I{1sg-agt} \I{2sg-pat} видеть\INF{\I{pos.aff}} \\
 \trans{Я тебя вижу.}
\end{interlin}

\noindent В первой строке находится исходный текст на На'ви.  Во второй показаны префиксы, суффиксы и инфиксы.  Третья строка объясняет каждую часть с цифрами для местоимений (\I{1sg} = первое лицо
единственное число), я различные сокращения разных частей речи в грамматике На'ви.  Наконец, в последней строке находится общепринятый перевод. 

К сокращениям нужно немного привыкнуть, но примеры должны быть краткими, и в конечном итоге их будет легче читать.

\begin{multicols}{2}
\noindent\I{s}: субъект, начальная форма \\
\I{agt}: agent, эргативный падеж, форма \N{-(ì)l} \\
\I{pat}: patient, винительный падеж, форма \N{-(i)t} \\
\I{dat}: dative, дательный падеж, форма \N{-(u)r} \\
\I{gen}: genitive, притяжательный (родительный) падеж, форма \N{-(y)ä} \\
\I{top}: topical, тематический падеж, топик \N{-(ì)ri} form \\
\I{voc}: vocative, звательная форма \N{ma} \\
\I{lig}: adjective, прилагательное \N{-a-} ligature\\
\I{dim}: diminutive, уменьшительное (\N{-tsyìp}) \\
\I{rel}: relative, относительное местоимение \N{a} \\
\I{pfv}: perfective, совершённый вид \\
\I{ipfv}: imperfective, несовершённый вид \\
\I{pst}: past, прошлое \\
\I{rem.pst}: remote past, недавнее прошлое \\
\I{fut}: future, будущее \\
\I{rem.fut}: remote future, ближайшее будущее \\
\I{subj}: subjunctive, сослагательное наклонение \\
\I{pos.aff}: positive attitude, положительное отношение \\
\I{neg.aff}: negative attitude, негативное отношение \\
\I{cerem}: ceremonial, церемониальное, торжественное \\
\I{infer}: inferential, определение \\
\I{act.pcpl}: active participle, активное причастие \\
\I{pass.pcpl}: passive participle, пассивное причастие \\
\I{caus}: causative, побудительное действие \\
\I{refl}: reflexive, возвратное действие \\
\I{quot}: quotative, цитирование \N{san...sìk}
\end{multicols}

\noindent Some Na'vi grammar affixes are infixes — they go into the
middle of a word.  Since English does not allow that, in the
grammatical explanation line (line 3), the infixes are on the right or
left side of the word.  The example above has the positive attitude
infix, which I note to the right of the word \E{see}, with the small ‹
and › signs, instead of dashes used for prefixes and suffixes.

Converting old examples to use interlinears started in Summer 2022,
and will take a while to complete.


\section{Путь Воды}
Во втором фильме \textit{Аватар: Путь Воды} (2022) появляется новый диалект На'ви.  Но в действительности основным диалектом является Лесной На'ви (\N{Lì'fya Na'ringä}) из первого фильма по которому за десятилетие уже собрали материал.
В \textit{Пути Воды}  новый диалект это Рифовый На'ви (\N{Lì'fya
Wionä}).

Основные различия между двумя диалектами включают в себя всю
фонологию, морфологию, синтаксис и лексику.  Среди фонологических отличий присутствуют различные произношения взрывных согласных, отличное от принятного использование глоттальной смычки (дифтонга) между гласными, а также слияние гласных, происходящее в Лесном На'ви, отсутствует в Рифовом На'ви, что сохраняет \N{u} и \N{ù} как отдельные гласные фонемы.
One example of a lexical difference is that Reef Na'vi
favors \N{syawm} \E{know} where Forest Na'vi uses \N{omum}.

Because we have so much more Forest than Reef Na'vi, this grammar
presents the dialect notes largely in reference to Forest
Na'vi.\footnote{In no way should this be taken to mean that Forest
Na'vi is somehow a standard, or more correct, than any other dialect.}
Notes on Reef Na'vi are spread throughout the grammar, but I index
these notes thoroughly, to make it easier to find the Reef Na'vi
details.



\vfill
Thanks are due to LearnNavi.org members `Eylan Ayfalulukanä, Taronyu
and Ftiafpi for looking at early drafts of this grammar and making
suggestions.  I did not always follow their advice, so any flaws are
my own.

Thanks also to everyone who has commented and suggested corrections
since this grammar first appeared, as well as to all those who have
asked Paul questions over the years and posted his answers in public
places for everyone to learn from.

Finally, many thanks to Paul Frommer, who keeps answering grammar
and vocabulary ques\-tions when he can, for more than a decade.



\bigskip
