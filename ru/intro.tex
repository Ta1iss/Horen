\nchapter{Введение}

У нас, на сегодняшний день, всё ещё нет официального издания грамматики На'ви. Та\-ко\-го, которое было бы написано рукой Пола Фроммера и издано, с одобрения владельцев
ком\-мер\-чес\-ких и интеллектуальных прав из Lightstorm Entertainment
и 20th Century Fox. И не похоже, что оно появится в обозримом
будущем.\footnote{Написано в июле 2010-го. На 2024-й год без изменений.} В свете этого, я решил сделать из краткого содержания
грамматики более развернутую работу.

Как и предыдущее издание, эта работа не несет цели обучать языку На'ви напрямую. Однако, здесь в краткой и точной форме собраны наши обощённые знания о языке.
Их собирали в течение двух лет с момента выхода первого фильма, основываясь на тща\-тель\-ном разборе доступных источников и сообщений Фроммера, прояснявших языковые моменты.

Я во многом полагался на материалы Wiki Corpus и Canon,
размещенные на LearnNavi.\-org,
без которых эта работа не вышла бы в свет. Посильную помощь также оказал блог Пола Фроммера.\footnote{Конец июня 2010,
\url{https://naviteri.org}}


\section{История расшифровки}
Начинающим изучать язык На'ви было бы важным понять каким образом мы стали обладателями нынешних знаний о языке.

Наши начальные догадки появились после первого интервью Фроммера, которое он дал в декабре 2009 года накануне выхода Аватара 1. Язык На'ви
имел эйективные соглас\-ные и трипартит падежей, а так же мы узнали несколько фраз.

Большим шагом вперед стало появление списка навийских слов. Его разместил один из пользователей IMDB.\footnote{\url{https://kcbluesman.websitetoolbox.com/post?id=4013403}
requires login} Этот список собрали на основе \textit{Activist
Survival Guide}.\footnote{Wilhelm, Maria; Mathison, Dirk (2009).
\textit{James
Cameron's Avatar: A Confidential Report on the Biological and
Social
History of Pandora (An Activist Survival Guide),} It Books
(HarperCollins).}
и вы\-ло\-жи\-ли 11
декабря.\footnote{\url{http://www.suburbandestiny.com/?p=611}}Практически
все словари берут от него начало. У нас оказалось достаточно
слов, чтобы начать разбирать предложения на На'ви в интервью
Фроммера.

15 декабря, в интервью для UGO Movie
Blog\footnote{\url{https://web.archive.org/web/20100818193039/http://www.ugo.com/movies/paul-frommer-interview}}
мы познакомились с главным привет\-стви\-ем На'ви, \N{oel
ngati kameie} \E{Я Тебя Вижу}. И первым появлением окончаний
эргативного (agenti\-ve) и винительного (patientive) падежей.
Сравнив со словарем мы выяснили что окончание \N{-l} относилось к
эргативному, а \N{-ti} к винительному падежу.

Следующая крупный сдвиг произошел спустя несколько дней после
выхода поста от 19 декабря в гостевом блоге Language
Log.\footnote{\url{https://languagelog.ldc.upenn.edu/nll/?p=1977}}
Это все еще важный к прочтению материал для каждого новичика
в На'ви. Здесь мы многое узнали о фонетике. А
также грамматике языка На’ви, чтобы в будущем улучшить весь наш анализ
примеров, появлявшихся в интервью.

И по сей день, многое, из того что мы знаем, появилось не в результате прямых указа\-ний Фроммера, например, ``это окончаниe притяжательного
падежа (genitive),'' а благодаря пред\-став\-лен\-ным им примерам, где
он говорил что здесь использован притяжательный падеж и люди
сами занимались разбором языковых ситуаций. Некоторые из ранних на\-блю\-дений
могли вво\-дить в заблуждение. Особенно часто это бывало с
падежами. Во всех ранних примерах притяжа\-тель\-ного падежа
было лишь окончание \N{-yä}. Только позже нам стало известно что кроме него есть окончание \N{-ä}. Так что в некоторых старых источниках может быть до сих пор указано только окончание
\N{-yä}.

Спустя месяцы Фроммер сам дал более подробные примеры на языке
На'ви, каждый из которых был подробно изучен, чтобы выудить побольше информации о грамматике. Фроммер, кроме того,
ответил на несколько прямых вопросов, связанных с языком. Таким образом мы часто подтвержаем наши догадки или исправляем казавшиеяся
изученными вещи либо получаем какую-то новую информацию.

Я постарался, насколько это возможно, чтобы в этом
грамматическом сборнике все имело прямое подтверждение от Фроммера,
либо подтверждалось достаточным коли\-че\-ством примеров от него. Тем не менее, этот документ точно ждут
правки. Фроммер как создатель языка имеет исключительное право
дополнять и исправлять На'ви в за\-ви\-си\-мос\-ти от его собственного понимания
того, что нужно языку или какие пробелы в правилах требуется
заполнить по мере развития.
Предположу, что будущие фильмы франшизы \textit{Avatar} могут
повлиять на язык На'ви самым неожиданным образом, и связано это
будет не только с требованиями Кэмерона к своему фильму, но и от того как на искусственном языке будут
разговаривать актеры.


\section{Условности и обозначения}

Текст на языке На'ви будет выделяться жирным шрифтом, а перевод
курсивом, \N{fìfya} \E{так}.

Если пример на На'ви является прямой цитатой из интервью, сообщения с почты или блога Пола Фроммера, то он будет помечен
знаком
$\mathcal{F}$ на полях. Вот так \Npawl{kìyevame}. 
\textit{Hunt Song - Песнь Охоты} и \textit{Weaving Song - Песнь
Ткацкого дела} из
\textit{Activist Survival Guide} отмечены так же. Примеры из
фильмов обозначены с помощью $\mathcal{A}$ и $\mathcal{A}^2$.

В этой работе используются диграфы \N{ts} и \N{ng} вместо научной орфографии, разрабо\-тан\-ной Фроммером
(\horenref{l-and-s:cg}). Диграфы людям более привычны.

В начальных текстах для актеров Фроммер выделял в
словах ударения подчеркивания нужный слог. Наше обощение
грамматики делает так же \N{\ACC{tu}te}
\E{личность} vs. \N{tu\ACC{te}}
\E{женщина}. Чтобы не было путаницы с обозначением ударений, то, для указания важных частей слов или фраз, используется волнистое
\uwave{подчеркивание}.

Примеры, предположительно неверные или содержащие ошибки разного
рода, от\-ме\-че\-ны стоящей впереди звездочкой, *\N{m'resh'tuyu}.
Префиксы выделены знаком дефиса на конце \N{fì-}. Префиксы, которые вызывают леницию (смягчение), (\horenref{l-and-s:lenition})
используются со знаком плюс \N{ay+}.
Суффиксы обозначеются знаком дефиса перед ними, \N{-it}, а инфиксы -
фи\-гур\-ны\-ми скобками \N{\INF{ol}}. Звуковая транскрипция записывается внутри квадратных скобок и использует международный звуковой алфавит [fɪ.ˈfja].

При цитировании одной из четырех песен, переведенных Фроммером
для фильма, я буду использовать косую черту для разделения на
строки. \N{Rerol tengkrr kerä
/ Ìlä fya’o avol}.

Начиная с сентября 2011 года, к новым материалам добавляются
ссылки на источники. Они появляются в
конце раздела и выглядят так:
\NTeri{11/7/2010}{https://naviteri.org/2010/07/diminutives-conversational-expressions/}.
Даты соответствуют евро\-пей\-ско\-му формату:
день/месяц/год.
``NT'' для блога Фроммера, включая его ответы в ком\-мен\-та\-ри\-ях,
``Wiki'' для LN.org Wiki, ``Forum'' для форума сайта
learnnavi.org, и
``Ultxa'' для встречи в октябре 2010 года. Здесь еще
присутствуют пробелы в цитировании, но я за\-пол\-ню их со временем.

Текст \QUAESTIO{в темно-бордовом цвете} для тем, которые, как
мне кажется, поднимают серь\-ез\-ные языковые вопросы, пока не имеющие ответа. Некоторым нужно лишь подтверждение от
Фроммера, а другие могут занять у него приличное время для размышлений и про\-ра\-бот\-ки. Когда-нибудь они
перестанут быть темно-бордовыми.


\subsection{Подстрочные описания}
В начале подразумевалось то, что читающие данный сборник грамматики люди уже будут владеть На'ви на базовом уровне. Сейчас 2024 год и ясно, что это не так.
Поэтому я начал добавлять больше деталей в примеры с помощью
подстрочного текста. Он позволяет детально
описать всё, что есть в примере. Грамматику, лексику,
морфологию. Даже для тех, кто далек от языка На'ви. В следующем формате:

\begin{interlin}
 \glll Oel ngati kameie. \\
     oe-l nga-ti kam‹ei›e \\
     \I{1sg-agt} \I{2sg-pat} видеть\INF{\I{pos.aff}} \\
 \trans{Я тебя вижу.}
\end{interlin}

\noindent В первой строке идет текст На'ви. Во
второй раздельно отмечены префиксы, суффиксы и инфиксы. Третья строка описывает что чем является. С числами у местоимений (\I{1sg} =
первое лицо
единственное число) и сокращениями для разных частей речи в На'ви. А в последней строке дан перевод.

К сокращениям нужно привыкнуть, но потом примеры будет проще читать.

\begin{multicols}{2}
\noindent\I{s}: subject, форма без маркеров (номинатив) \\
\I{agt}: agent, эргативный падеж, \N{-(ì)l} \\
\I{pat}: patient, винительный падеж, \N{-(i)t} \\
\I{dat}: dative, дательный падеж, \N{-(u)r} \\
\I{gen}: genitive, притяжательный падеж, \N{-(y)ä} \\
\I{top}: topical, тематический падеж, \N{-(ì)ri} \\
\I{voc}: vocative, звательная форма \N{ma} \\
\I{lig}: adjective, лигатура прилагательного \N{-a-} \\
\I{dim}: diminutive, уменьшительность (\N{-tsyìp}) \\
\I{rel}: relative, маркер придаточного предложения \N{a} \\
\I{pfv}: perfective, совершенный вид \\
\I{ipfv}: imperfective, несовершенный вид \\
\I{pst}: past, прошлое \\
\I{rem.pst}: remote past, недавнее прошлое \\
\I{fut}: future, будущее \\
\I{rem.fut}: remote future, близкое будущее \\
\I{subj}: subjunctive, сослагательное наклонение \\
\I{pos.aff}: positive attitude, положительное отношение \\
\I{neg.aff}: negative attitude, отрицательное отношение \\
\I{cerem}: ceremonial, honorific, церем. или почтительность \\
\I{infer}: inferential, предположительность \\
\I{act.pcpl}: active participle, активное причастие \\
\I{pass.pcpl}: passive participle, пассивное причастие \\
\I{caus}: causative, принудельность (каузатив) \\
\I{refl}: reflexive, возвратность \\
\I{quot}: quotative, цитирование \N{san...sìk}
\end{multicols}

\noindent Некоторые аффиксы грамматики На'ви являются инфиксами
— их помещают внутрь слова. В грамматическом пояснении
(строка 3) инфиксы находятся в слове справа или слева. В примере
выше есть инфикс положительного отношения, который я выделил в
правой части слова \E{видеть}, значками ‹
и ›, вместо тире, которое используется с префиксами и
суффиксами.

Переработка старых примеров с добавлением подстрочных описаний
началась с лета 2022 года и займет много времени.


\section{Путь Воды}
Во втором фильме \textit{Аватар: Путь Воды} (2022) появился
новый диалект На'ви. Тем не менее, основным диалектом
все ещё остается Лесной На'ви (\N{Lì'fya Na'ringä}) из первого фильма. За десять лет по нему было собрано много информации.
Рифовый На'ви (\N{Lì'fya Wionä}) это новый диалект из фильма \textit{Путь Воды}.

У двух диалектов есть различия во всём, что связано с фонологией, морфологией, синтаксисом и лексикой. В фонологии по разному произносятся эйективные согласные, по другому ведёт себя гортанная смычка \N{'} (tìFtang) между гласными. В Рифовом На'ви нет слияния гласных звуков как в Лесном, поэтому \N{u}
и \N{ù} остаются отдельными гласными фонемами.
One example of a lexical difference is that Reef Na'vi
favors \N{syawm} \E{know} where Forest Na'vi uses \N{omum}.

Поскольку большая часть грамматики относится к Лесному На'ви, то заметки о Ри\-фо\-вом, в
основном,будут ссылаться на лесной диалект.\footnote{Однако, это не  значит то, что Лесной На'ви является
неким эталоном по сравнению с другими диалектами.}
Заметки о Рифовом На'ви в большом числе указаны в грамматике, но
я тщательно их пометил, чтобы легче было находить детали
относящиеся к рифовому диалекту.



\vfill
Выражаю благодарность участникам LearnNavi.org `Eylan Ayfalulukanä,
Taronyu
и Ftia\-fpi за разбор черновых версий грамматики и полезные
советы. Да, я не всегда следовал им, так что могут быть некоторые
изъяны.

Спасибо всем кто комментировал и предлагал улучшения с самого
выхода обощенной грамматики в свет, а также тем кто годами спрашивал 
Пола и делился его ответами, чтобы помочь изучающим язык людям.

И, конечно, огромное спасибо Полу Фроммеру, который, по мере
возможностей, про\-дол\-жа\-ет отвечать на грамматические и
словарные вопросы уже более десятилетия.



\bigskip
