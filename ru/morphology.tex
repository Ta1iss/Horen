\nchapter{Морфология}

\section{Существительное}

\subsection{Падежи} Окончания падежей На'ви могут меняться в завистимости от того, какая буква на конце слова - согласная, гласная или псевдогласная.\footnote{Названия падежей взяты Фроммером из терминологии Бернарда Комри для эргативных языков.  В большинстве лингвистическиз описаний фроммеровский ``subjective'' называют непереходным, ``agentive'' - эргативным, а ``patientive'' - винительным.}
\index{noun!declension}

\begin{center}
\begin{tabular}{llccc}
 & { }  & Гласная  & Согл. \& Псевдогл. & Дифтонг \\
\hline
Subjective & Номинативный & --- & --- & --- \\
Agentive & Эргативный & \N{-l} & \N{-ìl} & \N{-ìl} \\
Patientive & Винительный & \N{-t}, \N{-ti} & \N{-it}, \N{-ti} & \N{-ti}, \N{-it}, \N{-ay-t}, \N{-ey-t} \\
Dative & Дательный & \N{-r}, \N{-ru} & \N{-ur}, \N{-'-ru} & \N{-ru}, \N{-ur}, \N{-aw-r}, \N{-ew-r} \\
Genitive & Притяжательный & \N{-yä}, \N{-o-ä}, \N{-u-ä} & \N{-ä} & \N{-ä} \\
Topical & Тематический  & \N{-ri} & \N{-ìri} & \N{-ri}  \\
\end{tabular}\end{center}

\noindent\LNWiki{24/3/2010}{http://wiki.learnnavi.org/index.php/Canon/2010/March-June\%23Declension_with_Diphthongs_and_Deixis}

% For gen. with i: http://forum.learnnavi.org/language-updates/genitive-case-refinement-declension-of-tsaw/msg150927/#msg150927

\subsubsection{} Учтите, что после стоящих в конце слова псевдогласных \N{ll} и \N{rr} используются окончания как для согласных: \N{trr-ä}, \N{'ewll-it}.

\index{pseudovowel!declension}\label{morph:decl:pseudovowel}

\subsubsection{} После \N{o} и \N{u} притяжательный падеж выражается окончанием
\N{-ä}, а с другими гласными - с помощью \N{-yä}.  Как в \N{tsulfätuä}
от \N{tsulfätu}, однако будет \N{Na'viyä} от \N{Na'vi} и \N{lì'fyayä}
от \N{lì'fya}.

\subsubsection{} Nouns in \N{-ia} have the genitive in \N{-iä}, as in
\N{soaiä} from \N{soaia}.
\NTeri{25/5/2011}{http://naviteri.org/2011/05/some-miscellaneous-vocabulary/}
% This used to be covered by a single example in following "case
% refinement" section link; later generalized.

\subsubsection{} Вдобавок к некоторым местоимениям
(\horenref{morph:pron:irreg-gen}) есть несколько существительных с особой формой притяжательного падежа: \N{Omatikayaä} (от \N{Omatikaya}).  Это исключения.
\LNForum{19/3/2010}{https://forum.learnnavi.org/language-updates/genitive-case-refinement-declension-of-tsaw/msg150927/\#msg150927}

\subsubsection{} Из-за созвучности между \N{y} и
\N{i} происходит упрощение окончания винительного падежа \N{-it} при присоединении к дифтонгам с \N{y} на конце. Как в \N{keyeyt} \E{ошибки} вместо *\N{keyeyit}. Такая же созвучность между \N{w} и
\N{u} вызывает аналогичное упрощение окончания дательного падежа \N{-ur}. Как в \N{'etnawr} \E{ плечу}, но уже в дополнение к допустимому варианту \N{'etnawur}.
\NTeri{1/25/2013}{http://naviteri.org/2013/01/awvea-posti-zisita-amip-first-post-of-the-new-year/}

\subsubsection{} После смычки в дательном падеже ставится окончание \N{-ru}, как в \Npawl{lì'fyaolo\uwave{'ru}}.  Во всех прочих случаях после согласной идет \N{-ur}.
\NTeri{6/10/2012}{https://naviteri.org/2012/10/audio-and-video-learning-materials-for-navi-101/}
\LNForum{25/12/2020}{https://forum.learnnavi.org/language-updates/dative-ru/}

\subsubsection{} У дательного и винительного падежей разница между сокращёнными и полными окончаниями зависит только от стиля и благозвучия.

\paragraph{Рифовый На'ви} \label{morph:reef-navi:pat} \index{Reef Na'vi!patientive}
В рифовом далекте для винительного падежа предпочитается окончание \N{-ti}, вместо \N{-it} и \N{-t}. \Omaticon

\subsection{Заимствованные су\-ще\-стви\-тельные}\index{noun!declension!foreign}
При заимствовании человеческих слов про\-ис\-хо\-дит их фо\-но\-логи\-чес\-кая адап\-та\-ция в языке На'ви. Например, \N{Kerìsmìsì} от \E{Christmas} - \E{Рождество}.  Среди прочих изменений в конце слова добавлена гласная \N{-ì}, так как слова в На'ви не могут кончаться на \N{s}.
Поскольку в некоторых падежных окончаниях есть гласные, которые в противном случае делают недопустимые сочетания букв возможными, то по\-след\-няя \N{-ì} у таких заимствований будет опущена. Как в притяжательной форме \N{Kerìsmìsä}. Возь\-мём название города Кёльн:

\begin{center}
\begin{tabular}{llll}
Subjective & Номинативный & \N{Kelnì} & \\
Agentive   & Эргативный & \N{Kelnìl} & \N{Keln-ìl}, но не \N{Kelnì-l} \\
Patientive & Винительный & \N{Kelnit} & \\
Dative     & Дательный & \N{Kelnur} & \\
Genitive   & Притяжательный & \N{Kelnä} & \\
Topical    & Тематический & \N{Kelnìri} & \N{Keln-ìri}, но не \N{Kelnì-ri}
\end{tabular}
\end{center}

\noindent Учтите, что у винительного и дательного падежей допускаются только те окончания, ко\-то\-рые начинающиеся с гласной --- \N{-it} и \N{-ur}. \N{*Kelnti} - недопустимая форма слова в На'ви.
\NTeri{8/1/2022}{http://naviteri.org/2022/01/aawa-tipangkxotsyip-a-teri-horen-lifyaya-a-few-little-discussions-about-grammar/}

\subsection{Неопреде\-лен\-ный суф\-фикс -o} Когда существительное имеет суффикс неопре\-де\-лен\-нос\-ти \N{-o}, ``какой-то, некоторый,''  то падежные окончания идут после \N{-o}. Как в \N{puk-o-t}.
\index{-o@\textbf{-o}}\index{indefinite noun}
\LNWiki{14/3/2010}{http://wiki.learnnavi.org/index.php/Canon/2010/March-June\%23
A_Collection}
\NTeri{5/9/2011}{http://naviteri.org/2011/09/\%E2\%80\%9Cby-the-way-what-are-you-reading\%E2\%80\%9D/comment-page-1/\%23comment-1093}

\subsection{Число} Навийские существительные и местоимения могут находиться в ед\-ин\-ствен\-ном, двойственном, тройственном и множественном (четыре и более) числе.  Число указы\-ва\-ется префиксом. Все эти префиксы вызывают леницию.\index{dual}\index{trial}\index{plural}

\begin{center}
\begin{tabular}{lrl}
Двойственное & \N{me+} & \N{mefo} ($<$ \N{me+} $+$ \N{po}) \\
Тройственное & \N{pxe+} & \N{pxehilvan} ($<$ \N{pxe+} $+$ \N{kilvan}) \\
Множественное & \N{ay+} & \N{ayswizaw} \\
\end{tabular}
\end{center}

\subsubsection{} Префикс множественного числа можно не указывать \textit{только} в случае лениции.  \N{Prrnen} во множественном числе  \N{ayfrrnen}, а сокращенная форма без \N{ay+} это \N{frrnen} (вни\-ма\-ние на
\horenref{syn:adp:short-plural}).\footnote{Исключение: слово \N{'u} \E{вещь}
никогда не принимает сокращённую форму множественного числа и всегда пишется с префиксом
\N{ayu}.\index{'u@\textbf{'u}!no short plural}}
Префиксы двойственного и тройственного чисел указывают всегда.
\index{plural!short} \label{morph:short-plural}
\LanguageLog

\subsubsection{} Когда к слову, которое начинается с \N{e} или
\N{’e}, присоединяется префикс двой\-ствен\-но\-го или тройственного числа, то получившаяся последовательность гласных \N{*ee} сокращается. Таким образом \N{me+} $+$ \N{’eveng}
будет \N{meveng}.  Смотрите также \horenref{l-and-s:phonotactics:nsc}.

\subsubsection{} В Рифовом На'ви \N{pxe-} будет произноситься как \N{be-}. \index{Reef Na'vi!trial prefix}
\NTeri{13/1/2023}{http://naviteri.org/2023/01/2653/}

\section{Местоимение}

\subsection{Основные местоимения}
Местоимения получают те же падежные окончания и префиксы числа, что и существительные.

\begin{center}
\begin{tabular}{rllll}
Лицо      & Единств. & Двойств. & Тройств. & Множ. \\ 
\hline
1-е л. экскл.   & \N{\ACC{o}e}  & \N{m\ACC{o}e}  & \N{px\ACC{o}e}   & \N{ay\ACC{o}e} \\
1-е л. инкл.   & —      & \N{o\ACC{e}ng} & \N{px\ACC{o}eng} & \N{ayo\ACC{e}ng}, \N{aw\ACC{nga}} \\
2-е л.         & \N{nga} & \N{me\ACC{nga}} & \N{pxe\ACC{nga}} & \N{ay\ACC{nga}} \\
3-е л. одушев. & \N{po}  & \N{me\ACC{fo}} & \N{pxe\ACC{fo}}  & \N{ay\ACC{fo}, fo} \\
3-е л. неодушев.   & \N{\ACC{tsa}'u}, \N{tsaw} & \N{me\ACC{sa}'u} & \N{pxe\ACC{sa}'u} & \N{ay\ACC{sa}'u, sa'u} \\
Возвратное & \N{sno} & — & — & — \\
Неопределённое & \N{fko} & — & — & — \\
\end{tabular}
\end{center}

\subsubsection{} В повседневной речи, если корень первого лица \N{oe} не находится в конце слова, то его произношение меняется на
\N{we}, а так же \N{oel} будет произноситься как \N{wel} и \N{oeru} как \N{weru}.
Однако, такого произношения нет у двойственных и тройственных форм \N{moe} и \N{pxoe}, так как может вызвать недопустимое у начала слова сочетание согласных как в *\N{mwel}.  Правильное произношение отмечается подчеркиванием
\N{e}. \label{morph:pron:oe-we}

Если корень \N{oe} появляется после слова, оканчивающегося на \N{-u}, тогда \N{oe} будет про\-из\-но\-сит\-ся как \N{we}. Например, \N{'efu oe} звучит как \N{'efu we}.
\NTeri{30/4/2021}{http://naviteri.org/2021/04/mipa-ayliu-mipa-sioeykting-new-words-new-explanations/}

\subsubsection{} Местоимения первого лица неединственного числа являются как эксклюзивными (исключают собеседника), так и инклюзивными (включают последнего). Инклюзивное окончание \N{-ng} образовано от \N{nga}, которое снова получит полную форму, если добавить падежное окончание.
Поэтому \N{oeng} в эргативном падеже будет \N{oengal}, а не \N{*oengìl}.

\subsubsection{} \N{Ayoeng} имеет сокращённую форму \N{aw\uline{nga}}.
Падежные окончания свободно ис\-поль\-зу\-ют\-ся с обоими. Тем не менее, форма \N{awnga} более распространена.\index{ayoeng@\textbf{ayoeng}}\index{awnga@\textbf{awnga}}

\subsubsection{} У третьего лица есть отдельные местоименя для одушевлённых и неоду\-шев\-лён\-ных предметов. Одушевлённое местоимение \N{po} можно применять к животным, но нельзя к насекомым. Чем более значимым является отношение говорящего к животному, тем веротянее использование формы \N{po}.\index{animacy}
\LNForum{25/2/2017}{https://forum.learnnavi.org/language-updates/about-fmetok-and-pofo/}

\subsubsection{} Одушевлённое местоимение третьего лица \N{po} не различается по полу. Поэтому одинаково подходит для ``он'' и ``она''.  Не смотря на то что существуют имеющие пол формы \N{po\ACC{an}} \E{он} и \N{po\ACC{e}}
\E{она}, но они постоянно отклоняются и не имеют множественного числа.  Об их использовании тут  \horenref{syn:pron:gender}.
\label{morph:pron:gender}
\NTeri{17/10/2017}{http://naviteri.org/2017/09/zisikrr-amip-ayliu-amip-new-words-for-the-new-season/\#comment-27696}

\subsubsection{} \label{morph:pron:irreg-gen}
У нескольких местоимений есть нестандартная форма притяжательного падежа с заменой гласной,

\begin{center}
\begin{tabular}{cc}
Номинативный & Притяжательный \\
\hline
\N{fko} & \N{fkeyä} \\
\N{nga} & \N{ngeyä} \\
\N{po} & \N{peyä} \\
\N{sno} & \N{sneyä} \\
\N{tsa'u} & \N{tseyä} \\
\N{ayla} & \N{ayleyä}
\end{tabular}
\end{center}

\noindent Такая замена гласной происходит во всех формах множественного числа \N{feyä} $<$
\N{fo} и инклюзивах первого лица \N{awngeyä} $<$ \N{awnga}.
\index{nga@\textbf{nga}!genitive \textbf{ngeyä}}
\index{fko@\textbf{fko}!genitive \textbf{fkeyä}}
\index{po@\textbf{po}!genitive \textbf{peyä}}
\index{sno@\textbf{sno}!genitive \textbf{sneyä}}
\index{awnga@\textbf{awnga}!genitive \textbf{awngeyä}}

В местоимениях, образованных от \N{po} (\N{'awpo} \E{один
человек, некто}, \N{frapo} \E{каждый человек}, \N{lapo} \E{другой человек}, \N{fìpo} \E{этот человек}, \N{tsapo} \E{тот человек}), таким же образом меняется гласная и получается \N{frapeyä}, а не *\N{frapoä}.
\LNForum{6/7/2023}{https://forum.learnnavi.org/language-updates/frapo-and-other-po-genitives/msg688200/}

\subsubsection{} В неформальной и отрывистой военной речи финальный
\N{ä} в притяжательной форме местоимений может отбрасываться, \N{ngey
'upxaret}.\label{morph:pron:gen-clipped} \index{genitive!shortened
in pronouns}\index{pronouns!short genitive}

\subsubsection{} Третье лицо для неодушевленных предметов \N{tsa'u} всего лишь является ука\-за\-тель\-ным местоимением ``тот'', которое в притяжательном падеже имеет форму \N{tseyä}.
В быстрой и неформальной речи может принимать форму \N{tsaw}. Она может использоваться с пред\-ло\-га\-ми (\N{tsawfa}), но к ней нельзя присоединять падежные окончания (таких как \N{*tsawl} нет).  Тем не менее, падежные окончания может получать основа \N{tsa-}  (\N{tsal, tsar} и т.д.), а так же использоваться с предлогами (\N{tsafa}), но, опять же, в быстрой речи.
\label{morph:pron:tsa}\index{tsaw@\textbf{tsaw}}\index{tsa'u@\textbf{tsa'u}}
\LNWiki{6/5/2010}{https://wiki.learnnavi.org/Canon/2010/March-June\%23History_of_Tsaw}
\NTeri{3/8/2011}{http://naviteri.org/2011/08/new-vocabulary-clothing/comment-page-1/\#comment-917}

\subsubsection{} Возвратное местоимение \N{sno} не изменяется для указания числа. \index{sno@\textbf{sno}!not marked for number}

\subsubsection{} Непределённое местоимение третьего лица для одушевлённых предметов это
\N{fko} (притяжат.\ \N{fkeyä}).
\LNWiki{17/5/2013}{https://wiki.learnnavi.org/Canon/2013\%23Double_Dative_and_more}

\subsection{Церемони\-аль\-ные/по\-чти\-тель\-ные место\-имения}

\begin{center}
\begin{tabular}{rllll}
      & Единств. & Двойств. & Тройств. & Множ. \\ 
\hline
1-е л. экскл. & \N{\ACC{o}he}  & \N{\ACC{mo}he}  & \N{\ACC{pxo}he}   & \N{ay\ACC{o}he} \\
1-е л. инкл. & -         & \N{\ACC{o}heng} & \N{\ACC{pxo}heng} & \N{a\ACC{yo}heng} \\
2-е л.         & \N{nge\ACC{nga}} & \N{menge\ACC{nga}} & \N{pxenge\ACC{nga}} & \N{aynge\ACC{nga}} \\
3-е л. одушев.   & \N{\ACC{po}ho} \\
3-е л. жен.      & \N{po\ACC{he}} \\
3-е л. муж.    & \N{po\ACC{han}} 
\end{tabular}
\end{center}\index{pronouns!honorific}\label{morph:hon-pron}

\QUAESTIO{Притяжательный падеж для \N{poho} не ясен.}

\NTeri{28/2/2022}{http://naviteri.org/2022/02/lifyengteri-concerning-honorific-language/}

\subsubsection{} Для инклюзивов первого лица стандартом является формы \N{oheng}.  Однако, ис\-поль\-зо\-ва\-ние отдельных местоимений \N{ohe
ngengasì} с присоединяемым союзом \N{sì} \E{и} тоже допус\-ка\-ет\-ся.
\LNForum{25/10/2022}{https://forum.learnnavi.org/language-updates/a-collection-of-questions-answered/}

\subsection{Lahe}\index{lahe@\textbf{lahe}!declension}\label{morph:lahe:short}
Когда \N{lahe} \E{другой} используется в качестве местоимения, то имеет следующие сокращённые формы множественного числа:

\begin{center}
\begin{tabular}{lll}
       & Полная & Сокращённая \\
Номинатив & \N{ay\ACC{la}he}     & \N{ay\ACC{la}} \\
Эргативный   & \N{ay\ACC{la}hel}    & \N{ay\ACC{lal}} \\
Винительный & \N{ay\ACC{la}het(i)} & \N{ay\ACC{la}t(i)} \\
Дательный     & \N{ay\ACC{la}her(u)} & \N{ay\ACC{la}r(u)} \\
Притяжательный   & \N{ay\ACC{la}heyä}   & \N{ay\ACC{le}yä} \\
Тематический    & \N{ay\ACC{la}heri}   & \N{ay\ACC{la}ri}
\end{tabular}
\end{center}

\noindent Заметьте, что у сокращённой формы (\N{ayla}) в притяжательном падаже происходит замена гласной, которую мы уже видели в других местоимениях
(\horenref{morph:pron:irreg-gen}).
\NTeri{5/5/2023}{http://naviteri.org/2023/05/reef-navi-part-2-morphology-syntax-lexicon-and-more/}
% Stress accent verified (out of paranoia) by email, 2023may07.

\section{Предсуществительные}

\noindent Так называемые ``предсуществительные'', в оригинале ``prenouns'', это префиксы, которые присоединяются к существительным, наподобие того что есть у прилагательных. \index{prenoun}

\subsection{Fì-} Это префикс для близкого указания, \E{этот.}  Если за ним идет префикс мно\-же\-ствен\-но\-го числа \N{ay+}, то в обычной речи их сокращают до \N{fay+}, \E{эти}.  Однако, \N{fìay+} может ис\-поль\-зо\-вать\-ся для точной или формальной речи, \Npawl{oel foru fìaylì'ut tolìng
a krr, kxawm oe harmahängaw}. \label{morph:prenoun:fi}
\index{fiì-@\textbf{fì-}} \index{fay+@\textbf{fay+}}
\LNForum{27/7/2013}{https://forum.learnnavi.org/language-updates/navi-details-from-avatarmeet-2013/}

\subsubsection{} В паре с \N{fì-} некоторые существительные и прилагательные образуют наречия как \N{fìtrr} \E{сегодня} и \N{fìtxan} \E{настолько}.

\subsection{Tsa-} Это префикс дальнего указания, \E{тот.}  Если следующим идет префикс \N{ay+}, то они сокращаются в \N{tsay+}
\E{те}. 
\index{tsa-@\textbf{tsa-}} \index{tsay+@\textbf{tsay+}}

\subsection{-Pe+} \label{morph:pre:pe} Этот вопросительный префикс означает
\E{какой, который} как в \N{pelì'u} \E{какое слово?}  Необычно то, что он может быть как префиксом (\N{pelì'u}), так и суффиксом (\N{lì'upe}).
Как префикс вызывает леницию в стоящем за ним слове. Если после \N{pe+}идет префикс множественного числа \N{ay+}, то они сокращаются в \N{pay+}.

\subsection{Fra-} Означает \E{все, каждый.}  Если после него стоит префикс множественного числа \N{ay+}, то их сокращают во \N{fray+}.
\index{fra-@\textbf{fra-}} \index{fray+@\textbf{fray+}}
\LNForum{27/7/2013}{https://forum.learnnavi.org/language-updates/navi-details-from-avatarmeet-2013/}

\subsection{Fne-} Означает \E{тип (чего-то), вид (чего-то)}.
\index{fne-@\textbf{fne-}}

\subsubsection{} Данный префикс связан с существительным \N{fnel}, которое так же значит \E{тип, вид.}  \N{Fnel} может появляться в паре с находящимся в притяжательном падеже существительным. Эта связка будет иметь такой же смысл как у префикса.  \N{Tsafnel syulangä} и
\N{tsafnesyulang} оба означают \E{тот тип цветов}.
\index{fnel@\textbf{fnel}}

\subsection{Сокращение гласной} Одинаковые гласные в конце префикса и начале слова сокращаются как в \N{tsatan}
\E{тот свет} от \N{tsa-atan} (\horenref{l-and-s:phonotactics:precontract}).
\index{prenoun!contraction}\label{morph:prenoun:contraction}

\subsection{Комбинации} Порядок расположения префиксов в рамках одного слова --- \index{prenoun!combinations}

\begin{center}
\begin{tabular}{cccccc}
\N{fì-} \\
\N{tsa-} & \N{fra-} & указание числа & \N{fne-} & существительное & \N{-pe} \\
\N{pe+}
\end{tabular}
\end{center}

\noindent  В каждом стобце доступен только однин вариант, а вопросительный префикс/суффикс применяется один раз.  \QUAESTIO{Пока не все детали этой очередности подтверждены для \N{fra-}.}

\subsubsection{} Указательные префиксы никогда не используются с сокращённой формой мно\-же\-ствен\-ного числа (\horenref{morph:short-plural}); \N{tsaytele} \E{те дела} и никогда
*\N{tsatele} (в ед.ч. \N{txele}). \index{plural!short!not used with prenouns}


\section{Коррелятивные слова}

\noindent Указательные местоимения и некоторые простые наречия времени, образа действия и места являются лишь существительными в паре с ``предсуществительными'' (ориг. prenouns). Хотя система не является полностью цельной.

% This is what I get for attaching several words to the same footnote.
\addtocounter{footnote}{1}
\newcounter{coraccent}\setcounter{coraccent}{\value{footnote}}

\begin{center}
\begin{tabular}{rllllll}% name person thing time place manner
 & Личность & Вещь & Действие & Время & Место & Образ \\
\hline
\multirow{2}{*}{это} & \N{\ACC{fì}po} & \N{fì\ACC{'u}} &
  \N{fì\ACC{kem}} & \N{set} & \N{fì\ACC{tseng}(e)} & \N{fì\ACC{fya}}  \\ 
 & \E{этот} & \E{эта (вещь)} & \E{это (действие)} & \E{сейчас} &
  \E{здесь} & \E{таким образом} \\
\multirow{2}{*}{то} & \N{\ACC{tsa}tu} & \N{\ACC{tsa}'u} & \N{tsakem}\footnotemark[\value{coraccent}] & \N{tsa\ACC{krr}} &
   \N{tsatseng}\footnotemark[\value{coraccent}] & \N{\ACC{tsa}fya} \\
 & \E{тот} & \E{та (вещь)} & \E{то (действие)} & \E{тогда} &
  \E{там} & \E{тем образом} \\
\multirow{2}{*}{всё} & \N{\ACC{fra}po} & \N{\ACC{fra}'u} & --- &
  \N{\ACC{fra}krr} & \N{\ACC{fra}tseng} & \N{\ACC{fra}fya}  \\
 & \E{каждый} & \E{каждая (вещь)} &  & \E{всегда} & \E{везде} &
  \E{всячески} \\
\multirow{2}{*}{нет} & \N{\ACC{kaw}tu} & \N{\ACC{ke}'u} & \N{\ACC{ke}kem} &
  \N{\ACC{kaw}krr} & \N{\ACC{kaw}tseng} & --- \\
 & \E{никто} & \E{ничто} & \E{никакое действие} & \E{никогда} & \E{нигде} \\
\end{tabular}
\end{center}\label{morph:correlatives}
\footnotetext[\value{coraccent}]{Ударение может падать на любой слог.}
\LNWiki{18/5/2011}{http://wiki.learnnavi.org/index.php/Canon/2011/April-December\%23Kawtseng.2C_tsapo_and_prefixes}
\NTeri{24/7/2011}{http://naviteri.org/2011/07/txantsana-ultxa-mi-siatll-great-meeting-in-seattle/comment-page-1/\%23comment-845} % kekem

\subsubsection{} \QUAESTIO{Множественное число у этих слов немного причудливое.  Хотя \N{tsa'u} происходит от \N{tsa-} и \N{'u}, во множественном числе это \N{(ay)sa'u}.
Подтверждено, но уточнения были бы к месту.  Как бы это работало в \N{tsapo}?}

\subsubsection{} Формы \N{tsa'u} здесь \horenref{morph:pron:tsa}.

\subsection{Вопросы} Как и в случае с существительными, вопросительный аффикс \N{-pe+} может быть префиксом с леницией или суффиксом.

\begin{center}
\begin{tabular}{rl}
кто? & \N{pe\ACC{su}}, \N{\ACC{tu}pe} \\
какая (вещь)? & \N{pe\ACC{u}}, \N{\ACC{'u}pe} \\
какое (действие)? & \N{pe\ACC{hem}} \N{\ACC{kem}pe} \\
когда? & \N{pe\ACC{hrr}}, \N{\ACC{krr}pe} \\
\end{tabular}
\hskip2em
\begin{tabular}{rll}
где? & \N{pe\ACC{seng}}, \N{\ACC{tseng}pe} \\
как? & \N{pe\ACC{fya}}, \N{\ACC{fya}pe} \\
по какой причине (почему)? & \N{pe\ACC{lun}}, \N{\ACC{lum}pe} \\
какой тип/вид (чего-то)? & \N{pe\ACC{fnel}}, \N{\ACC{fne}pe}\\
\end{tabular}
\end{center}

\subsubsection{} Вопросительное слово для людей \N{tupe} / \N{pesu}
\E{кто} имеет внушительную кол\-лек\-цию, которая представляет формы полов и неединственных чисел:

\begin{center}
\begin{tabular}{lccc}
 & Обычный & Мужской & Женский \\
\hline
Единственное & \N{pe\ACC{su}}, \N{\ACC{tu}pe} & 
           \N{pe\ACC{stan}}, \N{tu\ACC{tam}pe} &
           \N{pe\ACC{ste}}, \N{tu\ACC{te}pe} \\
Двойственное     & \N{pem\ACC{su}}, \N{me\ACC{su}pe} & 
           \N{pem\ACC{stan}}, \N{me\ACC{stam}pe} &
           \N{pem\ACC{ste}}, \N{me\ACC{ste}pe} \\
Тройственное    & \N{pep\ACC{su}}, \N{pxe\ACC{su}pe} & 
           \N{pep\ACC{stan}}, \N{pxe\ACC{stam}pe} &
           \N{pep\ACC{ste}}, \N{pxe\ACC{ste}pe} \\
Множественное   & \N{pay\ACC{su}}, \N{(ay)\ACC{su}pe} & 
           \N{pay\ACC{stan}}, \N{(ay)\ACC{stam}pe} &
           \N{pay\ACC{ste}}, \N{(ay)\ACC{ste}pe} \\
\end{tabular}
\end{center}

\noindent Их форма притяжательного падежа схожа с той что у существительных \N{pesuä}, а не как расстмотренная ранее замена гласной у местоимения \N{po} в\ \N{peyä}.

Формы неединственного числа для \N{pehem} / \N{kempe} следуют тому же шаблону:

\begin{center}
\begin{tabular}{lc}
Единственное & \N{pe\ACC{hem}}, \N{\ACC{kem}pe} \\
Двойственное & \N{pem\ACC{hem}}, \N{me\ACC{hem}pe} \\
Тройственное & \N{pep\ACC{hem}}, \N{pxe\ACC{hem}pe} \\
Множественное & \N{pay\ACC{hem}}, \N{(ay)\ACC{hem}pe} \\
\end{tabular}
\end{center}

\noindent\NTeri{30/7/2011}{http://naviteri.org/2011/07/number-in-na’vi/}
\LNForum{19/12/2021}{https://forum.learnnavi.org/language-updates/confirmation-on-genitive-form-of-pesu/}

\subsection{Fì'u и Tsa'u в номинализации предложений} \label{morph:fwa-tsawa}
Номинализация предложения - процесс при котором предложение целиком может быть выражено существительными, входящими в состав другого предложения.
Указательное местоимение \N{fì'u} и местоимение для неодушевлённых предметов \N{tsa'u}
используются с аттрибутивной частицей \N{a} для но\-ми\-на\-ли\-за\-ции предложений
(\horenref{syn:clause-nom}). Когда определительная частица следует за опре\-де\-лен\-ны\-ми па\-деж\-ны\-ми формами местоимений, то их пары сокращаются и пишутся слитно:
% http://forum.learnnavi.org/language-updates/txelanit-hivawl/

\begin{center}
\begin{tabular}{rcc}
Падеж & Сокращение \N{Fì'u} & Сокращение \N{Tsaw} \\
\hline
Номинативный & \N{fwa} ($<$ \N{fì'u a}) & \N{\ACC{tsa}wa} \\
Эргативный & \N{\ACC{fu}la} ($<$ \N{fì'ul a}) & \N{\ACC{tsa}la} \\
Винительный & \N{\ACC{fu}ta} ($<$ \N{fì'ut a}) & \N{\ACC{tsa}ta} \\
Дательный & \N{\ACC{fu}ra} ($<$ \N{fì'ur a}) & \N{\ACC{tsara}} \\
Тематический & \N{\ACC{fu}ria} ($<$ \N{fì'uri a}) & \N{\ACC{tsa}ria} \\
\end{tabular}
\end{center}
\index{fwa@\textbf{fwa}}\index{tsawa@\textbf{tsawa}}
\index{fula@\textbf{fula}}
\index{futa@\textbf{futa}}\index{tsata@\textbf{tsata}}
\index{furia@\textbf{furia}}\index{tsaria@\textbf{tsaria}}
\LNWiki{18/6/2010}{http://wiki.learnnavi.org/index.php/Canon/2010/March-June\%23The_contrast_between_fwa.2Ftsawa.2C_furia.2Ftsaria}
\LNForum{21/4/2020}{https://forum.learnnavi.org/language-updates/tsa-words-the-complete-set/}

В быстрой речи \N{futa} может произноситься как \N{fta},
а \N{tsata} как \N{tsta}. 
\LNForum{25/2/2022}{https://forum.learnnavi.org/language-updates/contrasting-nominalized-phrases-futa-vs-tsata/}

\subsection{Прочие существительные в номинализации предложений} В то время как \N{fì'u}
и \N{tsaw} могут номинализировать практически любые предложения, то с глаголами, выражающими действие слуха, речи или вопроса, предпочитается использовать су\-щес\-тви\-тель\-ные \N{fmawn} \E{новость}, \N{tì'eyng}
\E{ответ} and \N{faylì'u} \E{эти слова;} а с повелительными глаголами \N{tson} \E{долг, обязательство, требование}. Однако, коротких форм у них меньше: \label{morph:fmawn}

\begin{center}
\begin{tabular}{rc}
Падеж & Сокращение \\
\hline
Subjective & \N{teynga} ($<$ \N{tì'eyng a}) \\
Agentive & \N{teyngla} ($<$ \N{tì'eyngìl a}) \\
Patientive & \N{teyngta} ($<$ \N{tì'eyngit a})
\end{tabular}
\end{center}
\index{fmawnta@\textbf{fmawnta}} \index{teynga@\textbf{teynga}}
\index{teyngla@\textbf{teyngla}} \index{teyngta@\textbf{teyngta}}
\index{fmawn@\textbf{fmawn}} \index{tiì'eyng@\textbf{tì'eyng}}
\index{fayluta@\textbf{fayluta}} \index{fayliì'u@\textbf{faylì'u}}
\index{tsonta@\textbf{tsonta}}

\noindent  У \N{fmawn}
и \N{faylì'u} только с винительным падежом есть краткая форма, \N{fmawnta} ($<$ \N{fmawnit a}) и
\N{fayluta} ($<$ \N{faylì'ut a}), так же \N{tsonta} ($<$ \N{tsonit
a}).  See \horenref{syn:quot:nominalized} for the syntax.
\NTeri{31/8/2011}{http://naviteri.org/2011/08/reported-speech-reported-questions/}
\NTeri{2/10/2014}{http://naviteri.org/2014/10/tson-si-fpomron-obligation-and-mental-health/}


\section{Прилагательное}
\subsection{Указание} Аффикс \N{-a-} используется для связи прилагательного с определяемым существительным. Он присоединяется к прилагательному с ближайшей стороны от оп\-ре\-де\-ля\-ем\-ого существительного. Как в \N{yerik awin} или \N{wina yerik} означающими
``быстрый шес\-ти\-ног.''\label{morph:adj-attr}
\index{-a-@\textbf{-a-}\index{adjective!attributive!formation}}

\subsubsection{} Прилагательные образованные с помощью \N{le-} обычно отбрасывают маркер \N{a-} со стороны \N{le-}, поэтому могут быть \N{ayftxozä lefpom} или реже \N{ayftxozä alefpom}.  Тем не менее, если
 \N{le-}прилагательное стоит идет существительным, то маркер \N{-a-} обязателен, \N{lefpoma ayftxozä}.

\begin{center}
\begin{tabular}{ll}
\N{ayftxozä lefpom} & обычно \\
\N{ayftxozä \uwave{a}lefpom} &  допустимо \\
$*$\N{lefpom ayftxozä} &  неправильно \\
\N{lefpom\uwave{a} ayftxozä} &  правильно \\
\end{tabular}
\end{center}


\section{Глагол}
\subsection{Положения инфиксов} Фроммер приводит три позиции инфиксов глагола: пред\-пер\-вую, первую и вторую. В каждой могут находиться только инфиксы определённого типа (опи\-са\-но ниже).

\subsubsection{} У глагола все инфиксы появляются только в последнем (ultima) и предпоследнем (penult) слоге корня. Они помещаются в такой слог перед гласной, псевдогласной или дифтонгом. Как в
\N{kä} $>$ \N{k‹ìm›ä} и \N{taron} $>$ \N{t‹ol›ar‹ei›on}.

\subsubsection{} Если в слоге нет начальной согласной(-ных), то инфикс всё равно ставится перед гласной. Как в \N{omum} $>$ \N{‹iv›omum} и \N{ftia}
$>$ \N{fti‹ats›a}.

\subsubsection{} Ударение остается на той же гласной, которая имела его до добавляения каких-либо инфиксов \N{\ACC{haw}nu} $>$
\N{h‹il\ACC{v›aw}nu}.\footnote{Исключение: глагол \N{o\ACC{mum}}
меняет ударение на \N{o} во всех склоняемых и производных формах
\N{i\ACC{vo}mum}, \N{nìaw\ACC{no}num}. \index{omum@\textbf{omum}!accenting}
Глагол \N{i\ACC{nan}} следует тому же принципу - \N{o\ACC{li}nan}.
\index{inan@\textbf{inan}!accenting}
}

\subsubsection{} У составных слов, как правило, инфиксы получает только одна часть.  Например, \N{yom-tìng} \E{кормить} является комбинацией
\N{yom} \E{есть} и \N{tìng} \E{давать}.  Совершенный вид этого слова будет не
*\N{y\INF{ol}omtìng}, а \N{yomt\INF{ol}ìng}.  Обычно, часть составных слов, по\-лу\-чаю\-щая инфиксы, находится в конце. Однако, в лексиконе есть несколько слов, где она в начале. Такие составные слова просто нужно запомнить.  \index{verb!compound!infix location}

\subsubsection{} Несколько составных слов модели глагол+глагол могут в обоих частях получать инфиксы. Как в \N{kan'ìn} \E{интересоваться, специализироваться},
составленном из \N{kan} \E{на\-ме\-ре\-вать\-ся, целиться, направлять (оружие или что-либо) на цель} и \N{'ìn} \E{иметь занятие}.
\Ultxa{2/10/2010}{http://wiki.learnnavi.org/index.php/Canon/2010/UltxaAyharyu\%C3\%A4\%23Transitivity_and_Infix_Positions}

\subsection{Предпервая позиция} Инфиксы предпервой позиции меняют переходность гла\-го\-лов. Ставятся внутрь предпоследнего слога перед гласной. Либо перед самим глаголом, если он состоит из одного слога.
\label{morph:pre-first}
\index{reflexive!formation}\index{causative!formation}

\begin{center}
\begin{tabular}{lr}
Принудительный (Causative) & \N{\INF{eyk}} \\
Возвратный (Reflexive) & \N{\INF{äp}} \\
\end{tabular}
\end{center}

\noindent\LNWiki{1/2/2010}{http://wiki.learnnavi.org/index.php/Canon\%23Reflexives_and_Naming} % reflexive
\LNWiki{15/2/2010}{http://wiki.learnnavi.org/index.php/Canon\%23Trials_.26_transitivity} % causative

\subsubsection{}
\index{si construction@\textbf{si} construction!reflexive!pronunciation}
В обычном разговоре возвратный \N{si}-глагол совершенного вида \N{säpo\ACC{li}} часто про\-из\-но\-си\-тся как
\N{spo\ACC{li}}, а \N{säpi\ACC{vi}} как \N{spi\ACC{vi}}.
\NTeri{30/9/2010}{http://naviteri.org/2010/09/quick-follow-up/}
\NTeri{3/8/2011}{http://naviteri.org/2011/08/new-vocabulary-clothing/}


\subsubsection{} \index{reflexive!of a causative}
Возвратно-принудительный инфикс ``заставлять себя'' образован от \N{\INF{äp}\INF{eyk}}, следовательно \N{po täpeykerkup} \E{он заставляет себя умереть}.  Переходностью начального выражения устанавливает то, каким образом будут использоваться падежи с глаголом, имеющим \N{\INF{äp}\INF{eyk}}.
Про их использование тут \horenref{reflexive-of-causative}.

\subsection{Первая позиция} Инфиксы первой позиции задают время, вид и настроение, а так же образуют причастия.  Ставятся перед гласной предпоследнего слога глагола, либо перед глаголом, если он состоит из одного слога.  Эти инфиксы всегда стоят после ин\-фик\-сов предпервой позиции. \label{morph:verb:first-position}

\begin{center}
\begin{tabular}{r|ccc}
 & Только время & Совершенный & Несовершенный \\
\hline
Будущее & \N{\INF{ay}, \INF{asy}} & \N{\INF{aly}} & \N{\INF{ary}} \\
Ближайшее будущее & \N{\INF{ìy}, \INF{ìsy}} & \N{\INF{ìly}} & \N{\INF{ìry}} \\
Обобщённое   &  — & \N{\INF{ol}} & \N{\INF{er}} \\
Ближайшее прошлое & \N{\INF{ìm}} & \N{\INF{ìlm}} & \N{\INF{ìrm}} \\
Прошлое & \N{\INF{am}} & \N{\INF{alm}} & \N{\INF{arm}} \\
\end{tabular}
\end{center}
\LanguageLog{}
\LNWiki{27/1/2010}{http://wiki.learnnavi.org/index.php/Canon\%23Extracts_from_various_emails}
\LNWiki{19/2/2010}{http://wiki.learnnavi.org/index.php/Canon\%23Evidential} %ìrm

\subsubsection{} Будущие времена с \N{s}, \N{\INF{asy}}
и \N{\INF{ìsy}}, обозначают намерение (\horenref{syn:verb:intenfut}).

\subsubsection{} Сослагательный инфикс \N{\INF{iv}} имеет ограниченный набор комбинаций и меньше градаций времени.
\index{subjunctive!infix}

\begin{center}
\begin{tabular}{r|ccc}
         & Только время & Совершенный & Несовершенный \\
\hline
Будущее & \N{\INF{ìyev}, \INF{iyev}} & — & — \\
Обобщённое & \N{\INF{iv}} & \N{\INF{ilv}} & \N{\INF{irv}} \\
Прошлое & \N{\INF{imv}} & — & —
\end{tabular}
\end{center}

\noindent\LNWiki{9/1/2010}{http://wiki.learnnavi.org/index.php/Canon\%23Extracts_from_various_emails}
\LNWiki{30/1/2010}{http://wiki.learnnavi.org/index.php/Canon\%23Vocabulary_and_.3Ciyev.3E}
\LNWiki{30/1/2010}{http://wiki.learnnavi.org/index.php/Canon\%23Fused_-iv-_Infixes}

\subsubsection{} Есть только два инфикса причастия. Их никогда не используют вместе с ин\-фик\-са\-ми времени, вида и настроения. \index{participle!formation}

\begin{center}
\begin{tabular}{lr}
Активное & \N{\INF{us}} \\
Пассивное & \N{\INF{awn}} \\
\end{tabular}
\end{center}

\noindent Поскольку причастия это прилагательные, которые нельзя использовать в качестве ска\-зуе\-мо\-го, то они всегда будут появляться с аффиксом прилагательного  \N{-a-} (\horenref{morph:adj-attr}, \horenref{syn:part:attr}).
\LNWiki{13/3/2011}{http://wiki.learnnavi.org/index.php/Canon/2010/March-June\%23Participial_Infixes}


\subsection{Вторая позиция} Инфиксы настроения указывают на отношение говорящего к сказанному, сомнение в высказанном утверждении, почтение к собеседнику или тор\-жест\-вен\-ность речи. Появляются в последнем слоге глагола, либо после инфиксов первой позиции в глаголе, состоящем из одного слога.
\index{attitude infixes}\label{morph:verb:2nd-pos}

\begin{center}
\begin{tabular}{rl}
Позитивное отношение & \N{\INF{ei}}, \N{\INF{eiy}} (\horenref{l-and-s:eiy-epenth}) \\
Негативное отношение & \N{\INF{äng}}, \N{\INF{eng}} (\horenref{l-and-s:eng}) \\
Почтение, церемония & \N{\INF{uy}} \\
Предположение & \N{\INF{ats}} \\
\end{tabular}
\end{center}

\noindent\LNWiki{19/2/2010}{http://wiki.learnnavi.org/index.php/Canon\%23Evidential} % <ats>

\subsection{Примеры} Описанные выше правила довольно абстрактны, поэтому я добавил здесь несколько примеров возможных изменений в формах слов у нескольких глаголов. 
\N{Eyk} \E{возглавлять} будет примером глагола из одного слога без начальной согласной, \N{fpak} \E{приостанавливать (какое-либо действие в процессе)} - глагола из одного слога с начальной согласной, \N{taron} \E{охотиться} - для обычного двухслогового глагола, который Фроммер часто использует в прмиерах и \N{yom·tìng} \E{кормить} в качестве составного глагола, где ин\-фик\-сы получает последняя часть.

\begin{center}\small
\begin{tabular}{lllll}
           & \N{eyk} & \N{fpak} & \N{\ACC{ta}ron} & \N{\ACC{yom}·tìng} \\
\hline
Ближ. буд. & \N{ì\ACC{meyk}} & \N{fpì\ACC{mak}} & \N{tì\ACC{ma}ron} & \N{\ACC{yom}tìmìng} \\
Возвр.  & \N{ä\ACC{peyk}} & \N{fpä\ACC{pak}} & \N{tä\ACC{pa}ron} & \N{\ACC{yom}täpìng} \\
Возвр., ближ. прошл. & \N{äpì\ACC{meyk}} & \N{fpäpì\ACC{mak}} & \N{täpì\ACC{ma}ron} & \N{\ACC{yom}täpìmìng} \\
Церемониальность & \N{u\ACC{yeyk}} & \N{fpu\ACC{yak}} & \N{\ACC{ta}ruyon} & \N{\ACC{yom}tuyìng} \\
Соверш. в., церем. & \N{olu\ACC{yeyk}} & \N{fpolu\ACC{yak}} & \N{to\ACC{la}ruyon} & \N{\ACC{yom}toluyìng} \\
Возвр., соверш. в., церем. & \N{äpolu\ACC{yeyk}} & \N{fpäpolu\ACC{yak}} & \N{täpo\ACC{la}ruyon} & \N{\ACC{yom}täpoluyìng} \\
\end{tabular}
\end{center}

\noindent Толкование некоторых примеров может не поддаваться логике. Цель состоит только в том, чтобы показать расположение инфиксов с помощью последовательного набора гла\-голь\-ных форм.
