% Sun Dec  2 12:35:11 2012 - REORGANIZE the charts to look more like
% https://forum.learnnavi.org/navi-lernen/das-navi-zahlensystem/
\nchapter{Числа}
\noindent Язык На'ви использует восьмеричную систему счисления \textit{octal} с основанием восемь. Как в немногих человеческих языках.\footnote{Видимо результат пересчета не пальцев, а промежутков между ними.}  Вместо подсчёта чисел по десятеричной формуле $(m \times 10) + n$ (как в $(4 \times 10) + 2 = 42_{10}$,
\E{сорок два}) используется формула $(m \times 8) + n$
(как с $(5 \times 8) + 2 = 52_8$, \N{mrrvomun}, $42_{10}$).

\section{Количественные числительные}
\index{numbers!cardinal}

\subsection{``Единицы''} Независимые формы числительных от одного до восьми:

\begin{center}
\begin{tabular}{ll}
1 & \N{'aw} \\
2 & \N{\ACC{mu}ne} \\
3 & \N{pxey} \\
4 & \N{tsìng} \\
\end{tabular}
\hskip 3em
\begin{tabular}{ll}
5 & \N{mrr} \\
6 & \N{\ACC{pu}kap} \\
7 & \N{\ACC{ki}nä} \\
8 & \N{vol} \\
\end{tabular}
\end{center}

\subsection{Степени Восьми} Вместо привычных ``десятков''  в На'ви используются вось\-ме\-рич\-ные основания, которые я бы забавно назвал ``восьмятками:''

\begin{center}
\begin{tabular}{ll}
8 (1 $\times$ 8) & \N{vol} \\
16 (2 $\times$ 8) & \N{\ACC{me}vol} \\
24 (3 $\times$ 8) & \N{\ACC{pxe}vol} \\
32 (4 $\times$ 8) & \N{\ACC{tsì}vol} \\
\end{tabular}
\hskip 3em
\begin{tabular}{ll}
40 (5 $\times$ 8) & \N{\ACC{mrr}vol} \\
48 (6 $\times$ 8) & \N{\ACC{pu}vol} \\
56 (7 $\times$ 8) & \N{\ACC{ki}vol} \\
64 (8 $\times$ 8) & \N{zam} \\
\end{tabular}
\end{center}

\noindent Большая степень восьмеричного основания это \N{\ACC{vo}zam} (512, восьмеричная
1000) и \N{\ACC{za}zam} (4096, восьмеричные 10000).

\subsection{Зависимые формы} В сочетании со словани степеней, слова основных чис\-ли\-тель\-ных принимают сокращённую до одного слога аббревиатуру, которая получает леницию, ес\-ли возможно: \label{numbers:dependent} \index{lenition!numbers}

\begin{center}
\begin{tabular}{ll}
1 & \N{(l)-aw} \\
2 & \N{-mun} \\
3 & \N{-pey} \\
4 & \N{-sìng} \\
\end{tabular}
\hskip 3em
\begin{tabular}{ll}
5 & \N{-mrr} \\
6 & \N{-fu} \\
7 & \N{-hin} \\
\end{tabular}
\end{center}

\subsubsection{} Все зависимые формы, кроме ``единицы'', \N{(l)-aw},
изгоняют \N{-l},  стоящую в конце ``восьмятков''.  Подобным образом, последняя \N{-m} в формах \N{zam, vozam,} и \N{zazam} убирается перед всеми зависимыми формами, кроме ``единицы,'' \N{zamaw,} но \N{za\ACC{mun},
za\ACC{pey}}, и т.д.

\subsubsection{} Ударение падает на присоединяемые зависимые формы.
В сочетании с \N{vol} \E{восемь}: 

\begin{center}
\begin{tabular}{ll}
9 (1$\times$8 $+$ 1) & \N{vo\ACC{law}} \\
10 (1$\times$8 $+$ 2) & \N{vo\ACC{mun}} \\
11 (1$\times$8 $+$ 3) & \N{vo\ACC{pey}} \\
12 (1$\times$8 $+$ 4) & \N{vo\ACC{sìng}} \\
\end{tabular}
\hskip 3em
\begin{tabular}{ll}
13 (1$\times$8 $+$ 5) & \N{vo\ACC{mrr}} \\
14 (1$\times$8 $+$ 6) & \N{vo\ACC{fu}} \\
15 (1$\times$8 $+$ 7) & \N{vo\ACC{hin}} \\
16 (2$\times$8 $+$ 0) & \N{\ACC{me}vol} \\
\end{tabular}
\end{center}

\noindent Эта закономерность продолжится в \N{\ACC{me}vol}:
\N{mevo\ACC{law}}, \N{mevo\ACC{mun}}, \N{mevo\ACC{pey}}, и т.д.

После \N{zam} счёт идет: \N{zam, za\ACC{maw}, za\ACC{mun},
za\ACC{pey}, za\ACC{sìng}, za\ACC{mrr}, za\ACC{fu},
za\ACC{hin}, \ACC{za}vol}, а затем продолжается как \N{zavo\ACC{law}},
т.д.  Например, восьмеричным 211 будет \N{mezavolaw}.
\NTeri{1/4/2014}{https://naviteri.org/2014/03/value-and-worth/\#comment-2678}
\LNForum{27/1/2021}{https://forum.learnnavi.org/language-updates/cardinal-zam-vozam-and-zazam/}

\section{Порядковые числительные}

\subsection{Суффикс -ve} Порядковые числительные образуются при попощи суффикса \N{-ve}, который не меняет на ударение в слове, хотя приводит к изменению некоторых корней. \index{-ve@\textbf{-ve}}
\index{numbers!ordinal}

\begin{center}
\begin{tabular}{rll}
Порядковые & Независимая & Зависимая \\
\hline
первый & \N{\ACC{'aw}ve} & \N{(l)-\ACC{aw}ve} \\
второй & \N{\ACC{mu}ve} & \N{-\ACC{mu}ve} \\
третий & \N{\ACC{pxey}ve} & \N{-\ACC{pey}ve} \\
четвёртый & \N{\ACC{tsì}ve} & \N{-\ACC{sì}ve} \\
пятый & \N{\ACC{mrr}ve} & \N{-\ACC{mrr}ve} \\
шестой & \N{\ACC{pu}ve} & \N{-\ACC{fu}ve} \\
седьмой & \N{\ACC{ki}ve} & \N{-\ACC{hi}ve} \\

\end{tabular}
\hskip2em
\begin{tabular}{rll}
\\
восьмой & \N{\ACC{vol}ve} & \N{-\ACC{vol}ve} \\
64-ый & \N{\ACC{za}ve} & \N{-\ACC{za}ve} \\
512-ый & \N{vo\ACC{za}ve} & \N{-vo\ACC{za}ve} \\
4096-ой & \N{za\ACC{za}ve} & \N{-za\ACC{za}ve} \\ 
\end{tabular}
\end{center}

\subsubsection{} Порядковые числительные ведут себя как прилагательные и получают аффикс \N{-a-}, если они использованы как определение (\horenref{morph:adj-attr}). Например, \Npawl{mrrvea ikran} \E{пятый икран}.
\LNForum{27/1/2021}{https://forum.learnnavi.org/language-updates/cardinal-zam-vozam-and-zazam/}

\subsubsection{} Все порядковые числительные можно свободно сочетать с  \N{nì-} для образования наречий, \N{nì'awve} \E{во-первых,} \N{nìmuve} \E{во-вторых,} и т.д.
\LNWiki{5/6/2013}{https://wiki.learnnavi.org/Canon/2013\%23Ordinals_.26_nume}


\section{Дроби}
\index{numbers!fractions}\index{fractions}

\subsection{-Pxì} За исключением \E{половины} и \E{трети}, у которых есть свои названия, дроби образуются с помощью замены \N{-ve} на \N{-pxì} в порядковом числительном.  Обратите вни\-ма\-ние на изменение ударений:
\index{-pxiì@\textbf{-pxì}}

\begin{center}
\begin{tabular}{rl}
половина & \N{mawl} \\
треть & \N{pan} \\
четверть & \N{tsì\ACC{pxì}} \\
пятая часть & \N{mrr\ACC{pxì}} \\
\end{tabular}
\hskip2em
\begin{tabular}{rl}
шестая часть & \N{pu\ACC{pxì}} \\
седьмая часть & \N{ki\ACC{pxì}} \\
восьмая часть & \N{vo\ACC{pxì}} \\
\\
\end{tabular}
\end{center}

\subsubsection{Word Class} Запомните, что в отличие от количественных и порядковых числительных, дро\-би являются существительными, а не прилагательными
(смотрите \horenref{syn:partitive-gen} для синтаксиса).

\subsection{Нумератор} Что бы получить большие дроби, объедините указывающее ко\-ли\-че\-ство числительное с существительным дроби,  \N{munea mrrpxì} \E{две пятых}.

\subsubsection{Две трети} Дробь \E{две трети} имеет отдельную форму, \N{mefan}. Двойственное число у \N{pan}. \index{mefan@\textbf{mefan}}


\section{Другие формы}

\subsection{Alo} Слово \N{\ACC{a}lo} \E{случай, раз} объединяют с числительными для образования наречий, которые перечисляют разы совершения действия. Четыре таких наречия записываются одним словом,
\N{\ACC{'aw}lo} \E{единожды}, \N{\ACC{me}lo} \E{дважды}, \N{\ACC{pxe}lo}
\E{трижды, три раза} и \N{\ACC{fra}lo} \E{каждый раз, во всех случаях}.
А все остальные образуют при помощи сочетания с прилагательным, \Npawl{\uwave{alo
amrr} poan polawm} \E{он спросил \uwave{пять раз}}. \index{melo@\textbf{melo}}
\index{'awlo@\textbf{'awlo}}\index{alo@\textbf{alo}}\index{fralo@\textbf{fralo}}
\index{pxelo@\textbf{pxelo}}

\subsection{-lie} Слово \N{'aw\ACC{li}e} \E{однажды} отсылает к единичному событию в прошлом. \index{-lie@\textbf{-lie}}\index{'awlie@\textbf{'awlie}}

\subsection{Чужие цифры} При цитировании английских цифр, язык На'ви использует слова
\N{'eyt} для \E{восьми} и \N{nayn} для \E{девяти}. Их не используют для подсчета, а только в  таких вещах как, например, номера телефонов.
\index{'eyt@\textbf{'eyt}}\index{nayn@\textbf{nayn}}

\subsubsection{} \N{Kew} это \E{ноль}.  \QUAESTIO{Текущая документация не уточняет, является ли это понятие род\-ным или оно заимствовано у людей.}
\index{kew@\textbf{kew}}
