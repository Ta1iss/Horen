\nchapter{Построение слов}

\section{Аффиксы словообразования}\index{affix!derivational}\label{lingop:affixes}
\noindent В На'ви есть определённый набор аффиксов для образования создания слов.
Какие-то из этих аффиксов просто меняют часть речи слова, например, делая из существительного прилагательное.  Однако, значение образованных слов не всегда бывает полностью ясным.  Удостовериться в точности перевода можно со словарём (о наречиях или ``adverbs'' \horenref{syn:nifyao}).  \uline{Аффиксы словообразования нельзя считать полностью продуктивным, если не указано обратное.}

Не смотря на то, что при словообразовании ударение ведёт себя довольно предсказуемо, следуя определённым паттернам, однако универсальных правил для ударения нет. Точно определить положение ударения в образованных словах поможет словарь.


\subsection{Префиксы} Данные префиксы словообразования не всегда меняют ударение, делая его отличным от того, что было в исходном слове, \N{ngay} \E{правдивый, истинный}
$>$ \N{tì\ACC{ngay}} \E{правда}. 

\subsubsection{} \N{Le-} образует прилагательные от существительных. Например,
\N{le\ACC{hrr}ap} \E{опасный} от \N{\ACC{hrr}ap} \E{опасность}.
\index{le-@\textbf{le-}}

\subsubsection{} \N{Nì-} образует наречия от существительных, местоимений, прилагательных и глаголов. Например, \N{nìNa'vi} \E{по-На'вийски, подобно На'ви} от
\N{Na'vi} \E{Народ}, \N{nìayfo} \E{подобно им}, \N{nì\ACC{ftu}e} \E{легко} от
\N{\ACC{ftu}e} \E{легкий, простой}, и \N{nì\ACC{tam}} \E{достаточно, довольно} от \N{tam}
\E{быть достаточным}. \index{niì-@\textbf{nì-}}

Префикс \N{nì-} является ограниченно продуктивным. Его можно свободно использовать только с прилагательными, местоимениями и порядковыми числительными (\E{первый, второй и т.д.}). Непродуктивен с другими частями речи.
\NTeri{11/7/2010}{https://naviteri.org/2010/07/diminutives-conversational-expressions/}
\LNWiki{5/6/2013}{https://wiki.learnnavi.org/Canon/2013\%23Ordinals_.26_nume}
\LNWiki{27/7/2013}{https://wiki.learnnavi.org/index.php/Canon/2013\%23Na.27vi_details_from_Avatarmeet_2013}
\LNForum{16/8/2016}{https://forum.learnnavi.org/language-updates/eapressions-of-'like-we'/}

\subsubsection{} \N{Sä-} образует ``инструментальные'' существительные от глаголов и прилагательных. Такие существительные обозначают некий инструмент или средство, использование которого приводит к результату, который представляет исходное слово. Например, \N{sä\ACC{nu}me} \E{обучение, инструктаж} от
\N{\ACC{nu}me} \E{учиться}. Обучение приводит к учёбе. Или \N{sä\ACC{spxin}} \E{болезнь, недомогание} от \N{spxin}
\E{больной}. В этом примере подразумевается не состояние болезни, а то что её вызывает - инфекция. Например, простуда и т.д.  \index{saä-@\textbf{sä-}}

\subsubsection{} \N{Sä-} также используется для обозначения конкретного примера общего действия. \N{Sätsyìl} \E{подъем} это конкретный пример действия для \N{tsyìl} \E{взбираться, карабкаться}.  От одной основы могут образовываться слова как при помощи префикса \N{tì-}, так и \N{sä-}. Например, \N{'ipu}
\E{юморной}.  \N{Tì'ipu} будет являться общим понятием, то есть самим юмором. А \N{sä'ipu} это конкретный пример юмора - \E{шутка}.\\
\NTeri{29/2/2012}{https://naviteri.org/2012/02/trr-asawnung-lefpom-happy-leap-day/}

\subsubsection{} \N{Tì-} образует существительные от прилагательных, глаголов и, реже, от других существительных. Например, слово \N{tì\ACC{ngay}} \E{правда} образовано от прилагательного
\N{ngay} \E{правдивый, истинный}, \N{tìfti\ACC{a}} \E{изучение} от глагола
\N{fti\ACC{a}} \E{изучать}, \N{tì\ACC{'awm}} \E{кемпинг} от \N{'awm}
\E{лагерь}. \index{tiì-@\textbf{tì-}}

\subsubsection{} При присоединении этих префиксов коренные слоги исходного слова могут терять гласную, если начальная согласная исходного слова окажется разрешенной финалью. Финаль это согласные, закрывающие слог. В нашем случае - \N{p, t, k, px, tx, kx, ', m, n, l, r, ng}. \N{Nìm\ACC{wey}pey}
\E{терпеливо} $<$ \N{ma\ACC{wey}·pey} \E{терпеть, быть терпеливым}.


\subsection{Префикс отрицания} Некоторые слова, и не всегда прилагательные, создаются с использованием слова \N{ke} \E{не} в качестве префикса.

\subsubsection{} Если \N{ke-} идет перед префиксом прилагательного
\N{le-}, то префикс прилагательного сокращается до \N{-l-}. Как в
\N{kel\ACC{tsun}} \E{невозможный} в сравнении с \N{le\ACC{tsun}slu}
\E{возможный}, и \N{kel\ACC{fpom}tokx} \E{нездоровый} от
\N{lefpom\ACC{tokx}} \E{здоровый}.

\subsubsection{} Если \N{le-} идет перед \N{ke-}, то префикс отрицания сокращается до \N{-k-}. Как в \N{lek\ACC{ye}'ung} \E{безумный} от
\N{ke\ACC{ye}'ung} \E{безумие}.

\subsubsection{} Префикс \N{ke-} может использоваться с коренными прилагательными и причастиями. В таких случаях ударение обычно смещается на \N{ke-}.
Как в \N{\ACC{ke}teng} \E{различный} от \N{teng} \E{похожий, подобный} и
\N{\ACC{ke}rusey} \E{мёртвый} от \N{ru\ACC{sey}} \E{живой}.  Однако,
отметьте то, что \N{ke\ACC{yawr}} \E{неверный, неправильный} от \N{e\ACC{yawr}} \E{верный, правильный} ударение остается неизменным.
\label{lingop:prefix:ke}

\subsubsection{} Префикс \N{ke-} также может образовывать существительные и соединяться с ними. Например, \N{ke\ACC{ye}'ung} \E{безумие} и \N{\ACC{ke}tuwong}
\E{чужак}. Но, для определения поведения ударения, примеров недостаточно.


\subsection{``A-'' у наречий} Два глагола состояния \N{lìm} \E{быть далеко}
и \N{sim} \E{быть рядом} имеют два собственных наречия \N{a\ACC{lìm}} \E{далеко} и \N{a\ACC{sim}} \E{близко}.  Возможно, они произошли от старых сокращений в \N{nìfya'o a
lìm} (\horenref{syn:nifyao}).  Это устоявшиеся словарные единицы и не имеют таких форм как \N{*lìma} и \N{*sima}.
\index{-a-@\textbf{-a-}!with adverbs}\index{aliìm@\textbf{alìm}}\index{asim@\textbf{asim}}
\LNWiki{17/5/2010}{https://wiki.learnnavi.org/index.php/Canon/2010/March-June\%23Near.2C_Distant_and_Irregular_Adverbs}


\subsection{Префикс с инфиксом} Только в одном случае при словообразовании используется сочетание префикса с инфиксом.

\subsubsection{} \N{Tì- ‹us›} образует герундий.  Данное сочетание можно свободно использовать с основами глаголов и составных слов (\N{si}-глаголы,
\horenref{lingop:si-const}, преобразовывать в герундий нельзя).  Это наиболее полезно,  если простое образование от \N{tì-} уже имеет устоявшийся смысл. Как в случае с \N{rey} \E{жить}, \N{tì\ACC{rey}} \E{жизнь},
тогда \N{tìru\ACC{sey}} будет означать \E{живущий}.  В составных словах \N{tì-} ставится в начало слова, а \N{‹us›} отправляется в глагольную часть, которая получает инфиксы, \N{yomtìng} \E{кормить} станет \N{tìyomtusìng} \E{кормление}. Смотрите также
\horenref{syn:gerund}.
\index{gerund!formation}\label{lingop:gerund}
\index{si construction@\textbf{si} construction!no gerund}
\LNForum{31/1/2013}{https://forum.learnnavi.org/language-updates/confirmations-(comparisons-gerund)/msg572997/}


\subsection{Агентные суффиксы} Данные суффиксы не изменяют ударение в слове.

\subsubsection{} \index{-tu@\textbf{-tu}}
\N{-tu} образует \E{лат. nomen agentis} - агентные существительные, которые обозначают исполнителя значения исходного слова. Чаще создает слова от других частей речи, нежели глаголов. Например, \N{\ACC{pam}tseotu} \E{музыкант}
от \N{\ACC{pam}tseo} \E{музыка}, \N{tsul\ACC{fä}tu} \E{мастер} от \N{tsul\ACC{fä}} \E{мастерство}.  При присоединении к глаголу, существительное может отсылать как к исполнителю, так и к получателю глагольного действия. Как с \N{\ACC{frr}tu} \E{гость, посетитель}
от \N{\ACC{frr}fen} \E{посещать, навещать} (исполнитель) \N{spe\ACC{'e}tu} \E{пленник}
от \N{spe\ACC{'e}} \E{захватывать} (получатель), но не \N{spe\ACC{'e}yu} \E{захватчик}.  Суффикс \N{-tu} непродуктивен - его нельзя свободно использовать.
\NTeri{30/4/2021}{https://naviteri.org/2021/04/mipa-ayliu-mipa-sioeykting-new-words-new-explanations/}

\subsubsection{} \N{-yu} образует агентные существительные из глаголов, чтобы указать на того, кто регулярно выполняет какую-либо деятельность или роль. Как с
\N{taronyu} \E{охотник} от \N{taron} \E{охотиться}.  Этот инфикс продуктивен и его можно свободно использовать как с обычными, так и с \N{si}-глаголами, например, \N{stiwisiyu} \E{проказник, шалун} от \N{stiwi si} \E{проказничать, шалить}.
\index{-yu@\textbf{-yu}}
\NTeri{11/7/2010}{https://naviteri.org/2010/07/diminutives-conversational-expressions/}
\LNForum{30/10/2020}{https://forum.learnnavi.org/language-updates/yu-is-officially-productive-on-si-verbs/}


\subsection{Уменьшительно-ласкательный суффикс} Безударный суффикс \N{-tsyìp} можно свободно использовать для создания уменьшительных и ласкательных форм существительных или местоимений.  Личные имена могут терять слоги при присоединении этого суффикса. Как в \N{Kamtsyìp} или
\N{Kamuntsyìp} для \E{Камунчик}.  Уменьшительность используется в трёх случаях.
\label{lingop:dimin}\index{diminutive}\index{tsyiìp@\textbf{-tsyìp}}
\NTeri{11/7/2010}{https://naviteri.org/2010/07/diminutives-conversational-expressions/}

\subsubsection{} Во-первых, уменьшительное может быть прямым лексическим образованием.  Такие слова указываются в словаре напрямую. Так, например,
\N{puktsyìp} \E{буклет, памфлет} от \N{puk} \E{книга}.  Уменьшительная сила достаточно слаба, поэтому можно использовать прилагательное \N{tsawl} \E{большой, высокий} с уменьшительным без противоречия как, например, 
\N{tsawla utraltsyìp} \E{большой куст}.

\subsubsection{} Во-вторых, уменьшительное может выражать ласку или нежность, \Npawl{za'u fì\-tseng, ma 'itetsyìp} \E{подойди сюда, доченька}. Такое использование не подразумевает возраст.  Дочь в предыдущем примере может быть уже взрослой.

\subsubsection{} В-третьих, уменьшительное может выражать пренебрежение или оскорбление, \Npawl{fìtaron\-yu\-tsyìp ke tsun ke'ut stivä'nì} \E{этот
(жалкий) охотничек ничего не может поймать}.  Пренебрежительный тон может быть направлен на самого себя, \Npawl{nga nìawnomum to
\uwave{oetsyìp} lu txur nìtxan} \E{как известно, ты намного сильнее, чем \uwave{жалкий я}}.  Использование пренебрежения или ласки возможно определить только по контексту.


\subsection{-Nay} Этот суффикс образует существительные, которые указывают на что-то, находящееся ниже по иерахии, размеру, чину, достижениям и т.д.
Суффикс получает ударение, \N{karyu\ACC{nay}} \E{помощник учителя} от \N{karyu} \E{учитель}.  Если существительное уже оканчивается на
\N{-n}, то суффикс теряет свою \N{-n-}, \N{'eyla\ACC{nay}}
\E{знакомый} от \N{'eylan} \E{друг}, \N{ikra\ACC{nay}}
\E{лесной банши} от \N{ikran} \E{банши}.  Данный суффикс непродуктивен.
\index{-nay@\textbf{-nay}}
\NTeri{2/28/2013}{https://naviteri.org/2013/02/vospxi-ayol-posti-apup-short-post-for-a-short-month/}


\subsection{Суффиксы рода} Гендерные суффиксы необычны тем, что они используются не только с существительными, но и с местоимениями 3-го лица
(\horenref{morph:pron:gender}).  \label{lingop:suffix:gender}

\subsubsection{} Суффикс \N{-an} указывает на мужской род, например
\N{po\ACC{an}} \E{он} и \N{\ACC{'i}tan} \E{сын}.

\subsubsection{} Суффикс \N{-e} указывает на женский род, например
\N{po\ACC{e}} \E{она} и \N{\ACC{'i}te} \E{дочь}.

\subsubsection{} Влияние этих суффиксов на ударение трудно предугадать, \N{tu\ACC{tan}} \E{мужчина} от \N{\ACC{tu}te}
\E{личность,} но \N{mun\ACC{txa}tan} \E{муж} от
\N{mun\ACC{txa}tu} \E{супруг}.


\section{Удвоение}
\noindent Удвоение - непродуктивный процесс словообразования.
Тем не менее, некоторые обычные слова его используют. \index{reduplication}

\subsection{Повторение} Со словами времени, удвоение указывает на повторение или обычное явление, \N{letrrtrr} \E{обычный, повседневный,} которое происходит каждый день; и \N{krro krro} \E{иногда}.

\subsection{Изменение степени} У глаголов \N{'ul} \E{увеличиваться} и
\N{nän} \E{уменьшаться}, удвоение в наречиях указывает на изменение до крайней степени, \N{nì'ul'ul} \E{увеличивая, больше и больше,}
\N{nìnänän}\footnote{Удвоение является частичным, поскольку согласные не могут удваиваться.} \E{уменьшая, меньше и меньше}.\\
\NTeri{29/2/2012}{https://naviteri.org/2012/02/trr-asawnung-lefpom-happy-leap-day/}

\newpage
\section{Составные слова}

\subsection{Заглавность} Основной элемент составного слова может быть первым или последним.\footnote{Многие человеческие языки более строги.  Английские составные слова, как правило, имеют основной элемент, или ``голову,'' последним. Как в \textit{blueberry},
\textit{night-light}, \textit{blackboard}.  В свою очередь, вьетнамский язык для родных составных слов использует такой порядок, где основная часть находится в начале, а для составных слов, заимствованных из китайского, помещает основную часть в конец.}  Вместе с тем есть тенденция к тому, что в составных словах основная часть является последней.  А составные глаголы чаще имеют основную часть в начале.

\subsubsection{} Часть речи составных слов определяется их ``головой'',
поэтому \N{txam\ACC{pay}} \E{море} является существительным, так как \N{pay} \E{вода, жидкость} это существительное.

\subsubsection{} Подобно коренным словам, составные слова так же меняют свою часть речи при присоединении словообразующих аффиксов, перечисленных выше,
\N{lefpom\ACC{tokx}} \E{здоровый} от \N{fpom\ACC{tokx}} \E{здоровье}.


\subsection{Апокопа} Слова могут лишаться своих частей, если используются в составных словах. Как в \N{\ACC{ven}zek} \E{палец ноги} $<$ \N{\ACC{ve}nu} \E{нога} $+$
\N{\ACC{zek}wä} \E{палец} и \N{sìl\ACC{pey}} \E{надеяться} $<$
\N{sìltsan} \E{хороший} $+$ \N{pey} \E{ждать}.


\subsection{``Si'' конструкция} Распространенный способ преобразования существительного или прилагательного в глагол — это соединение грамматически неизменное существительное с глаголом-помощником
\N{si}, который используется только в составе таких конструкций. Порядок слов закрепленный \N{N si}, а в часть \N{si} вставляются аффиксы глаголов.\label{lingop:si-const}
\index{si construction@\textbf{si} construction}

\subsubsection{} В глаголе \N{irayo si} \E{благодарить} порядок менее строг.
\LNWiki{12/5/2010}{https://wiki.learnnavi.org/index.php/Canon/2010/March-June\%23Word_Order_Issues}

\subsubsection{} Кроме того, принятый порядок \N{N si} может не соблюдаться при отрицании, \N{oe pamrel ke si} \E{я не писал} (\horenref{syn:neg:si-const}),
\N{txopu rä'ä si} \E{не бойся} (\horenref{syntax:prohibitions}).


\section{Общие и видные элементы составных слов}

\subsection{-fkeyk} Означает состояние существительного, к которому он присоединён. Этот безударный суффикс образован от существительного \N{tìfkeytok} \E{состояние, ситуация} и образует некоторые слова с особым идиоматическим смыслом, таким как \N{\ACC{ya}fkeyk}
\E{погода}. Суффикс, тем не менее, довольно продуктивен, \Npawl{kilvan\uwave{fkeyk} lu fyape fìtrr?} \E{какое сегодня \uwave{состояние} реки?}
\index{-fkeyk@\textbf{-fkeyk}}
\NTeri{1/4/2011}{https://naviteri.org/2011/04/yafkeykiri-plltxe-frapo-everyone-talks-about-the-weather/}

\subsection{Hì(')-} Произошедший от прилагательного \N{hì'i} \E{маленький}, этот ударный префикс \N{hì-} или \N{hì'-} используется с некоторыми словами для создания уменьшительных форм и не считается продуктивным (see \horenref{lingop:dimin}),
например \N{\ACC{hì}'ang} \E{насекомое} ($<$ \N{hì'} + \N{ioang}
\E{животное}), \N{\ACC{hì}krr} \E{секунда, очень короткое время} ($<$ \N{hì} +
\N{krr} \E{время}).  \index{hiì(')-@\textbf{hì(')-}}

\subsection{-ìva} Когда существительное \N{ìlva} \E{чешуйка, капля, щепка} используется в составном слове, то \N{l} убирают, \N{\ACC{txe}pìva} \E{пепел, зола,}
\N{\ACC{her}wìva} \E{снежинка}.
\NTeri{1/4/2011}{https://naviteri.org/2011/04/yafkeykiri-plltxe-frapo-everyone-talks-about-the-weather/}
\index{-iva@\textbf{-ìva}}\index{ilva@\textbf{ìlva}}

\subsection{-nga'} Данный суффикс, произошедший от глагола \N{nga'}
\E{содержать}, создает из существительных прилагательные, которые описывают то, что ``содержит'' существительное. Например, \N{\ACC{txum}nga'} \E{ядовитый (содержащий яд)}.  Этот способ встречается реже, чем \N{le-}.  Существительные могут иметь образованные прилагательные как от \N{le-}, так и от \N{-nga'}, \N{lepay} \E{водянистый} vs.\
\N{\ACC{pay}nga'} \E{влажный}.
\index{-nga'@\textbf{-nga'}}
\NTeri{5/5/2011}{https://naviteri.org/2011/05/weather-part-2-and-a-bit-more-2/}

\subsection{-pin} Произошедший от существительного \N{'opin} \E{цвет}, этот безударный суффикс присоединяется к прилагательным, описывающим цвета, и образует из них существительные, которые являются названиями цветов, сущ.
\N{\ACC{rim}pin} \E{желтый (цвет)} от прил. \N{rim} \E{желтый}.  Из-за ассимиляции, последняя буква \N{-n} в цветовых прилагательных становится \N{-m}, \N{\ACC{e}ampin} от \N{\ACC{e}an}.
\index{'opin@\textbf{'opin}}\index{-pin@\textbf{-pin}}

\subsection{Pxi-} Прилагательное \N{pxi} \E{острый}, присоединяется в качестве префикса к наречиям и предлогам времени для обозначения непосредственности.  На префикс не падает ударение, \N{pxi\ACC{sre}} \E{непосредственно перед},
\N{pxi\ACC{set}} \E{в настоящий момент, прямо сейчас}.
\index{pxi-@\textbf{pxi-}}

\subsection{Sna-} Сокращенная форма существительного \N{sna'o} \E{группа, множество, комплект}. Данный префикс можно свободно использовать с живыми объектами, за исключением людей, чтобы обозначать естественное скопление, например,
\N{snatalioang} \E{стадо стурмбистов}, \N{snautral} \E{роща}.  Префикс используется и с неживыми объектами, но в таком случае не является продуктивным, а производные это словарныме слова, \N{snatxärem} \E{скелет}.
\NTeri{31/3/2012}{https://naviteri.org/2012/03/spring-vocabulary-part-2/}
\index{sna-@\textbf{sna-}}

\subsection{-tseng} Существительное \N{tseng} \E{место} в разговорной речи
может присоединяться в качестве суффикса к глаголам для создания слов, означающих места где происходит соответствующее действие, как \N{yomtseng}, обозначающее место приема пищи.  Такие слова могут быть не представлены в словаре.
\LNForum{5/7/2023}{https://forum.learnnavi.org/language-updates/verb-tseng-compounds-in-informal-speech/}

В некоторых ``официальных'' составных словах с \N{tseng} другие части могут иметь различные фонетические изменения, как в слове \N{numtseng} \E{школа,} в котором глагол \N{nume} изменился внутри составного слова.

\subsection{-tsim} Существительное \N{tsim} \E{источник} может присоединяться к существительным в качестве суффикса, чтобы указать источник или причину какого-либо состояния, как \N{\ACC{sngum}tsim} \E{источник беспокойства, тревожный повод}
от \N{sngum} \E{беспокойство}, \N{ya\ACC{yayr}tsim} \E{что-то сбивающее с толку,
источник путаницы} от \N{yayayr} \E{путаница,}
\N{\ACC{ing}yentsim} \E{тайна, загадка}
от \N{ingyen} \E{чувство тайны и непостижимости}.  Отметьте что при присоединении ударение не меняется.  Этот суффикс непродуктивный.
\NTeri{25/1/2013}{https://naviteri.org/2013/01/awvea-posti-zisita-amip-first-post-of-the-new-year/}
\index{-tsim@\textbf{-tsim}}

\subsection{Tsuk-} Этот безударный префикс произошел от \N{tsun fko}. Он образует из переходных глаголов прилагательные, описывающие возможность, \N{tsuk\ACC{yom}}
\E{съедобный} (от \N{yom} \E{есть}).  Или невозможность. Тогда для отрицания к префиксу добавляется \N{ke-}. \N{Ketsuk-} также является безударным префиксом,
\N{ketsuk\ACC{tswa'}} \E{незабываемый} (от \N{tswa'} \E{забывать}).
\index{tsuk-@\textbf{tsuk-}}\index{ketsuk-@\textbf{ketsuk-}}
\NTeri{22/3/2011}{https://naviteri.org/2011/03/\%E2\%80\%9Creceptive-ability\%E2\%80\%9D-and-hesitation/}

\subsubsection{} В добавок, непереходные глаголы можно сочетать с \N{tsuk-}, но связь между существительным и получившимся в итоге прилагательным будет слабее, \Npawl{fìtseng lu tsuk\-tsurokx}
\E{здесь можно отдохнуть, это место “отдыхаемое,''} \Npawl{lu na’rìng
tsukhahaw} \E{в лесу можно спать, лес доступный для сна}.

\subsection{-tswo} Этот суффикс можно свободно использовать с любым глаголом, чтобы образовывать существительные, означающие способность совершать действие глагола,
\N{wemtswo} \E{способность драться,} \N{roltswo} \E{способность петь}.
Этот суффикс связан со словом \N{tsu'o} \E{способность}.
\NTeri{31/3/2012}{https://naviteri.org/2012/03/spring-vocabulary-part-2/}
\index{-tswo@\textbf{-tswo}}

\subsubsection{} Суффикс \N{-tswo} присоединяется к существительным или прилагательным, которые являются элементами \N{si}-глаголов, как в \N{srung\-tswo} \E{способность помогать} и \N{tstutswo} \E{способность закрывать}.

\subsection{-vi}  Безударный суффикс \N{-vi} произошел от существительного \N{'evi}, являющегося сокращенной формой слова \N{'eveng} \E{ребенок}. Этот суффикс довольно свободно используется для обозначения малой части чего-то большего,
\N{\ACC{txep}vi} \E{искра} ($<$ \N{txep} \E{огонь}), \N{\ACC{lì'}fyavi}
\E{выражение, фраза} ($<$ \N{lì'fya} \E{язык}).  Может вызывать небольшие изменения в слове, к которому присоединяется, \N{sä\ACC{num}vi}
\E{урок} от \N{sä\ACC{nu}me} \E{инструкция, обучение}.
\index{-vi@\textbf{-vi}}
\LNWiki{14/3/2010}{https://wiki.learnnavi.org/index.php/Canon/2010/March-June\%23A_Collection}

\subsection{``Kä-'' и ``Za-''} Два глагола, которые описывают движение \N{kä} \E{идти}
и \N{za'u} \E{приходить} (сокращено до \N{za-}), используются в некоторых составных словах для обозначения направления движения, \N{kä\ACC{mak}to}
\E{выезжать, выдвигаться}.  Обратите внимание на разницу между \N{kä\ACC{'ä}rìp} \E{толкать}
и \N{za\ACC{'ä}rìp} \E{тянуть} от \N{\ACC{'ä}rìp} \E{двигать (что-либо)}.
\index{kaä-@\textbf{kä-}}\index{za-@\textbf{za-}}


\section{Время}
\noindent Наречия времени образуются от существительных по предсказуемой схеме.

\subsection{Текущее время} Префикс \N{fì-}
(\horenref{morph:prenoun:fi}) образует наречия для текущих единиц времени, \N{fìtrr} \E{сегодня} (``этот день''), \N{fìrewon} \E{этим утром}. \index{fiì-@\textbf{fì-}!in adverbs of time}

\subsection{Предыдущее время} Ударный суффикс \N{-am} образует наречия для предыдущих единиц времени, \N{trr\ACC{am}} \E{вчера,}
\N{pxiswaw\ACC{am}} \E{только что}.
\index{-am@\textbf{-am}}

\subsection{Дальнейшее время} Ударный суффикс \N{-ay} образует наречия для дальнейших единиц времени, \N{trr\ACC{ay}} \E{завтра}, 
\N{ha'ngir\ACC{ay}} \E{во второй половине завтрашнего дня}.
\index{-ay@\textbf{-ay}}
