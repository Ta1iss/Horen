\nchapter{Буквы и Звуки}


\section{Фонетика}
\noindent В языке На'ви 20 согласных, 7 гласных
и два слоговых резонирующих звука, которые Фроммер назвал ``псевдогласными.''
\LanguageLog

\subsection{Согласные}
Согласные На'ви перечислены ниже. В круглых скобках указано про\-из\-но\-ше\-ние на Рифовом На'ви некоторых фонем и звуковых ассимиляций, которые не имеют соб\-ствен\-но\-го написания. В квадратных скобках находятся транскрипции из меж\-ду\-на\-род\-но\-го звукового алфавита. Часть неочевидных звуков я снабдил русской транс\-крип\-ци\-ей. При\-ме\-ры произношения можно послушать тут: \href{https://www.ipachart.com/}{www.ipachart.com}

\begin{center}
\begin{tabular}{llllll}
 & Губные & Альвеолярные & Нёбные & Заднеязычные & Глоттальные \\
Эйективы &	\N{px} [p'] & \N{tx} [t'] & & \N{kx} [k'] \\
Глухие плозивы & \N{p} [p] & \N{t} [t] & & \N{k} [k] & \N{’} [ʔ] \\
Звонкие плозивы    &  ([b])    & ([d])    &  & ([g]) \\
Аффрикаты &             & \N{ts}  [ts]/[ц] & ([tʃ]/[ч]) \\
Глухие фрикативы & \N{f} [f] & \N{s} [s] & ([ʃ]/[щ]) & & \N{h} [h] \\
Звонкие фрикативы & \N{v} [v] & \N{z} [z] \\
Назальные &         \N{m} [m] & \N{n} [n] & & \N{ng} [ŋ] \\
Текучие &         &  \N{r} [ɾ], \N{l} [l] \\
Скользящие &       \N{w} [w] & &  \N{y} [j] \\
\end{tabular}
\end{center}

\subsubsection{} Глухие плозивы произносятся без придыхания в начале и середине слова. В конце слова, равно как и слога, их финальная артикуляция прерывается в том случае, если следом идет согласная, 
(как в \N{txe\uwave{p}mì} и \N{'o\uwave{k}trr}).  Вместе с тем внутри фразы последний плозив, стоящий перед гласной, в естественной речи будет произноситься так как если бы слова сливались воедино, \N{oel se\uwave{t o}mum}.
Без придыха наиболее выражены во время крупных пауз, как в \N{oel
omum se\uwave{t}.}
\NTeri{23/12/2020}{https://naviteri.org/2020/12/mrra-tipangkxotsyip-five-little-discussions/}

\subsubsection{} Звук \N{r} это альвеолярный дрожащий согласный. Аналогичен русскому ``Р''. Звук \N{l} это чистый ``Л.'' Не как веляризованный ``тёмный Л'' из английских ``call'' или``ball'.'

\subsubsection{} Фроммер разработал научную орфографию, в которой два диграфа писались од\-ной буквой, \N{c} для \N{ts} и
\N{g} для \N{ng}.  Актеры легче воспринимали двухбуквенные диграфы, да и сам Фроммер использовал их в интервью и почте. Однобуквенное обозначение появлялось лишь в ранних сообщениях Фроммера.  \label{l-and-s:cg}

\subsubsection{} Поскольку плозивы могут стоять в конце слога, то общепринятое обозначение для эйективов \N{p'} может вводить в заблуждение, учтите что
\N{tsap'alute}, это не *\N{tsapxalute}.
\LNWiki{21/12/2009}{https://wiki.learnnavi.org/index.php/Canon\%23Extracts_from_various_emails}

\subsection{Гласные}
Гласная в круглых скобках является отдельной фонемой Рифового На'ви.  Остальные используются в обоих диалектах.

\begin{center}
\begin{tabular}{ccccc}
\N{i} [i], \N{ì} [{\footnotesize I}]  & & & & (\N{ù} [ʊ]), \N{u} [u],[ʊ] \\
 & \N{e} [ɛ] & & & \N{o} [o] \\
 & & \N{ä} [æ] &  \N{a} [a] \\
\end{tabular}
\end{center}

\subsubsection{} В Рифовром На'ви \N{u} всегда звучит как [u] - аналогично русской ``у'',
а \N{ù} как звук [ʊ], представляющий что-то среднее между ``о'' и ``у''. \index{Reef Na'vi!ù@\textbf{ù}}
\LNWiki{8/1/2023}{http://naviteri.org/2023/01/reef-navi-part-1-phonetics-and-phonology/}

\subsubsection{} Лесной На'ви объединяет гласные \N{u} и \N{ù}, 
по-этому они всегда звучат как [u] в открытых слогах. В закрытых допускаются оба произношения: [u]
или [ʊ].  \N{Lu} всегда звучит как [lu],
но для \N{pum} возможны оба варианта: [pum] и [pʊm].
\LNWiki{20/5/2010}{https://wiki.learnnavi.org/index.php/Canon/2010/March-June\%23The_Dual_sounds_of_.22u.22}

% 2023jan10 - don't do this for now.
%In this grammar words that are known to have \N{ù} in them in Reef
%Na'vi are spelled as such, such as \N{pùm}.  When using Forest Na'vi,
%simply treat these words as though spelled \N{pum}.

\subsubsection{}  \N{Aw}, \N{ay}, \N{ew} и \N{ey} это дифтонги.
Только в дифтонгах \N{w} и \N{y} могут оказаться на конце слога (\N{new}) либо перед согласной, закрывающей слог, (\N{hawng}). Такие слоги как \N{*niw} или \N{*hoyng} появляться не могут.

\subsection{Псевдогласные} Псевдогласная \N{rr} это слоговый растянутый звук [r̩ː], а \N{ll} слоговый растянутый звук [l̩ː]. Псевдогласная \N{rr} произносится как растянутая русская ``Р'', подобно «рычанию», а \N{ll} как растянутая русская ``Л''.

\subsection{Строение слога}
 В На'ви имеет жесткие, но простые правила построения слога.

\begin{itemize*}
  \item В слоге допускается отсуствие начальной согласной (т.е., может начинаться с глас\-ной).
  \item В слоге допускается отсутствие конечной согласной (т.е., может кончаться на глас\-ной).
  \item Слог может начинаться с любой согласной.
  \item Группы согласных \N{f s ts} $+$ \N{p, t, k, px, tx, kx,
    m, n, ng, r, l, w, y} могут начинать слог (например, \N{tslam}, \N{ftu}).
  \item \N{P t k px tx kx ' m n l r ng} могут появляться в конце слога.
  \item \N{Ts f s h v z w y}  \textit{НЕ} могут появляться в конце слога.
  \item В конце слога не бывает групп согласных.
  \item \label{l-and-s:pseudo-no-null} Слог, содержащий псевдогласную, должен начинаться с согласной или груп\-пы со\-глас\-ных и не дол\-жен иметь со\-глас\-ной в конце; это играет свою роль в лениции (смягчении)
    (\horenref{l-and-s:lenition:pseudovowel}) и склонении существительных
    (\horenref{morph:decl:pseudovowel}).
\end{itemize*}

\newpage
\noindent Наглядный пример основных правил слогообразования:

\begin{center}\footnotesize
\begin{tikzpicture}[every path/.style={>=latex},every node/.style={draw,rectangle}]
\node[rounded corners,fill=gray!20,align=center] (sos) at (-4,1) {начало\\слога};
\node[align=center] (zero-onset) at (-0.5,2.5) { нет согласной\\в начале };
\node[align=center,font=\bfseries] (onset) at (-0.25,1) {p t k px tx kx '\\ m n ng r l w y\\f v s z ts h};
\node[font=\bfseries] (clust-c1) at (-2.5,-0.5) { f s ts };
\node[align=center,font=\bfseries] (clust-c2) at (0,-0.5) {p t k px tx kx\\m n ng r l w y};
\node[align=center] (vowel) at (3.5,2) { гласная или\\дифтонг };
\node (pseudovowel) at (3.5,0.5) { псевдогласная };
\node[align=center,font=\bfseries] (final) at (6.25,2.5) {p t k px tx kx ’\\m n l r ng };
\node[align=center] (zero-coda) at (6.25,1) { нет согласной\\в конце };
\node[rounded corners,fill=gray!20,align=center] (eos) at (8.75,1) { конец\\слога };

\draw[->] (sos) edge (clust-c1);
  \draw[->] (sos) edge (onset);
  \draw[->] (sos) edge (zero-onset);
\draw[->] (clust-c1) edge (clust-c2);
\draw[->] (zero-onset) edge (vowel);
\draw[->] (clust-c2) edge (vowel);
  \draw[->] (clust-c2) edge (pseudovowel);
\draw[->] (onset) edge (vowel);
  \draw[->] (onset) edge (pseudovowel);
\draw[->] (vowel) edge (final);
\draw[->] (final) edge (eos);
\draw[->] (vowel) edge (zero-coda);
\draw[->] (pseudovowel) edge (zero-coda);
\draw[->] (zero-coda) edge (eos);
\end{tikzpicture}
\end{center}

\subsubsection{} Так как у слога может не быть начальной или конечной согласных, то возможно появление в слове ряда стоящих друг за другом гласных.
В таком случае одиночная гласная - отдельный слог, \N{muiä} [mu.i.æ], \N{ioang}
[i.o.aŋ].

\subsubsection{} Принято цепочку ГСГ разбивать на слоги в виде Г.СГ, а не ГС.Г, так \N{tsenge} будет [tsɛ.ŋɛ], а не *[tsɛŋ.ɛ].
Звукоподражание может отменить это правило, как в \N{kxang\-ang\-ang} [k'aŋ.aŋ.aŋ], где нужен эффекта эха.

\subsubsection{} В Лесном На'ви нет долгих гласных. Это значит что одинаковые гласные не могут находиться друг за другом (смотри тут
\horenref{l-and-s:contract}).  В Рифовом На'Ви при некоторых случаях могут появляться долгие гласные из-за опускания произношения гортанной смыч\-ки \N{ ' } (\horenref{rn:stop-elision}) и отсутствия сокращения гласных
(\horenref{rn:no-contract}). 

\subsubsection{} Сдвоенные согласные не встречаются в корнях слов, но могут появляться на границах морфем. Как, например, в производных
\N{tsukkäteng} $<$ \N{tsuk-} $+$ \N{käteng} или эн\-кли\-ти\-ках, 
когда группа слов произносится с одним ударением и воспринимается на слух как единое целое 
\N{Mo'atta} $<$ \N{Mo'at} $+$ \N{ta} (\horenref{l-and-s:stress:enclisis}).
% https://naviteri.org/2011/03/“receptive-ability”-and-hesitation/comment-page-1/#comment-604

\subsubsection{} Междометия, как это принято в человеческих языках, могут нарушать эти правила. Например, \N{oìsss} - звук гнева или 
\N{saa} - угрожающий крик.


\subsection{Ударение}
У каждого слово в На'ви есть ударение и не всегда очевидное.  Бывает так, что ударение это единственное различие между полностью одинаковыми словами, как, например, в \N{\ACC{tu}te} \E{личность}
и \N{tu\ACC{te}} \E{женщина}.

\subsubsection{} Конкретно для слова \E{женщина} ударение принято указывать отдельно, \N{tuté}.
\index{tuté@\textbf{tuté}}

\subsubsection{} В процессе словообразования акцент ударения может меняться
(\horenref{lingop:prefix:ke}, \horenref{lingop:suffix:gender}).

\subsubsection{} Все предлоги, а также некоторые союзы и частицы могут быть энклитичны.  Они теряют свое ударение, когда становятся частью слова, к которому присоединяются и пишутся соответствующе. \N{\ACC{tsa}ne} ($<$ \N{tsaw} $+$ \N{ne}),
\N{ho\ACC{ren}\-ti\-sì} ($<$ \N{ho\ACC{ren}ti} $+$ \N{sì}).
\label{l-and-s:stress:enclisis}\index{enclitics}

\subsubsection{} В составных суще\-стви\-тель\-ных каждая их часть может сохранять ударение, на\-при\-мер, \N{ti\ACC{re}a\ACC{fya}'o} \E{путь духа}.
\index{compound word!accent}

%\subsubsection{} Word stress is a property of stem words.  No matter
%how many affixes a root word takes, no secondary accents develop.

\subsection{Рифовый На'ви} У рифового диалекта, по сравнению с лесным, есть отличия в гласных и согласных звуках. \index{Reef Na'vi}
\NTeri{8/1/2023}{http://naviteri.org/2023/01/reef-navi-part-1-phonetics-and-phonology/}

\subsubsection{} В Рифовом На'ви эйективные согласные в начале слога, а следовательно и в начале слова, произносятся как звонкие плозивы.  Так, \N{px tx kx} → [b d g].

\begin{center}
\begin{tabular}{lll}
\N{txon}    & [don] & \E{ночь} \\
\N{hol\ACC{pxay}} & [hol.ˈbaj] & \E{число} \\
\N{kxitx}   & [git'] & \E{смерть} \\
\N{skxawng} & [sk'awŋ] (as in Forest Na'vi) & \E{дурак}
\end{tabular}
\end{center}

\noindent Как правило, это не отображатся письменно. По-этому \E{ночь} на рифовом диалекте будет писаться как \N{txon}, но выговариваться при чтении как \N{don}. При желании использование рифового диалекта можно выделить прямой записью, \N{don}.

Слово \N{tìkankxan} может читаться двояко при записи на Рифовом На'ви, \N{*tìkangan}.  Чтобы не было интерпретации \N{...ng...} как [ŋ] вместо сочетания [ŋg], то добавляется ин\-тер\-пункт (желательнее) или дефис: \N{tìkan·gan}
и \N{tìkan-gan}.
\Omaticon{} \NTeri{15/1/2023}{http://naviteri.org/2023/01/2653/\#comment-45092}

\paragraph{} При присоединении суффикса, начинающегося с гласной, к слову, окан\-чи\-ва\-ющ\-им\-ся на эйективную согласную, произношение меняется в соответствии с новой сло\-го\-вой струк\-ту\-рой. Например, транскрипция слова \N{'awkx} это
[ʔawk'], однако голосом \N{'awkxit} будет как [ˈʔaw.git].
\NTeri{13/1/2023}{http://naviteri.org/2023/01/2653/}



\subsubsection{} Звуки \N{s} и \N{ts} палатализируются, если стоят вместе с \N{y}.  Так что, \N{sy} будет про\-из\-но\-сит\-ся как [ʃ] [щ], а \N{tsy} как [tʃ] [ч].
Такая палатализация может происходить только в начале слога, из-за правил слогообразования в На'ви.

\begin{center}
\begin{tabular}{lll}
\N{syaw} & [ʃaw] & \E{звать} \\
\N{tsìsyì} & [ˈtsɪ.ʃɪ] & \E{шептать} (нпрх.гл.) \\
\end{tabular}
\end{center}

\noindent Данное изменение звуков в Рифовом На'ви никогда не выделяют письменно.

\subsubsection{} \index{Reef Na'vi!glottal stop elision}\label{rn:stop-elision}
Произношение гортанной смычки опускается, если она находится между двумя гласными. Однако, обе гласных все равно выговариваются четко, даже если они оди\-на\-ко\-вые, как в \N{rä'ä} ниже,

\begin{center}
\begin{tabular}{lll}
\N{fra'u} & \N{frau} & \E{всё} \\
\N{Lo'ak} & \N{Loak} & \E{Ло'ак} (имя личное) \\
\N{rä'ä}  & \N{rää} & \E{не}
\end{tabular}
\end{center}

\noindent Такое изменение присутствует в определенной степени в Лесном На'ви. Например, в имени \N{Lo'ak} (\horenref{names-with-oa}).

Добавление аффиксов к гортанной смычке в начале или конце слова создает разрыв в произношении между гласными, который в Рифовый На'ви можно опустить. Например, в Лесном На'ви фраза обозначающая \E{забавную личность}, это \N{tute a'ipu}, а в Рифовом На'ви будет \N{tute aipu}, или наречие \N{nìaw}, которое по-лесному \N{nì'aw}.
\NTeri{14/1/2023}{http://naviteri.org/2023/01/2653/\#comment-45068}

\subsubsection{}
В Рифовом На'ви находящийся в безударном слоге звук \N{ä}, может превратиться в \N{e}.

\begin{center}
\begin{tabular}{lll}
\N{\ACC{nge}yä} & \N{ngeye} & \E{твоё} \\
\N{tä\ACC{txaw}} & \N{tedaw} & \E{возвращаться} (нпрх.гл.) \\
\N{\ACC{kä}}     & \N{kä}  & \E{идти}
\end{tabular}
\end{center}

\Omaticon

\subsection{Разговорный алфавит}
Все названия фонем, за исключением гортанной смычки \N{tìftang}, предоставляют информацию о том как применяется звук в языке. Заглавную ис\-поль\-зу\-ют при записи непривычным путем: \index{alphabet!spoken}

\begin{center}\small
\begin{tabular}{lll}
\N{tìftang} & \N{Ì} & \N{ReR} \\
\N{A}  & \N{KeK}   & \N{'Rr} \\
\N{AW} & \N{KxeKx} & \N{Sä} \\
\N{AY} & \N{LeL}   & \N{TeT} \\
\N{Ä}  & \N{'Ll}   & \N{TxeTx} \\
\N{E}  & \N{MeM}   & \N{Tsä} \\
\N{EW} & \N{NeN}   & \N{U} \\
\N{EY} & \N{NgeNg} & \N{Vä} \\
\N{Fä} & \N{O}     & \N{Wä} \\
\N{Hä} & \N{PeP}   & \N{Yä} \\
\N{I}  & \N{PxePx} & \N{Zä} \\
\end{tabular}
\end{center}

\subsubsection{} Гласные и дифтонги читаются так же как пишутся. Гортанная смычка перед псевдогласными указывает на обязательность наличия согласной перед \N{rr} и \N{ll}(\horenref{l-and-s:pseudo-no-null}).

\subsubsection{} К согласным, которые не могут стоять в конце слога, добавлена гласная \N{ä}, как в \N{Tsä}. А те, которые могут, используют гласную \N{e} и дублируются, \N{PeP}.


\section{Лениция}
\noindent В ходе некоторых грамматических процессов происходит изменение начальной согласной в слове. Такое преобразование называется ``леницией''.  Меняются только восемь согласных.\index{lenition}\label{l-and-s:lenition}
\LanguageLog

\begin{center}
\begin{tabular}{lll}
Согласная & Лениция & Пример \\
\N{px, tx, kx} & \N{p, t, k} & \N{\uwave{tx}ep} -> \N{mì \uwave{t}ep} \\
\N{p, t, k} & \N{f, s, h} & \N{\uwave{k}elku} -> \N{ro \uwave{h}elku} \\
\N{ts} & \N{s} & \N{\uwave{ts}mukan} -> \N{ay\uwave{s}mukan} \\
\N{’} & исчезает & \N{’eylan} -> \N{fpi eylan} \\
\end{tabular}
\end{center}

\noindent Лениция не указывается во второй линии подстрочных описаний, где дана мор\-фо\-ло\-ги\-чес\-кая разбивка текста. (Как в примере \ref{lenition:ex01} ниже). 

\subsection{Гортанная смычка} Гортанная смычка не убирается в ходе лениции, если после неё стоит псевдогласная. (\N{mì 'Rrta}, а не *\N{mì Rrta}).
\index{glottal stop!lenition}\label{l-and-s:lenition:pseudovowel}
\NTeri{3/28/2012}{https://naviteri.org/2012/03/spring-vocabulary-part-1/}

\subsection{Предлоги} В оригинальной грамматике На'ви обозначаются термином ``adposition'' или по-русски ``адлог''. Адлог - собирательное название для предлогов и послелогов. Это одна и та же служебная часть речи, но в качестве предлога ставится перед определяемым словом раздельно, а как послелог - в конце слитно. Для адлогов в дальнейшем будет использоваться более привычный термин ``предлог''. Просто нужно держать в голове то, что в языке На'ви он может не только стоять перед словом, а еще присоединяться в конце. Некоторые предлоги вызывают леницию: \N{fpi}, \N{ìlä}, \N{mì}, \N{nuä}, \N{ro}, \N{sko},
\N{sre} (и образованные от него \N{lisre} с \N{pxisre}), \N{wä}. Однако, при использовании их в качестве суффикса (с конца), её не происходит как у слова к которому они присоединены, так и у идущего следом.
\index{fpi@\textbf{fpi}!lenition}\index{ilä@\textbf{ìlä}!lenition}
\index{miì@\textbf{mì}!lenition}\index{ro@\textbf{ro}!lenition}
\index{sre@\textbf{sre}!lenition}\index{pxisre@\textbf{pxisre}!lenition}
\index{waä@\textbf{wä}!lenition}\index{nuaä@\textbf{nuä}!lenition}
\index{sko@\textbf{sko}!lenition}
\index{lenition!adpositions}\index{adpositions!lenition}
\NTeri{7/7/2010}{https://naviteri.org/2010/07/thoughts-on-ambiguity/}

\subsection{Префиксы числа} Если вызывают леницию, то отмечаются знаком плюс, а не дефисом как в префиксе множественного числа \N{ay+}. \index{lenition!number prefixes}

\subsection{Вопросительная частица} Когда используется как префикс \N{pe+}, то вызывает ле\-ни\-цию (\horenref{morph:pre:pe}). Например, \N{pehem} \E{какое (действие)?}, образованное от \N{kem} \E{действие, дея\-тель\-ность}.

\subsection{Числа}\index{lenition!numbers}
Лениции подвергаются зависимые части при составлении цифр.
(\horenref{numbers:dependent}) Как в \N{vopey} \E{одиннадцать (8 + 3)},
но \N{pxey} это \E{три}.

\subsection{Имена собственные} Лениция так же изменяет имена собственные.

\begin{interlin} \label{lenition:ex01}
\glll Oe kelku si mì Helutral. \\
      oe kelku si mì Kelutral \\
     \I{1ед.ч.} дом делать в Дерево-дом \\
\trans{Я живу в дереве-доме}
\end{interlin}

\index{lenition!proper nouns}\NTeri{10/28/2010}{https://naviteri.org/2010/09/getting-to-know-you-part-2/}

\subsection{Рифовый На'ви} \index{Reef Na'vi!lenition}
Не смотря на то, что эйективные согласные в начале слов на рифовом диалекте показываются как звонкие фрикативы, но и на них распространяется лениция Лесного На'ви. По-этому если \N{txon} \E{ночь} пишется на Рифовом На'ви как \N{don}, то смягченной формой все равно станет \N{ton}.
\LNWiki{8/1/2023}{http://naviteri.org/2023/01/reef-navi-part-1-phonetics-and-phonology/}

\section{Морфофонология}

\subsection{Сокращение гласных} Так как две одинаковых гласные не могут стоять друг за другом, то некоторые грамматические процессы сокращают сдвоенные гласные лишь до одной.\index{vowel!contraction}\label{l-and-s:contract}

\subsubsection{} Морфема для обозначения прилагательных \N{-a-} сокращается, если присоединена к \N{a}, которая уже находится в начале или конце слова, как в
\N{apxa tute}, но не *\N{apxaa tute}.
\index{-a-@\textbf{-a-}!with \textbf{a} in an adjective}
\index{adjective!contraction}

\subsubsection{} Если после использования префиксов двойственного или тройственного числа получается цепочка из \N{e}, как в \N{me} $+$ \N{'eveng} $>$ *\N{meeveng} (учитываем леницию), то две гласные объединяют в одну, \N{meveng}.
\label{l-and-s:phonotactics:nsc} \index{dual!contraction}
\index{trial!contraction}
\LNWiki{20/1/2010}{https://wiki.learnnavi.org/index.php/Canon\%23Extracts_from_various_emails}

\subsubsection{} Если гласные в конце префикса и начале существительного одинаковы, то их сокращают до одной. \N{tsatan} $<$
\N{tsa-} $+$ \N{atan}, \N{fìlva} $<$ \N{fì-} $+$ \N{ìlva}
(\horenref{morph:prenoun:contraction}).\footnote{Гортанная смычка - согласная, так \N{fì'ìheyu} будет от \N{fì-} $+$ \N{'ìheyu}.}
\label{l-and-s:phonotactics:precontract}\index{prenoun!contraction}
\LNWiki{18/5/2011}{https://wiki.learnnavi.org/index.php/Canon/2011/April-December\%23Kawtseng.2C_tsapo_and_prefixes}

\subsubsection{} Сокращения не происходит с маркером неопеределенности \N{-o} или энклитичными предлогами. Парные согласные в таком случае пишутся через дефис \N{fya'o-o}
\E{какой-то путь,} \N{zekwä-äo} \E{под пальцем}.\footnote{При том что в Лесном На'ви, формально, нет долгих гласных, но их эффект возникает в этом случае. Проследите за тем чтобы обе гласные \N{ä} отчетливо произносились в таком слове как \N{zekwä-äo}.}\index{vowel!contraction!inhibited}
% https://wiki.learnnavi.org/index.php/Canon/2010/UltxaAyharyuä#Phonological_Questions

\subsubsection{} В Рифовом На'ви сокращение гласных не применяется.
Слова вида \N{meeveng} или \N{apxaa} останутся без изменений там, где Лесной На'ви сократил бы парные гласные.
\index{Reef Na'vi!vowel contraction inhibited}\label{rn:no-contract}
\NTeri{13/1/2023}{http://naviteri.org/2023/01/2653/}

\subsection{Сокращение псевдо\-гласных} \index{pseudovowel!contraction}
 Форма инфиксов вида действия, \N{\INF{er}}
и \N{\INF{ol}}, может создать ситуацию, при которой псевдогласная будет находиться сразу после своего аналога среди согласных, как тут \N{*p\INF{ol}ll\ACC{txe}}.  Если это произошло в безударном слоге, то исчезает псевдогласная, \N{pol\ACC{txe}}.  А если в ударном, то инфикс, \N{*\ACC{f}\INF{er}\ACC{rr}fen} $>$
\N{\ACC{frr}fen}.  Псевдогласные в односложных словах ведут себя таким образом, будто они безударные, \N{vol} от \N{*v\INF{ol}ll}, и \N{ner}
от \N{*n\INF{er}rr}.
\LNWiki{23/3/2010}{https://wiki.learnnavi.org/index.php/Canon/2010/March-June\%23Misc_Answers}
\NTeri{19/6/2012}{https://naviteri.org/2012/06/spring-vocabulary-part-3/}

\subsection{Эпентеза инфиксов отношения} Если после инфикса положительного от\-но\-ше\-ния \N{\INF{ei}} сразу следует гласная \N{i}, \N{ì} либо псевдогласная, то между ними добавляют 
\N{y}, \N{seiyi} $<$ \N{*s\INF{ei}i}, \N{veykrreiyìn} $<$
\N{*veykrr\INF{ei}ìn}; \N{v\INF{ei}yll} $<$ \N{*veill}.
\label{l-and-s:eiy-epenth}
\NTeri{19/6/2012}{https://naviteri.org/2012/06/spring-vocabulary-part-3/}

\subsubsection{Рифовый На'ви} \index{Reef Na'vi!affect infix epenthesis}
В Рифовом На'ви разделения гласных внутри \N{seii} на \N{seiyi} не происходит.
\NTeri{31/1/2023}{http://naviteri.org/2023/01/2653/}


\subsection{Рифовый На'ви и ассимиляция звонких плозивов} \index{Reef Na'vi!voiced stop assimilation}
Когда присутствует группа эйективных согласных как в \N{atxkxe} \E{земля}, то в Рифовом На'ви происходит регрессивная ассимиляция, когда последующий звук влияет на предыдущий.
Так, плозив в начале второго слога в (*\N{atxge}) вызывает озвучивание стоящего перед ним эйектива, давая в итоге \N{adge}.
Так же как \N{ekxtxu} \E{грубый} в рифовом диалекте станет \N{egdu}.
\LNWiki{8/1/2023}{http://naviteri.org/2023/01/reef-navi-part-1-phonetics-and-phonology/}

\subsection{Назальная ассимиляция} Назальные звуки в конце многих составных выражений и идиом ассимилируются в зависимости от идущего следом слова. Как в примере с \N{lumpe} - вариацией \N{pelun}.  Это не всегда отражают письменно, чтобы этимология слова оставалась более ясной, как в \N{zenke} вместо \N{*zengke}, образованного от \N{zene ke}, или в ряде идиом от глагола \N{tìng} \E{давать}, \N{tìng mikyun} произносится как
\N{tìm mikyun}. \index{nasal assimilation} \label{l-and-s:nasalassim}

\subsection{Сингармонизм} Это уподобление гласных в рамках одного слова по одному или нескольким фонетическим признакам, таким как ряд, подъём (открытость) или огуб\-лен\-ность.
В глагольных инфиксах На'ви есть два случая регрессивного сингармонизма, или умлаута - фонетического явления, заключающегося в изменении артикуляции и тембра гласных: частичной или полной ассимиляции предыдущего гласного последующим.\index{vowel!harmony}

\subsubsection{} Сослагательный инфикс будущего, \N{\INF{iyev}}, чаще всего появляется как \N{\INF{ìyev}}, с поддержкой первой гласной.

\subsubsection{}\label{l-and-s:eng}
Гласная в инфиксе негативного отношения, \N{\INF{äng}}, может поднять свое звучание, если сразу за ней идет гласная \N{i}, становясь \N{\INF{eng}},

\begin{interlin}
\glll Tsap'alute sengi oe. \\
      tsap'alute s\INF{äng}i oe \\
      извинение делать\INF{\I{neg.aff}} \I{1sg} \\
\trans{Я извиняюсь.}
\end{interlin}

\Ultxa{2/10/2010}{https://wiki.learnnavi.org/index.php/Canon/2010/UltxaAyharyu\%C3\%A4\%23.C3.A4ng.2Feng}

\subsection{Элизия} Это отпадение звука в слове или фразе с целью облегчения произношения для говорящего. В быстрой речи На'ви конечная \N{-e} часто опускается, если последующие слова начинаются с гласных.  \Npawl{Kìyevam$\not$e
ult$\not$e Eywa ngahu}. Элизию не отображают пись\-мен\-но.\index{elision}
Конечную \N{-e} не опускают, если она находится в односложном слове (\N{ke, sre}) или под ударением (\N{tuté}).
\LNForum{25/10/2022}{https://forum.learnnavi.org/language-updates/a-collection-of-questions-answered/}

\subsubsection{} Гласная \N{ì} в \N{mì}, \N{sì} и префиксе наречия \N{nì-} не произносится перед префиксом множественного числа \N{ay+}. Это не отображается письменно.  Так, \N{nìayfo} \E{подобно им} вы\-го\-ва\-ри\-ва\-ют как
\N{nayfo}. \label{l-and-s:elision-i}
\index{miì@\textbf{mì}!elision with plural}
\index{siì@\textbf{sì}!elision with plural}
\index{niì-@\textbf{nì-}!elision with plural}
\NTeri{1/7/2010}{https://naviteri.org/2010/07/thoughts-on-ambiguity/}

\subsubsection{} Гласная в \N{nì-} обычно опускается перед ударной \N{e}, как здесь \N{nì-} + \N{etrìp} > \N{netrìp}. Если \N{e}
безударная, то нередко, хоть и не всегда, не произносится, \N{nì-} +
\N{eyawr} > \N{nìyawr}. Единственное исключение: \N{nìean} вместо ожидаемого \N{*nean}.
\index{niì-@\textbf{nì-}!elision before e}
\LNForum{9/8/2017}{https://forum.learnnavi.org/language-updates/if-ni-will-attached-at-e/}

\subsection{Другие фонетические процессы}

\subsubsection{Имена с -o'a-} \label{names-with-oa}
В разговорной речи имена, содержащие последовательность \N{o'a} могут избавляться от произношения гортанной смычки. Как в \N{Mo'at} \textasciitilde{} \N{Moat} 
и \N{Lo'ak} \textasciitilde \N{Loak}.
\NTeri{1/3/2017}{https://naviteri.org/2017/02/ayioang-amip-si-ayu-alahe-new-animals-and-other-things/\#comment-26401}


\section{Орфографические условности}
\noindent В целом, На'ви следует привычным правилам орфографии, пунктуации и использования заглавных, но есть несколько отличий.

\subsection{Имена собственные} При получении лексических префиксов
(\horenref{lingop:affixes})имена соб\-ствен\-ные сохраняют заглавные буквы, как в \N{lì'fya le\uwave{Na'vi}}.

\subsection{Цитирование} Прямые цитаты в На'ви не выделяют кавычками.  Вместо этого применяются частицы для цитирования
\N{san\dots sìk} (see \horenref{syn:direct-quote}).
\index{quotation!punctuation}

\subsection{Этимологическое написание} В добавок к редким случаям записи назальных звуков для отражения этимологии (\horenref{l-and-s:nasalassim}),
есть еще несколько грамматических процессов, которые на письме отражают больше грамматику, чем произношение.

\subsubsection{} Корень местоимения первого лица \N{oe}, хоть и
произносится как \N{we}, но сохраняет исходное написание при получении суффикса 
(\horenref{morph:pron:oe-we}).

\subsubsection{} Перед словами, начинающимися на \N{y}, префикс множественного числа
\N{ay+} ос\-тав\-ля\-ют как есть, \N{ayyerik}.
\LNWiki{18/4/2010}{https://wiki.learnnavi.org/index.php/Canon/2010/March-June\%23ay.2Byerik}

\subsection{Дефис в определительных фразах} В определительных фразах на основе су\-ще\-стви\-тель\-ных, которые передают признак, качество или свойство предмета, элементы записываются через дефис.

\subsubsection{} Использующая \N{na} \E{как, подобно} определительная фраза для описания цвета
пи\-шет\-ся через дефис, \N{fìsyulang aean-na-ta'leng} \E{этот голубой-как-кожа
цветок} (\horenref{syn:attr:na}).

\subsubsection{} Причастия образованные от  \N{si}-глаголов так же записываются через дефис, \N{srung-susia tute} \E{помогающая личность}
(\horenref{syn:participle:si-const}).

