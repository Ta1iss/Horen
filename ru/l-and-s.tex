\nchapter{Буквы и Звуки}


\section{Фонетика}
\noindent В языке На'ви 20 согласных звуков, 7 гласных
и два слоговых резонирующих звука, которые Фроммер назвал ``псевдогласными.''
\LanguageLog

\subsection{Согласные}
Согласные На'ви перечислены ниже.  В круглых скобках указано произношение в Рифовом На'ви некоторых фонем и заимствований, которые не имеют отдельного написания. В квадратных скобках указаны транскрипции принятые в международном звуковом алфавите. Примеры произношения можно послушать тут: \href{https://www.ipachart.com/}{IPA chart}

\begin{center}
\begin{tabular}{llllll}
 & Губные & Альвеолярные & Нёбные & Заднеязычные & Глоттальные \\
Взрывные &	\N{px} [p'] & \N{tx} [t'] & & \N{kx} [k'] \\
Глухие плозивы & \N{p} [p] & \N{t} [t] & & \N{k} [k] & \N{’} [ʔ] \\
Звонкие плозивы    &  ([b])    & ([d])    &  & ([g]) \\
Аффрикаты &             & \N{ts}  [ts] & ([tʃ]) \\
Глухие фрикативы & \N{f} [f] & \N{s} [s] & ([ʃ]) & & \N{h} [h] \\
Звонкие фрикативы & \N{v} [v] & \N{z} [z] \\
Назальные &         \N{m} [m] & \N{n} [n] & & \N{ng} [ŋ] \\
Текучие &         &  \N{r} [ɾ], \N{l} [l] \\
Скользящие &       \N{w} [w] & &  \N{y} [j] \\
\end{tabular}
\end{center}

\subsubsection{} Глухие плозивы произносятся без придыхания в начале и середине слова.  Они не выражены в конце слова,
а также в конце слога, если за ними стоит согласная
(как в \N{txe\uwave{p}mì} и \N{'o\uwave{k}trr}).  Однако, внутри фразы последний плозив, находящийся перед гласной, в естественной речи будет произнесен так как если бы слова сливались воедино, \N{oel se\uwave{t o}mum}.
Глухие плозивы более заметны в больших фразах, как в \N{oel
omum se\uwave{t}.}
\NTeri{23/12/2020}{https://naviteri.org/2020/12/mrra-tipangkxotsyip-five-little-discussions/}

\subsubsection{} \N{r} аналогичен русскому ``Р''. \N{l} четкий вначале, как в слове ``лист,'' но не как ``гулкая-Л'' в английском слове
``call, звонить''.'

\subsubsection{} Фроммер разработал научную орфографию, в которой два диграфа писались одной буквой, \N{c} для \N{ts} и
\N{g} для \N{ng}.  Актерам проще было воспринимать двухбуквенные диграфы, да и сам Фроммер использовал их в интервью и электронных письмах. Однобуквенное обозначение появлялось лишь в ранних сообщениях Фроммера.  \label{l-and-s:cg}

\subsubsection{} Окончания закрытых слогов с плозивом в конце могут вводить в заблуждение своим произношением и похожим на взрывные согласные, но это не так:
\N{tsap'alute} верно, а *\N{tsapxalute} нет.
\LNWiki{21/12/2009}{https://wiki.learnnavi.org/index.php/Canon\%23Extracts_from_various_emails}

\subsection{Гласные}
Гласная в круглых скобках является отдельной фонемой только в Рифовом На'ви.  Остальные используют оба диалекта.

\begin{center}
\begin{tabular}{ccccc}
\N{i} [i], \N{ì} [{\footnotesize I}]  & & & & (\N{ù} [ʊ]), \N{u} [u],[ʊ] \\
 & \N{e} [ɛ] & & & \N{o} [o] \\
 & & \N{ä} [æ] &  \N{a} [a] \\
\end{tabular}
\end{center}

\subsubsection{} В рифовром На'ви \N{u} всегда звучит как [u] - аналогично русской ``у'',
а \N{ù} как звук [ʊ], представляющий что-то среднее между ``о'' и ``у''. \index{Reef Na'vi!ù@\textbf{ù}}
\LNWiki{8/1/2023}{http://naviteri.org/2023/01/reef-navi-part-1-phonetics-and-phonology/}

\subsubsection{} Лесной На'ви объединяет гласные \N{u} и \N{ù}, 
по-этому они всегда звучат как [u] в открытых слогах. В закрытых допускаются оба произношения: [u]
или [ʊ].  \N{Lu} всегда звучит как [lu],
а для \N{pum} возможны оба варианта: [pum] и [pʊm].
\LNWiki{20/5/2010}{https://wiki.learnnavi.org/index.php/Canon/2010/March-June\%23The_Dual_sounds_of_.22u.22}

% 2023jan10 - don't do this for now.
%In this grammar words that are known to have \N{ù} in them in Reef
%Na'vi are spelled as such, such as \N{pùm}.  When using Forest Na'vi,
%simply treat these words as though spelled \N{pum}.

\subsubsection{} Дифтонги это \N{aw}, \N{ay}, \N{ew} и \N{ey}.
Только в дифтонгах \N{w} или \N{y} могут находиться в конце слога (\N{new}) или перед конечной согласной (\N{hawng}).  Формы слогов \N{*niw} или \N{*hoyng} иметь места не могут.

\subsection{Псевдогласные} Псевдогласная \N{rr} это слоговый растянутый звук [r̩ː], а \N{ll} слоговый растянутый звук [l̩ː]. Псевдогласная \N{rr} произносится как растянутая русская ``Р'', подобно «рычанию», а \N{ll} как растянутая русская ``Л''.

\subsection{Строение слога}
 На'ви имеет строгое, но понятное слоговое строение.

\begin{itemize*}
  \item В слоге допускается отсуствие начальной согласной (т.е., может начинаться с гласной).
  \item В слоге допускается отсутствие конечной согласной (т.е., может кончаться на гласной).
  \item Слог может начинаться с любой согласной.
  \item Группы согласных \N{f s ts} $+$ \N{p, t, k, px, tx, kx,
    m, n, ng, r, l, w, y} могут начинать слог (например, \N{tslam}, \N{ftu}).
  \item \N{P t k px tx kx ' m n l r ng} могут появляться в конце слога.
  \item \N{Ts f s h v z w y}  \textit{НЕ} могут появляться в конце слога.
  \item В конце слога не бывает групп согласных.
  \item \label{l-and-s:pseudo-no-null} Слог с псевдогласной должен начинаться с согласной или группы согласных и не должен иметь согласной в конце; это играет свою роль в смягчении (лениции)
    (\horenref{l-and-s:lenition:pseudovowel}) и склонении существительных
    (\horenref{morph:decl:pseudovowel}).
\end{itemize*}

\newpage
\noindent Наглядный пример основных правил слогообразования:

\begin{center}\footnotesize
\begin{tikzpicture}[every path/.style={>=latex},every node/.style={draw,rectangle}]
\node[rounded corners,fill=gray!20,align=center] (sos) at (-4,1) {начало\\слога};
\node[align=center] (zero-onset) at (-0.5,2.5) { нет начальной\\согласной };
\node[align=center,font=\bfseries] (onset) at (-0.25,1) {p t k px tx kx '\\ m n ng r l w y\\f v s z ts h};
\node[font=\bfseries] (clust-c1) at (-2.5,-0.5) { f s ts };
\node[align=center,font=\bfseries] (clust-c2) at (0,-0.5) {p t k px tx kx\\m n ng r l w y};
\node[align=center] (vowel) at (3.5,2) { гласная или\\дифтонг };
\node (pseudovowel) at (3.5,0.5) { псевдогласная };
\node[align=center,font=\bfseries] (final) at (6.25,2.5) {p t k px tx kx ’\\m n l r ng };
\node[align=center] (zero-coda) at (6.25,1) { без конечной\\согласной };
\node[rounded corners,fill=gray!20,align=center] (eos) at (8.75,1) { конец\\слога };

\draw[->] (sos) edge (clust-c1);
  \draw[->] (sos) edge (onset);
  \draw[->] (sos) edge (zero-onset);
\draw[->] (clust-c1) edge (clust-c2);
\draw[->] (zero-onset) edge (vowel);
\draw[->] (clust-c2) edge (vowel);
  \draw[->] (clust-c2) edge (pseudovowel);
\draw[->] (onset) edge (vowel);
  \draw[->] (onset) edge (pseudovowel);
\draw[->] (vowel) edge (final);
\draw[->] (final) edge (eos);
\draw[->] (vowel) edge (zero-coda);
\draw[->] (pseudovowel) edge (zero-coda);
\draw[->] (zero-coda) edge (eos);
\end{tikzpicture}
\end{center}

\subsubsection{} Из-за того, что у слога могут отсуствовать начальные или конечные согласные, то возможно появление нескольких, стоящих друг за другом, гласных.
Тогда гласную считают отдельным слогом, \N{muiä} [mu.i.æ], \N{ioang}
[i.o.aŋ].

\subsubsection{} Принято последовательность ГСГ разбивать на слоги в виде Г.СГ, а не ГС.Г, так \N{tsenge} будет [tsɛ.ŋɛ], а не *[tsɛŋ.ɛ].
Звукоподражание может отменять это правило, как в \N{kxang\-ang\-ang} [k'aŋ.aŋ.aŋ],
где нужно достичь эффекта эха.

\subsubsection{} В Лесном На'ви нет долгих гласных. Это означает то, что одинаковые гласные не могут находиться друг за другом (смотри тут
\horenref{l-and-s:contract}).  В Рифовом На'Ви в некоторых случаях могут появляться долгие гласные из-за опускания произношения глоттального звука гортанной смычки (\horenref{rn:stop-elision}) и отсутствия сокращения гласных
(\horenref{rn:no-contract}). 

\subsubsection{} Сдвоенные согласные не встречаются в корнях слов, но могут появляться на границе морфем, например, производных
\N{tsukkäteng} $<$ \N{tsuk-} $+$ \N{käteng} или \href{https://ru.wikipedia.org/wiki/%D0%AD%D0%BD%D0%BA%D0%BB%D0%B8%D1%82%D0%B8%D0%BA%D0%B0}
{энклитиках}, 
когда группа слов произносится с одним ударением, воспринимаясь на слух как единое целое 
\N{Mo'atta} $<$ \N{Mo'at} $+$ \N{ta} (\horenref{l-and-s:stress:enclisis}).
% https://naviteri.org/2011/03/“receptive-ability”-and-hesitation/comment-page-1/#comment-604

\subsubsection{} Междометия могут выходить за рамки этих правил. Обычное явление для многих человеческих языков. Например, \N{oìsss} - звук гнева или 
\N{saa} - угрожающий крик.


\subsection{Ударение}
Каждого слово в На'ви есть ударение, причем не всегда предсказуемое.  Бывает так, что ударение это единственное различие между полностью одинаковыми словами, как, например, в \N{\ACC{tu}te} \E{личность}
и \N{tu\ACC{te}} \E{женщина}.

\subsubsection{} Конкретно для слова \E{женщина} ударение принято указывать отдельно, \N{tuté}.
\index{tuté@\textbf{tuté}}

\subsubsection{} В ходе словообразования акцент ударения может меняться
(\horenref{lingop:prefix:ke}, \horenref{lingop:suffix:gender}).

\subsubsection{} Все предлоги, а также некоторые союзы и частицы могут быть энклитичны.  Они теряют свое ударение, когда становятся частью слова, к которому присоединяются, и пишутся соответствующе, \N{\ACC{tsa}ne} ($<$ \N{tsaw} $+$ \N{ne}),
\N{ho\ACC{ren}\-ti\-sì} ($<$ \N{ho\ACC{ren}ti} $+$ \N{sì}).
\label{l-and-s:stress:enclisis}\index{enclitics}

\subsubsection{} В составных существительных, пишущихся как одно слово, каждая из частей может сохранять свое ударение, например, \N{ti\ACC{re}a\ACC{fya}'o} \E{путь духа}.
\index{compound word!accent}

%\subsubsection{} Word stress is a property of stem words.  No matter
%how many affixes a root word takes, no secondary accents develop.

\subsection{Рифовый На'ви} Отличия рифового диалекта от лесного присутствуют как в гласных так и в согласных звуках. \index{Reef Na'vi}
\NTeri{8/1/2023}{http://naviteri.org/2023/01/reef-navi-part-1-phonetics-and-phonology/}

\subsubsection{} Взрывные согласные в начале слога, а, следовательно, и в начале слова произносятся как звонкие плозивы рифового диалекта.  Так, \N{px tx kx} → [b d g].

\begin{center}
\begin{tabular}{lll}
\N{txon}    & [don] & \E{ночь} \\
\N{hol\ACC{pxay}} & [hol.ˈbaj] & \E{число} \\
\N{kxitx}   & [git'] & \E{смерть} \\
\N{skxawng} & [sk'awŋ] (as in Forest Na'vi) & \E{дурак}
\end{tabular}
\end{center}

\noindent Вообще, данное отличие не отражено письменно. Так что текст в рифовом диалекте будет писаться  \E{ночь} как \N{txon}, но будет подразумеваться что читающий произнесет это как \N{don}.  Если хочется подчеркнуть использование рифового диалекта, то можно написать \N{don}.

Запись слова \N{tìkankxan} на рифовом диалекте может привести к двусмысленности при чтении, \N{*tìkangan}.  Чтобы исключить толкование сочетания \N{...ng...} как [ŋ] вместо [ŋg], используется интерпункт (предпочтительнее) либо дефис: \N{tìkan·gan}
или \N{tìkan-gan}.
\Omaticon{} \NTeri{15/1/2023}{http://naviteri.org/2023/01/2653/\#comment-45092}

\paragraph{} При присоединении к слову, оканчивающемуся на взрывную согласную, суффикса, который начинается с гласной, происходит изменениее произношения этой взрывной в соответствии с новой слоговой структурой. Например, транскрипция слова \N{'awkx} с взрывной согласной это
[ʔawk'], но \N{'awkxit} будет читаться как [ˈʔaw.git] голосом.
\NTeri{13/1/2023}{http://naviteri.org/2023/01/2653/}



\subsubsection{}  \N{s} и \N{ts} смягчаются, если стоят вместе с \N{y}.  Так что, \N{sy} произносится как [ʃ] и \N{tsy} как [tʃ]. [ʃ] звучит как русская ``Щ''.
В связи с тем как образуются слоги в На'ви, такое может происходить только в начале слога.

\begin{center}
\begin{tabular}{lll}
\N{syaw} & [ʃaw] & \E{звать} \\
\N{tsìsyì} & [ˈtsɪ.ʃɪ] & \E{шептать} (нпрх.гл.) \\
\end{tabular}
\end{center}

\noindent This Reef Na'vi sound change is never indicated in spelling.

\subsubsection{} \index{Reef Na'vi!glottal stop elision}\label{rn:stop-elision}
When between two vowels, the glottal stop is normally dropped in Reef
Na'vi.  Both vowels are still distinctly enunciated, even if they are
identical, as in \N{rä'ä} below,

\begin{center}
\begin{tabular}{lll}
\N{fra'u} & \N{frau} & \E{everything} \\
\N{Lo'ak} & \N{Loak} & \E{Lo'ak} (personal name) \\
\N{rä'ä}  & \N{rää} & \E{don't}
\end{tabular}
\end{center}

\noindent This change is also present to a certain degree in Forest
Na'vi, such as with the name \N{Lo'ak} (\horenref{names-with-oa}).

While the glottal stop is retained at the start and end of words, when
affixes are added which then place the stop between vowels the stop
may be dropped.  For example, the phrase for \E{humorous person}
is \N{tute a'ipu} in Forest Na'vi, but \N{tute aipu} in Reef Na'vi, or
the adverb \N{nìaw} rather than Forest Na'vi \N{nì'aw}.
\NTeri{14/1/2023}{http://naviteri.org/2023/01/2653/\#comment-45068}

\subsubsection{}
In unstressed syllables \N{ä} may become \N{e} in Reef Na'vi.

\begin{center}
\begin{tabular}{lll}
\N{\ACC{nge}yä} & \N{ngeye} & \E{your} \\
\N{tä\ACC{txaw}} & \N{tedaw} & \E{return} (v.in.) \\
\N{\ACC{kä}}     & \N{kä}  & \E{go}
\end{tabular}
\end{center}

\Omaticon

\subsection{Spoken Alphabet}
Except for \N{tìftang}, the glottal stop, the names of the phonemes
encode information about how the sound is used.  They also have
unusual capitalization when written out: \index{alphabet!spoken}

\begin{center}\small
\begin{tabular}{lll}
\N{tìftang} & \N{Ì} & \N{ReR} \\
\N{A}  & \N{KeK}   & \N{'Rr} \\
\N{AW} & \N{KxeKx} & \N{Sä} \\
\N{AY} & \N{LeL}   & \N{TeT} \\
\N{Ä}  & \N{'Ll}   & \N{TxeTx} \\
\N{E}  & \N{MeM}   & \N{Tsä} \\
\N{EW} & \N{NeN}   & \N{U} \\
\N{EY} & \N{NgeNg} & \N{Vä} \\
\N{Fä} & \N{O}     & \N{Wä} \\
\N{Hä} & \N{PeP}   & \N{Yä} \\
\N{I}  & \N{PxePx} & \N{Zä} \\
\end{tabular}
\end{center}

\subsubsection{} Vowels and diphthongs are simply pronounced and
spelled as themselves.  The pseudo\-vowels take a leading glottal stop,
since they require a consonant onset (\horenref{l-and-s:pseudo-no-null}).

\subsubsection{} The name for consonants which cannot end a syllable
are formed by adding \N{ä}, as in \N{Tsä}.  Those which can end a
syllable use the vowel \N{e} and repeat the consonant at the end of
the name, \N{PeP}.


\section{Смягчение}
\noindent Certain grammatical processes cause changes in the first
consonant of a word.  This change is called ``lenition.''  Only eight
consonants undergo lenition.\index{lenition}\label{l-and-s:lenition}
\LanguageLog

\begin{center}
\begin{tabular}{lll}
Consonant & Lenition & Example \\
\N{px, tx, kx} & \N{p, t, k} & \N{\uwave{tx}ep} but \N{mì \uwave{t}ep} \\
\N{p, t, k} & \N{f, s, h} & \N{\uwave{k}elku} but \N{ro \uwave{h}elku} \\
\N{ts} & \N{s} & \N{\uwave{ts}mukan} but \N{ay\uwave{s}mukan} \\
\N{’} & disappears & \N{’eylan} but \N{fpi eylan} \\
\end{tabular}
\end{center}

\noindent In the interlinears, lenition is removed in the
morphological breakdown on the second line (as in
ex.\ref{lenition:ex01} below).

\subsection{Glottal Stop} The glottal stop is not lenited when it is
followed by a pseudovowel (\N{mì 'Rrta} not *\N{mì Rrta}).
\index{glottal stop!lenition}\label{l-and-s:lenition:pseudovowel}
\NTeri{3/28/2012}{https://naviteri.org/2012/03/spring-vocabulary-part-1/}

\subsection{Adpositions} A few adpositions cause lenition when they
precede a word: \N{fpi}, \N{ìlä}, \N{mì}, \N{nuä}, \N{ro}, \N{sko},
\N{sre} (and derived \N{lisre} and \N{pxisre}), \N{wä}. When suffixed
they do not cause lenition in either the word they are attached to or
to the following word.
\index{fpi@\textbf{fpi}!lenition}\index{ilä@\textbf{ìlä}!lenition}
\index{miì@\textbf{mì}!lenition}\index{ro@\textbf{ro}!lenition}
\index{sre@\textbf{sre}!lenition}\index{pxisre@\textbf{pxisre}!lenition}
\index{waä@\textbf{wä}!lenition}\index{nuaä@\textbf{nuä}!lenition}
\index{sko@\textbf{sko}!lenition}
\index{lenition!adpositions}\index{adpositions!lenition}
\NTeri{7/7/2010}{https://naviteri.org/2010/07/thoughts-on-ambiguity/}

\subsection{Number Prefixes} Prefixes which cause lenition are
indicated with a plus sign, rather than the usual dash, as in \N{ay+},
the leniting plural prefix. \index{lenition!number prefixes}

\subsection{Question Prenoun} When used as a prefix, the question
prenoun \N{pe+} causes lenition (\horenref{morph:pre:pe}), as
in \N{pehem} \E{what (action)?} from \N{kem} \E{action, activity}.

\subsection{Numbers}\index{lenition!numbers}
Suffixed, dependent forms of the numbers are lenited
(\horenref{numbers:dependent}), as in \N{vopey} \E{eleven (8 + 3)},
but \N{pxey} \E{three}.

\subsection{Proper Nouns} Proper nouns still undergo lenition.

\begin{interlin} \label{lenition:ex01}
\glll Oe kelku si mì Helutral. \\
      oe kelku si mì Kelutral \\
     \I{1sg} home do in hometree \\
\trans{I live in hometree}
\end{interlin}

\index{lenition!proper nouns}\NTeri{10/28/2010}{https://naviteri.org/2010/09/getting-to-know-you-part-2/}

\subsection{Reef Na'vi} \index{Reef Na'vi!lenition}
Although the ejectives in Reef Na'vi surface as voiced stops at the
start of a word, the rules of lenition still apply as in Forest Na'vi.
That is, though \N{txon} \E{night} is pronounced as \N{don} in Reef
Na'vi, the lenited form is still \N{ton}.
\LNWiki{8/1/2023}{http://naviteri.org/2023/01/reef-navi-part-1-phonetics-and-phonology/}

\section{Morphophonology}

\subsection{Vowel Contraction} Since identical vowels may not occur
next to each other, a few grammatical processes involve a doubled
vowel reducing to just one.\index{vowel!contraction}\label{l-and-s:contract}

\subsubsection{} The adjective morpheme \N{-a-} disappears when
attached to an \N{a} at the start or end of an adjective, as in
\N{apxa tute} not *\N{apxaa tute}.
\index{-a-@\textbf{-a-}!with \textbf{a} in an adjective}
\index{adjective!contraction}

\subsubsection{} When the dual and trial prefixes leave a sequence of
two \N{e}s, as in \N{me} $+$ \N{'eveng} $>$ *\N{meeveng} (note
lenition), the two vowels contract to just one, \N{meveng}.
\label{l-and-s:phonotactics:nsc} \index{dual!contraction}
\index{trial!contraction}
\LNWiki{20/1/2010}{https://wiki.learnnavi.org/index.php/Canon\%23Extracts_from_various_emails}

\subsubsection{} When the prenoun prefixes end in the same vowel the
following word starts with, they reduce to one, as in \N{tsatan} $<$
\N{tsa-} $+$ \N{atan}, \N{fìlva} $<$ \N{fì-} $+$ \N{ìlva}
(\horenref{morph:prenoun:contraction}).\footnote{The glottal stop is a
consonant, so \N{fì'ìheyu} from \N{fì-} $+$ \N{'ìheyu}.}
\label{l-and-s:phonotactics:precontract}\index{prenoun!contraction}
\LNWiki{18/5/2011}{https://wiki.learnnavi.org/index.php/Canon/2011/April-December\%23Kawtseng.2C_tsapo_and_prefixes}

\subsubsection{} Contraction does not occur for indefinite \N{-o} or
enclitic adpositions.  When two identical vowels occur next to each
other, they are written with a hyphen between them, \N{fya'o-o}
\E{some way,} \N{zekwä-äo} \E{under a finger}.\footnote{Though Forest
Na'vi does not technically have long vowels, the effect of long vowels
occurs in this situation.  Take care to pronounce both \N{ä} in a word
such as \N{zekwä-äo}.}\index{vowel!contraction!inhibited}
% https://wiki.learnnavi.org/index.php/Canon/2010/UltxaAyharyuä#Phonological_Questions

\subsubsection{} In Reef Na'vi vowel contraction is not applied.
Forms such as \N{meeveng} or \N{apxaa} remain, where Forest Na'vi
would simplify the doubled vowels.
\index{Reef Na'vi!vowel contraction inhibited}\label{rn:no-contract}
\NTeri{13/1/2023}{http://naviteri.org/2023/01/2653/}

\subsection{Pseudovowel Contraction} \index{pseudovowel!contraction}
 Due to the shape of the aspect infixes, \N{\INF{er}}
and \N{\INF{ol}}, it is possible for the pseudovowels to occur
immediately after their consonantal counterpart, as
in \N{*p\INF{ol}ll\ACC{txe}}.  When this happens in an unstressed
syllable, the pseudovowel disappears, \N{pol\ACC{txe}}.  In a stressed
syllable, the infix disappears, \N{*\ACC{f}\INF{er}\ACC{rr}fen} $>$
\N{\ACC{frr}fen}.  Pseudovowels in monosyllables behave as though
unaccented, \N{vol} from \N{*v\INF{ol}ll}, and \N{ner}
from \N{*n\INF{er}rr}.
\LNWiki{23/3/2010}{https://wiki.learnnavi.org/index.php/Canon/2010/March-June\%23Misc_Answers}
\NTeri{19/6/2012}{https://naviteri.org/2012/06/spring-vocabulary-part-3/}

\subsection{Affect Infix Epenthesis} When the positive affect infix
\N{\INF{ei}} is followed by the vowel \N{i}, \N{ì} or a pseudovowel, a
\N{y} is inserted, \N{seiyi} $<$ \N{*s\INF{ei}i}, \N{veykrreiyìn} $<$
\N{*veykrr\INF{ei}ìn}; \N{v\INF{ei}yll} $<$ \N{*veill}.
\label{l-and-s:eiy-epenth}
\NTeri{19/6/2012}{https://naviteri.org/2012/06/spring-vocabulary-part-3/}

\subsubsection{Reef Na'vi} \index{Reef Na'vi!affect infix epenthesis}
The dissimilation of \N{seii} into \N{seiyi} does not occur in Reef
Na'vi.
\NTeri{31/1/2023}{http://naviteri.org/2023/01/2653/}


\subsection{Reef Na'vi Voiced Stop Assimilation} \index{Reef Na'vi!voiced stop assimilation}
In Reef Na'vi, when a word has a cluster of ejectives, such as
in \N{atxkxe} \E{land}, regressive voicing assimilation takes place.
That is, the onset voiced stop (*\N{atxge}) causes the voicing of the
previous coda ejective as well, giving, \N{adge}.
Similarly, \N{ekxtxu} \E{rough} is \N{egdu} in Reef Na'vi.
\LNWiki{8/1/2023}{http://naviteri.org/2023/01/reef-navi-part-1-phonetics-and-phonology/}

\subsection{Nasal Assimilation} In many compounds as well as in some
idioms, final nasals assimilate to the position of the following
word, as in \N{lumpe} as a variant of \N{pelun}.  Such assimilation is
not always written, which may make the etymology of a word clearer, as
in \N{zenke} instead of \N{*zengke}, from \N{zene ke}, or in the several
idioms with the verb \N{tìng} \E{give}, \N{tìng mikyun} being pronounced
\N{tìm mikyun}. \index{nasal assimilation} \label{l-and-s:nasalassim}

\subsection{Vowel Harmony} Na'vi has two instances of optional
regressive vowel harmony in verb infixes.\index{vowel!harmony}

\subsubsection{} The subjunctive future infix, \N{\INF{iyev}}, most
frequently appears as \N{\INF{ìyev}}, with backing of the first vowel.

\subsubsection{}\label{l-and-s:eng}
The vowel of the negative attitude infix, \N{\INF{äng}}, may be raised
if it is immedately followed by the vowel \N{i}, becoming \N{\INF{eng}},

\begin{interlin}
\glll Tsap'alute sengi oe. \\
      tsap'alute s\INF{äng}i oe \\
      apology do\INF{\I{neg.aff}} \I{1sg} \\
\trans{I apologize.}
\end{interlin}

\Ultxa{2/10/2010}{https://wiki.learnnavi.org/index.php/Canon/2010/UltxaAyharyu\%C3\%A4\%23.C3.A4ng.2Feng}

\subsection{Elision} In rapid speech final \N{-e} is frequently elided
when the following word starts in a vowel.  \Npawl{Kìyevam$\not$e
ult$\not$e Eywa ngahu}.  This is not indicated in writing.\index{elision}
When the final \N{-e} is in a monosyllable (\N{ke, sre}), or when it is
stressed (\N{tuté}), it is not elided.
\LNForum{25/10/2022}{https://forum.learnnavi.org/language-updates/a-collection-of-questions-answered/}

\subsubsection{} The vowel \N{ì} in \N{mì}, \N{sì} and the adverb
prefix \N{nì-} drops before the plural prefix \N{ay+}, though there is
no change in writing.  So, \N{nìayfo} \E{like them} is pronounced as
\N{nayfo}. \label{l-and-s:elision-i}
\index{miì@\textbf{mì}!elision with plural}
\index{siì@\textbf{sì}!elision with plural}
\index{niì-@\textbf{nì-}!elision with plural}
\NTeri{1/7/2010}{https://naviteri.org/2010/07/thoughts-on-ambiguity/}

\subsubsection{} The vowel in \N{nì-} will usually elide before a
stressed \N{e}, as in \N{nì-} + \N{etrìp} > \N{netrìp}. If the \N{e}
is unstressed, it will usually, though not always, elide, \N{nì-} +
\N{eyawr} > \N{nìyawr}. One exception: \N{nìean} instead of the
expected \N{*nean}.
\index{niì-@\textbf{nì-}!elision before e}
\LNForum{9/8/2017}{https://forum.learnnavi.org/language-updates/if-ni-will-attached-at-e/}

\subsection{Other Phonetic Processes}

\subsubsection{Names with -o'a-} \label{names-with-oa}
In colloquial speech names containing the sequence \N{o'a} may
eliminate the glottal stop, such
as \N{Mo'at} \textasciitilde{} \N{Moat} 
and \N{Lo'ak} \textasciitilde \N{Loak}.
\NTeri{1/3/2017}{https://naviteri.org/2017/02/ayioang-amip-si-ayu-alahe-new-animals-and-other-things/\#comment-26401}


\section{Orthographic Conventions}
\noindent Na'vi in general follows the spelling, punctuation and
capitalization habits of English, but there are a few differences.

\subsection{Proper Names} When taking lexical prefixes
(\horenref{lingop:affixes}), proper names retain their original
capitalization, as in \N{lì'fya le\uwave{Na'vi}}.

\subsection{Quotation} Direct quotes are not punctuated with quotation
marks in Na'vi.  Instead it relies on the quotation particles
\N{san\dots sìk} (see \horenref{syn:direct-quote}).
\index{quotation!punctuation}

\subsection{Etymological Spelling} In addition to the occasional
spelling of nasals to reflect etymo\-logy (\horenref{l-and-s:nasalassim}),
there are a few grammatical processes which result in spelling that
reflects the grammar more than the pronunciation.

\subsubsection{} The first person pronoun root \N{oe}, though
pronounced \N{we} when taking a suffix, retains the original spelling
(\horenref{morph:pron:oe-we}).

\subsubsection{} Before words starting with \N{y} the plural prefix
\N{ay+} is unchanged, \N{ayyerik}.
\LNWiki{18/4/2010}{https://wiki.learnnavi.org/index.php/Canon/2010/March-June\%23ay.2Byerik}

\subsection{Attributive Phrase Hyphenation} Certain short attributive
phrases are written with hy\-phens joining the elements.

\subsubsection{} Attributive phrases of color using \N{na} \E{like}
are hyphenated, \N{fìsyulang aean-na-ta'leng} \E{this skin-blue
flower} (\horenref{syn:attr:na}).

\subsubsection{} Participles of \N{si} construction verbs are also
hyphenated, \N{srung-susia tute} \E{a helping person}
(\horenref{syn:participle:si-const}).

