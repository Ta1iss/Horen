\nchapter{Буквы и Звуки}


\section{Фонетика}
\noindent В языке На'ви 20 согласных звуков, 7 гласных
и два слоговых резонирующих звука, которые Фроммер назвал ``псевдогласными.''
\LanguageLog

\subsection{Согласные}
Согласные На'ви перечислены ниже.  В круглых скобках указано произношение в Рифовом На'ви некоторых фонем и заимствований, которые не имеют отдельного написания. В квадратных скобках указаны транскрипции принятые в международном звуковом алфавите. Примеры произношения можно послушать тут: \href{https://www.ipachart.com/}{IPA chart}

\begin{center}
\begin{tabular}{llllll}
 & Губные & Альвеолярные & Нёбные & Заднеязычные & Глоттальные \\
Взрывные &	\N{px} [p'] & \N{tx} [t'] & & \N{kx} [k'] \\
Глухие плозивы & \N{p} [p] & \N{t} [t] & & \N{k} [k] & \N{’} [ʔ] \\
Звонкие плозивы    &  ([b])    & ([d])    &  & ([g]) \\
Аффрикаты &             & \N{ts}  [ts] & ([tʃ]) \\
Глухие фрикативы & \N{f} [f] & \N{s} [s] & ([ʃ]) & & \N{h} [h] \\
Звонкие фрикативы & \N{v} [v] & \N{z} [z] \\
Назальные &         \N{m} [m] & \N{n} [n] & & \N{ng} [ŋ] \\
Текучие &         &  \N{r} [ɾ], \N{l} [l] \\
Скользящие &       \N{w} [w] & &  \N{y} [j] \\
\end{tabular}
\end{center}

\subsubsection{} Глухие плозивы произносятся без придыхания в начале и середине слова.  Они не выражены в конце слова,
а также в конце слога, если за ними стоит согласная
(как в \N{txe\uwave{p}mì} и \N{'o\uwave{k}trr}).  Однако, внутри фразы последний плозив, находящийся перед гласной, в естественной речи будет произнесен так как если бы слова сливались воедино, \N{oel se\uwave{t o}mum}.
Глухие плозивы более заметны в больших фразах, как в \N{oel
omum se\uwave{t}.}
\NTeri{23/12/2020}{https://naviteri.org/2020/12/mrra-tipangkxotsyip-five-little-discussions/}

\subsubsection{} \N{r} аналогичен русскому ``Р''. \N{l} четкий вначале, как в слове ``лист,'' но не как ``гулкая-Л'' в английском слове
``call, звонить''.'

\subsubsection{} Фроммер разработал научную орфографию, в которой два диграфа писались одной буквой, \N{c} для \N{ts} и
\N{g} для \N{ng}.  Актерам проще было воспринимать двухбуквенные диграфы, да и сам Фроммер использовал их в интервью и электронных письмах. Однобуквенное обозначение появлялось лишь в ранних сообщениях Фроммера.  \label{l-and-s:cg}

\subsubsection{} Окончания закрытых слогов с плозивом в конце могут вводить в заблуждение своим произношением, похожим на взрывные согласные, но это не так:
\N{tsap'alute} верно, а *\N{tsapxalute} нет.
\LNWiki{21/12/2009}{https://wiki.learnnavi.org/index.php/Canon\%23Extracts_from_various_emails}

\subsection{Гласные}
Гласная в круглых скобках является отдельной фонемой только в Рифовом На'ви.  Остальные используют оба диалекта.

\begin{center}
\begin{tabular}{ccccc}
\N{i} [i], \N{ì} [{\footnotesize I}]  & & & & (\N{ù} [ʊ]), \N{u} [u],[ʊ] \\
 & \N{e} [ɛ] & & & \N{o} [o] \\
 & & \N{ä} [æ] &  \N{a} [a] \\
\end{tabular}
\end{center}

\subsubsection{} В рифовром На'ви \N{u} всегда звучит как [u] - аналогично русской ``у'',
а \N{ù} как звук [ʊ], представляющий что-то среднее между ``о'' и ``у''. \index{Reef Na'vi!ù@\textbf{ù}}
\LNWiki{8/1/2023}{http://naviteri.org/2023/01/reef-navi-part-1-phonetics-and-phonology/}

\subsubsection{} Лесной На'ви объединяет гласные \N{u} и \N{ù}, 
по-этому они всегда звучат как [u] в открытых слогах. В закрытых допускаются оба произношения: [u]
или [ʊ].  \N{Lu} всегда звучит как [lu],
а для \N{pum} возможны оба варианта: [pum] и [pʊm].
\LNWiki{20/5/2010}{https://wiki.learnnavi.org/index.php/Canon/2010/March-June\%23The_Dual_sounds_of_.22u.22}

% 2023jan10 - don't do this for now.
%In this grammar words that are known to have \N{ù} in them in Reef
%Na'vi are spelled as such, such as \N{pùm}.  When using Forest Na'vi,
%simply treat these words as though spelled \N{pum}.

\subsubsection{} Дифтонги это \N{aw}, \N{ay}, \N{ew} и \N{ey}.
Только в дифтонгах \N{w} или \N{y} могут находиться в конце слога (\N{new}) или перед конечной согласной (\N{hawng}).  Формы слогов \N{*niw} или \N{*hoyng} иметь места не могут.

\subsection{Псевдогласные} Псевдогласная \N{rr} это слоговый растянутый звук [r̩ː], а \N{ll} слоговый растянутый звук [l̩ː]. Псевдогласная \N{rr} произносится как растянутая русская ``Р'', подобно «рычанию», а \N{ll} как растянутая русская ``Л''.

\subsection{Строение слога}
 На'ви имеет строгое, но понятное слоговое строение.

\begin{itemize*}
  \item В слоге допускается отсуствие начальной согласной (т.е., может начинаться с гласной).
  \item В слоге допускается отсутствие конечной согласной (т.е., может кончаться на гласной).
  \item Слог может начинаться с любой согласной.
  \item Группы согласных \N{f s ts} $+$ \N{p, t, k, px, tx, kx,
    m, n, ng, r, l, w, y} могут начинать слог (например, \N{tslam}, \N{ftu}).
  \item \N{P t k px tx kx ' m n l r ng} могут появляться в конце слога.
  \item \N{Ts f s h v z w y}  \textit{НЕ} могут появляться в конце слога.
  \item В конце слога не бывает групп согласных.
  \item \label{l-and-s:pseudo-no-null} Слог с псевдогласной должен начинаться с согласной или группы согласных и не должен иметь согласной в конце; это играет свою роль в смягчении (лениции)
    (\horenref{l-and-s:lenition:pseudovowel}) и склонении существительных
    (\horenref{morph:decl:pseudovowel}).
\end{itemize*}

\newpage
\noindent Наглядный пример основных правил слогообразования:

\begin{center}\footnotesize
\begin{tikzpicture}[every path/.style={>=latex},every node/.style={draw,rectangle}]
\node[rounded corners,fill=gray!20,align=center] (sos) at (-4,1) {начало\\слога};
\node[align=center] (zero-onset) at (-0.5,2.5) { нет согласной\\в начале };
\node[align=center,font=\bfseries] (onset) at (-0.25,1) {p t k px tx kx '\\ m n ng r l w y\\f v s z ts h};
\node[font=\bfseries] (clust-c1) at (-2.5,-0.5) { f s ts };
\node[align=center,font=\bfseries] (clust-c2) at (0,-0.5) {p t k px tx kx\\m n ng r l w y};
\node[align=center] (vowel) at (3.5,2) { гласная или\\дифтонг };
\node (pseudovowel) at (3.5,0.5) { псевдогласная };
\node[align=center,font=\bfseries] (final) at (6.25,2.5) {p t k px tx kx ’\\m n l r ng };
\node[align=center] (zero-coda) at (6.25,1) { нет согласной\\в конце };
\node[rounded corners,fill=gray!20,align=center] (eos) at (8.75,1) { конец\\слога };

\draw[->] (sos) edge (clust-c1);
  \draw[->] (sos) edge (onset);
  \draw[->] (sos) edge (zero-onset);
\draw[->] (clust-c1) edge (clust-c2);
\draw[->] (zero-onset) edge (vowel);
\draw[->] (clust-c2) edge (vowel);
  \draw[->] (clust-c2) edge (pseudovowel);
\draw[->] (onset) edge (vowel);
  \draw[->] (onset) edge (pseudovowel);
\draw[->] (vowel) edge (final);
\draw[->] (final) edge (eos);
\draw[->] (vowel) edge (zero-coda);
\draw[->] (pseudovowel) edge (zero-coda);
\draw[->] (zero-coda) edge (eos);
\end{tikzpicture}
\end{center}

\subsubsection{} Из-за того, что у слога могут отсуствовать начальные или конечные согласные, то возможно появление нескольких, стоящих друг за другом, гласных.
Тогда гласную считают отдельным слогом, \N{muiä} [mu.i.æ], \N{ioang}
[i.o.aŋ].

\subsubsection{} Принято последовательность ГСГ разбивать на слоги в виде Г.СГ, а не ГС.Г, так \N{tsenge} будет [tsɛ.ŋɛ], а не *[tsɛŋ.ɛ].
Звукоподражание может отменять это правило, как в \N{kxang\-ang\-ang} [k'aŋ.aŋ.aŋ],
где нужно достичь эффекта эха.

\subsubsection{} В Лесном На'ви нет долгих гласных. Это означает то, что одинаковые гласные не могут находиться друг за другом (смотри тут
\horenref{l-and-s:contract}).  В Рифовом На'Ви в некоторых случаях могут появляться долгие гласные из-за опускания произношения глоттального звука гортанной смычки (\horenref{rn:stop-elision}) и отсутствия сокращения гласных
(\horenref{rn:no-contract}). 

\subsubsection{} Сдвоенные согласные не встречаются в корнях слов, но могут появляться на границе морфем, например, производных
\N{tsukkäteng} $<$ \N{tsuk-} $+$ \N{käteng} или \href{https://ru.wikipedia.org/wiki/%D0%AD%D0%BD%D0%BA%D0%BB%D0%B8%D1%82%D0%B8%D0%BA%D0%B0}
{энклитиках}, 
когда группа слов произносится с одним ударением, воспринимаясь на слух как единое целое 
\N{Mo'atta} $<$ \N{Mo'at} $+$ \N{ta} (\horenref{l-and-s:stress:enclisis}).
% https://naviteri.org/2011/03/“receptive-ability”-and-hesitation/comment-page-1/#comment-604

\subsubsection{} Междометия могут выходить за рамки этих правил. Обычное явление для многих человеческих языков. Например, \N{oìsss} - звук гнева или 
\N{saa} - угрожающий крик.


\subsection{Ударение}
Каждого слово в На'ви есть ударение, причем не всегда предсказуемое.  Бывает так, что ударение это единственное различие между полностью одинаковыми словами, как, например, в \N{\ACC{tu}te} \E{личность}
и \N{tu\ACC{te}} \E{женщина}.

\subsubsection{} Конкретно для слова \E{женщина} ударение принято указывать отдельно, \N{tuté}.
\index{tuté@\textbf{tuté}}

\subsubsection{} В ходе словообразования акцент ударения может меняться
(\horenref{lingop:prefix:ke}, \horenref{lingop:suffix:gender}).

\subsubsection{} Все предлоги, а также некоторые союзы и частицы могут быть энклитичны.  Они теряют свое ударение, когда становятся частью слова, к которому присоединяются. Пишется соответствующе. \N{\ACC{tsa}ne} ($<$ \N{tsaw} $+$ \N{ne}),
\N{ho\ACC{ren}\-ti\-sì} ($<$ \N{ho\ACC{ren}ti} $+$ \N{sì}).
\label{l-and-s:stress:enclisis}\index{enclitics}

\subsubsection{} В составных существительных, пишущихся как одно слово, каждая из частей может сохранять свое ударение, например, \N{ti\ACC{re}a\ACC{fya}'o} \E{путь духа}.
\index{compound word!accent}

%\subsubsection{} Word stress is a property of stem words.  No matter
%how many affixes a root word takes, no secondary accents develop.

\subsection{Рифовый На'ви} Отличия рифового диалекта от лесного присутствуют как в гласных так и в согласных звуках. \index{Reef Na'vi}
\NTeri{8/1/2023}{http://naviteri.org/2023/01/reef-navi-part-1-phonetics-and-phonology/}

\subsubsection{} Взрывные согласные в начале слога, а, следовательно, и в начале слова произносятся как звонкие плозивы рифового диалекта.  Так, \N{px tx kx} → [b d g].

\begin{center}
\begin{tabular}{lll}
\N{txon}    & [don] & \E{ночь} \\
\N{hol\ACC{pxay}} & [hol.ˈbaj] & \E{число} \\
\N{kxitx}   & [git'] & \E{смерть} \\
\N{skxawng} & [sk'awŋ] (as in Forest Na'vi) & \E{дурак}
\end{tabular}
\end{center}

\noindent Обычно, данное изменение не отражается письменно. Так что \E{ночь} в рифовом диалекте будет писаться как \N{txon}, предполагая что читающий произнесет это как \N{don}.  Если хочется подчеркнуть использование рифового диалекта, то можно написать \N{don}.

Запись слова \N{tìkankxan} на рифовом диалекте может привести к двусмысленности при чтении, \N{*tìkangan}.  Чтобы исключить толкование сочетания \N{...ng...} как [ŋ] вместо [ŋg], используется интерпункт (предпочтительнее) либо дефис: \N{tìkan·gan}
или \N{tìkan-gan}.
\Omaticon{} \NTeri{15/1/2023}{http://naviteri.org/2023/01/2653/\#comment-45092}

\paragraph{} При присоединении к слову, оканчивающемуся на взрывную согласную, суффикса, который начинается с гласной, происходит изменениее произношения этой взрывной согласной в соответствии с новой слоговой структурой. Например, транскрипция слова \N{'awkx} это
[ʔawk'], но в тоже время \N{'awkxit} будет читаться как [ˈʔaw.git] голосом.
\NTeri{13/1/2023}{http://naviteri.org/2023/01/2653/}



\subsubsection{}  \N{s} и \N{ts} смягчаются, если стоят группой вместе с \N{y}.  Так что, \N{sy} будет произносится как [ʃ] и \N{tsy} как [tʃ]. [ʃ] звучит как русская ``Щ''.
По правилам образования слогов в На'ви, такое может происходить только в начале слога.

\begin{center}
\begin{tabular}{lll}
\N{syaw} & [ʃaw] & \E{звать} \\
\N{tsìsyì} & [ˈtsɪ.ʃɪ] & \E{шептать} (нпрх.гл.) \\
\end{tabular}
\end{center}

\noindent Это изменение звука в Рифовом На'ви никогда не выделяется на письме.

\subsubsection{} \index{Reef Na'vi!glottal stop elision}\label{rn:stop-elision}
В Рифовом На'ви гортанная смычка опускается, если находится между двумя гласными.  Обе гласных все равно произносятся отчетливо, даже если они одинаковые, как в \N{rä'ä} ниже,

\begin{center}
\begin{tabular}{lll}
\N{fra'u} & \N{frau} & \E{всё} \\
\N{Lo'ak} & \N{Loak} & \E{Ло'ак} (имя личное) \\
\N{rä'ä}  & \N{rää} & \E{не}
\end{tabular}
\end{center}

\noindent Это изменение в определенной степени присутствует и в Лесном На'ви. Например, с именем \N{Lo'ak} (\horenref{names-with-oa}).

Не смотря на то, что гортанная смычка остается в начале и конце слов, добавление к ним аффиксов, которые создают остановку между произношением гласных, дает возможность опустить произношение этой гортанной смычки.  Например, в Лесном На'ви фраза обозначающая \E{забавную личность}, это \N{tute a'ipu}, когда как в Рифовом На'ви это будет \N{tute aipu}, или пример с наречием \N{nìaw} которое на лесном диалекте \N{nì'aw}.
\NTeri{14/1/2023}{http://naviteri.org/2023/01/2653/\#comment-45068}

\subsubsection{}
В Рифовом На'ви \N{ä}, находящаяся в безударном слоге, может стать \N{e}.

\begin{center}
\begin{tabular}{lll}
\N{\ACC{nge}yä} & \N{ngeye} & \E{твоё} \\
\N{tä\ACC{txaw}} & \N{tedaw} & \E{возвращаться} (нпрх.гл.) \\
\N{\ACC{kä}}     & \N{kä}  & \E{идти}
\end{tabular}
\end{center}

\Omaticon

\subsection{Разговорный алфавит}
Все названия фонем, за исключением гортанной смычки \N{tìftang}, представляют информацию о том каким образом звуки используются в языке.  В них также присутствует необычная заглавная буква при написании: \index{alphabet!spoken}

\begin{center}\small
\begin{tabular}{lll}
\N{tìftang} & \N{Ì} & \N{ReR} \\
\N{A}  & \N{KeK}   & \N{'Rr} \\
\N{AW} & \N{KxeKx} & \N{Sä} \\
\N{AY} & \N{LeL}   & \N{TeT} \\
\N{Ä}  & \N{'Ll}   & \N{TxeTx} \\
\N{E}  & \N{MeM}   & \N{Tsä} \\
\N{EW} & \N{NeN}   & \N{U} \\
\N{EY} & \N{NgeNg} & \N{Vä} \\
\N{Fä} & \N{O}     & \N{Wä} \\
\N{Hä} & \N{PeP}   & \N{Yä} \\
\N{I}  & \N{PxePx} & \N{Zä} \\
\end{tabular}
\end{center}

\subsubsection{} Гласные и дифтонги произносятся так же как и пишутся.  У псевдогласных поставлена гортанная смычка, так как они всегда требуют наличия согласной перед собой (\horenref{l-and-s:pseudo-no-null}).

\subsubsection{} Названия согласных, которые не могут быть окончанием слога, образованы добавлением гласной \N{ä}, как в \N{Tsä}. А те, которые могут, используют гласную \N{e} и повторяются в конце, \N{PeP}.


\section{Смягчение}
\noindent Начальная согласная в слове может меняться из-за определенных грамматических процессов. Такое преобразование называется  ``смягчением'' или ``леницией''. Только восемь согласных подвержены смягчению.\index{lenition}\label{l-and-s:lenition}
\LanguageLog

\begin{center}
\begin{tabular}{lll}
Согласная & Смягчение & Пример \\
\N{px, tx, kx} & \N{p, t, k} & \N{\uwave{tx}ep} -> \N{mì \uwave{t}ep} \\
\N{p, t, k} & \N{f, s, h} & \N{\uwave{k}elku} -> \N{ro \uwave{h}elku} \\
\N{ts} & \N{s} & \N{\uwave{ts}mukan} -> \N{ay\uwave{s}mukan} \\
\N{’} & исчезает & \N{’eylan} -> \N{fpi eylan} \\
\end{tabular}
\end{center}

\noindent Смягчение не указывается во второй линии подстрочных описаний, где дается морфологическая разбивка. (Как в примере \ref{lenition:ex01} ниже). 

\subsection{Гортанная смычка} Гортанная смычка не смягчается, если за ней находится псевдогласная. (\N{mì 'Rrta}, а не *\N{mì Rrta}).
\index{glottal stop!lenition}\label{l-and-s:lenition:pseudovowel}
\NTeri{3/28/2012}{https://naviteri.org/2012/03/spring-vocabulary-part-1/}

\subsection{Предлоги} Некоторые предлоги вызывают смягчение: \N{fpi}, \N{ìlä}, \N{mì}, \N{nuä}, \N{ro}, \N{sko},
\N{sre} (и образованные от него \N{lisre} с \N{pxisre}), \N{wä}. Однако при использовании их в качестве суффикса (с конца), смягчение не происходит как у слова к которому они присоединены, так и у идущего следом.
\index{fpi@\textbf{fpi}!lenition}\index{ilä@\textbf{ìlä}!lenition}
\index{miì@\textbf{mì}!lenition}\index{ro@\textbf{ro}!lenition}
\index{sre@\textbf{sre}!lenition}\index{pxisre@\textbf{pxisre}!lenition}
\index{waä@\textbf{wä}!lenition}\index{nuaä@\textbf{nuä}!lenition}
\index{sko@\textbf{sko}!lenition}
\index{lenition!adpositions}\index{adpositions!lenition}
\NTeri{7/7/2010}{https://naviteri.org/2010/07/thoughts-on-ambiguity/}

\subsection{Префиксы числа} Префиксы, вызывающие смягчение, отмечаются знаком плюс, а не дефисом. Как в смягчающем префиксе множественного числа \N{ay+}. \index{lenition!number prefixes}

\subsection{Вопросительная частица} Когда используется как префикс \N{pe+}, то вызывает смягчение (\horenref{morph:pre:pe}). Например, \N{pehem} \E{какое (действие)?} Образовано от \N{kem} \E{действие, деятельность}.

\subsection{Числа}\index{lenition!numbers}
При образовании числен зависимая часть смягчается
(\horenref{numbers:dependent}), как в \N{vopey} \E{одиннадцать (8 + 3)},
но \N{pxey} это \E{три}.

\subsection{Имена собственные} Имена собственные смягчяются.

\begin{interlin} \label{lenition:ex01}
\glll Oe kelku si mì Helutral. \\
      oe kelku si mì Kelutral \\
     \I{1ед.ч.} дом делать в Дерево-дом \\
\trans{Я живу в дереве-доме}
\end{interlin}

\index{lenition!proper nouns}\NTeri{10/28/2010}{https://naviteri.org/2010/09/getting-to-know-you-part-2/}

\subsection{Рифовый На'ви} \index{Reef Na'vi!lenition}
Не смотря на то, что взрывные согласные в начале слов на рифовом диалекте пишутся как звонкие фрикативы, на них так же распространяется правило смягчения из Лесного На'ви. Так что, если \N{txon} \E{ночь} пишется на Рифовом На'ви как \N{don}, то смягченной формой все равно будет \N{ton}.
\LNWiki{8/1/2023}{http://naviteri.org/2023/01/reef-navi-part-1-phonetics-and-phonology/}

\section{Морфофонология}

\subsection{Сокращение гласных} В связи с тем, что две одинаковые гласные не могут стоять друг за другом, происходит сокращение сдвоенных гласных до одной буквы при некоторых грамматических процессах.\index{vowel!contraction}\label{l-and-s:contract}

\subsubsection{} Соединительная частица \N{-a-}, которой отмечаются прилагательные, исчезает если она присоединена к \N{a}, находящейся в начале или конце слова, как в
\N{apxa tute}, но не *\N{apxaa tute}.
\index{-a-@\textbf{-a-}!with \textbf{a} in an adjective}
\index{adjective!contraction}

\subsubsection{} Если после использования префиксов двойственного или тройственного числа получается цепочка из \N{e}, как в \N{me} $+$ \N{'eveng} $>$ *\N{meeveng} (учитывается смягчение), то две гласные объединяют в одну, \N{meveng}.
\label{l-and-s:phonotactics:nsc} \index{dual!contraction}
\index{trial!contraction}
\LNWiki{20/1/2010}{https://wiki.learnnavi.org/index.php/Canon\%23Extracts_from_various_emails}

\subsubsection{} Две гласные сокращаются в одну, если гласная в конце префикса и начале стоящего за ним слова одинакова \N{tsatan} $<$
\N{tsa-} $+$ \N{atan}, \N{fìlva} $<$ \N{fì-} $+$ \N{ìlva}
(\horenref{morph:prenoun:contraction}).\footnote{Гортанная смычка - согласная, по-этому \N{fì'ìheyu} образовано от \N{fì-} $+$ \N{'ìheyu}.}
\label{l-and-s:phonotactics:precontract}\index{prenoun!contraction}
\LNWiki{18/5/2011}{https://wiki.learnnavi.org/index.php/Canon/2011/April-December\%23Kawtseng.2C_tsapo_and_prefixes}

\subsubsection{} Сокращение не происходит с маркером неопеределенности \N{-o} или энклитичными предлогами. Две одинаковые согласные в таком случае пишутся через дефис \N{fya'o-o}
\E{какой-то путь,} \N{zekwä-äo} \E{под пальцем}.\footnote{При том что в Лесном На'ви, формально, нет долгих гласных, но их эффект возникает в этом случае. Проследите за тем чтобы обе гласных \N{ä} отчетливо произносились в таком слове как \N{zekwä-äo}.}\index{vowel!contraction!inhibited}
% https://wiki.learnnavi.org/index.php/Canon/2010/UltxaAyharyuä#Phonological_Questions

\subsubsection{} В Рифовом На'ви сокращения гласных не происходит.
Такие формы как \N{meeveng} или \N{apxaa} остаются без изменений там где Леснойм На'ви сдвоение убирает.
\index{Reef Na'vi!vowel contraction inhibited}\label{rn:no-contract}
\NTeri{13/1/2023}{http://naviteri.org/2023/01/2653/}

\subsection{Сокращение псевдо\-гласных} \index{pseudovowel!contraction}
 Форма инфиксов совершённости действия, \N{\INF{er}}
и \N{\INF{ol}}, может вызывать появление псевдогласной сразу после своего аналога среди согласных, как тут \N{*p\INF{ol}ll\ACC{txe}}.  Если это произошло в безударном слоге, то убирается псевдогласная, \N{pol\ACC{txe}}.  А если в ударном, то пропадает инфикс, \N{*\ACC{f}\INF{er}\ACC{rr}fen} $>$
\N{\ACC{frr}fen}.  Псевдогласные в односложных словах ведут себя так, как будто они безударные, \N{vol} от \N{*v\INF{ol}ll}, и \N{ner}
от \N{*n\INF{er}rr}.
\LNWiki{23/3/2010}{https://wiki.learnnavi.org/index.php/Canon/2010/March-June\%23Misc_Answers}
\NTeri{19/6/2012}{https://naviteri.org/2012/06/spring-vocabulary-part-3/}

\subsection{Инфикс отношения и вставка} Если после инфикса положительного отношения \N{\INF{ei}} сразу следует гласная \N{i}, \N{ì} либо псевдогласная, то между ними добавляется
\N{y}, \N{seiyi} $<$ \N{*s\INF{ei}i}, \N{veykrreiyìn} $<$
\N{*veykrr\INF{ei}ìn}; \N{v\INF{ei}yll} $<$ \N{*veill}.
\label{l-and-s:eiy-epenth}
\NTeri{19/6/2012}{https://naviteri.org/2012/06/spring-vocabulary-part-3/}

\subsubsection{Рифовый На'ви} \index{Reef Na'vi!affect infix epenthesis}
В Рифовом На'ви разделения гласных внутри \N{seii} на \N{seiyi} не происходит.
\NTeri{31/1/2023}{http://naviteri.org/2023/01/2653/}


\subsection{Рифовый На'ви и ассимиляция звонких плозивов} \index{Reef Na'vi!voiced stop assimilation}
Когда присутствует группа взрывных согласных как в \N{atxkxe} \E{земля}, то в Рифовом На'ви происходит регрессивное изменение звуков.
Так, начальный плозив в (*\N{atxge}) озвучивает стоящую перед ним взрывную, давая \N{adge} в результате.
Так же как \N{ekxtxu} \E{грубый} станет \N{egdu} в рифовом диалекте.
\LNWiki{8/1/2023}{http://naviteri.org/2023/01/reef-navi-part-1-phonetics-and-phonology/}

\subsection{Nasal Assimilation} In many compounds as well as in some
idioms, final nasals assimilate to the position of the following
word, as in \N{lumpe} as a variant of \N{pelun}.  Such assimilation is
not always written, which may make the etymology of a word clearer, as
in \N{zenke} instead of \N{*zengke}, from \N{zene ke}, or in the several
idioms with the verb \N{tìng} \E{give}, \N{tìng mikyun} being pronounced
\N{tìm mikyun}. \index{nasal assimilation} \label{l-and-s:nasalassim}

\subsection{Vowel Harmony} Na'vi has two instances of optional
regressive vowel harmony in verb infixes.\index{vowel!harmony}

\subsubsection{} The subjunctive future infix, \N{\INF{iyev}}, most
frequently appears as \N{\INF{ìyev}}, with backing of the first vowel.

\subsubsection{}\label{l-and-s:eng}
The vowel of the negative attitude infix, \N{\INF{äng}}, may be raised
if it is immedately followed by the vowel \N{i}, becoming \N{\INF{eng}},

\begin{interlin}
\glll Tsap'alute sengi oe. \\
      tsap'alute s\INF{äng}i oe \\
      apology do\INF{\I{neg.aff}} \I{1sg} \\
\trans{I apologize.}
\end{interlin}

\Ultxa{2/10/2010}{https://wiki.learnnavi.org/index.php/Canon/2010/UltxaAyharyu\%C3\%A4\%23.C3.A4ng.2Feng}

\subsection{Elision} In rapid speech final \N{-e} is frequently elided
when the following word starts in a vowel.  \Npawl{Kìyevam$\not$e
ult$\not$e Eywa ngahu}.  This is not indicated in writing.\index{elision}
When the final \N{-e} is in a monosyllable (\N{ke, sre}), or when it is
stressed (\N{tuté}), it is not elided.
\LNForum{25/10/2022}{https://forum.learnnavi.org/language-updates/a-collection-of-questions-answered/}

\subsubsection{} The vowel \N{ì} in \N{mì}, \N{sì} and the adverb
prefix \N{nì-} drops before the plural prefix \N{ay+}, though there is
no change in writing.  So, \N{nìayfo} \E{like them} is pronounced as
\N{nayfo}. \label{l-and-s:elision-i}
\index{miì@\textbf{mì}!elision with plural}
\index{siì@\textbf{sì}!elision with plural}
\index{niì-@\textbf{nì-}!elision with plural}
\NTeri{1/7/2010}{https://naviteri.org/2010/07/thoughts-on-ambiguity/}

\subsubsection{} The vowel in \N{nì-} will usually elide before a
stressed \N{e}, as in \N{nì-} + \N{etrìp} > \N{netrìp}. If the \N{e}
is unstressed, it will usually, though not always, elide, \N{nì-} +
\N{eyawr} > \N{nìyawr}. One exception: \N{nìean} instead of the
expected \N{*nean}.
\index{niì-@\textbf{nì-}!elision before e}
\LNForum{9/8/2017}{https://forum.learnnavi.org/language-updates/if-ni-will-attached-at-e/}

\subsection{Other Phonetic Processes}

\subsubsection{Names with -o'a-} \label{names-with-oa}
In colloquial speech names containing the sequence \N{o'a} may
eliminate the glottal stop, such
as \N{Mo'at} \textasciitilde{} \N{Moat} 
and \N{Lo'ak} \textasciitilde \N{Loak}.
\NTeri{1/3/2017}{https://naviteri.org/2017/02/ayioang-amip-si-ayu-alahe-new-animals-and-other-things/\#comment-26401}


\section{Orthographic Conventions}
\noindent Na'vi in general follows the spelling, punctuation and
capitalization habits of English, but there are a few differences.

\subsection{Proper Names} When taking lexical prefixes
(\horenref{lingop:affixes}), proper names retain their original
capitalization, as in \N{lì'fya le\uwave{Na'vi}}.

\subsection{Quotation} Direct quotes are not punctuated with quotation
marks in Na'vi.  Instead it relies on the quotation particles
\N{san\dots sìk} (see \horenref{syn:direct-quote}).
\index{quotation!punctuation}

\subsection{Etymological Spelling} In addition to the occasional
spelling of nasals to reflect etymo\-logy (\horenref{l-and-s:nasalassim}),
there are a few grammatical processes which result in spelling that
reflects the grammar more than the pronunciation.

\subsubsection{} The first person pronoun root \N{oe}, though
pronounced \N{we} when taking a suffix, retains the original spelling
(\horenref{morph:pron:oe-we}).

\subsubsection{} Before words starting with \N{y} the plural prefix
\N{ay+} is unchanged, \N{ayyerik}.
\LNWiki{18/4/2010}{https://wiki.learnnavi.org/index.php/Canon/2010/March-June\%23ay.2Byerik}

\subsection{Attributive Phrase Hyphenation} Certain short attributive
phrases are written with hy\-phens joining the elements.

\subsubsection{} Attributive phrases of color using \N{na} \E{like}
are hyphenated, \N{fìsyulang aean-na-ta'leng} \E{this skin-blue
flower} (\horenref{syn:attr:na}).

\subsubsection{} Participles of \N{si} construction verbs are also
hyphenated, \N{srung-susia tute} \E{a helping person}
(\horenref{syn:participle:si-const}).

